% ====   Order 30   ====
\section{Groups of Order 30}
Let \(G\) be a group of order \(30 = 2 \cdot 3 \cdot 5\), and let \(n_3\) and \(n_5\) denote the number of Sylow
3-subgroups and Sylow 5-subgroups of \(G\) respectively.
Then by Theorem~\ref{thm:sylow3}:
\[n_3 = 1 \ \text{or} \ 10 \quad \text{and} \quad n_5 = 1 \ \text{or} \ 6\]
If \(n_3 = 10\), then there are 20 elements of order 3, and if \(n_5 = 6\) then
there are 24 elements of order 5 in \(G\).
\(G\) only has 30 elements, so then either:
\[n_3 = 1 \ \text{and} \ n_5 = 6, \quad n_3 = 10 \ \text{and} \ n_5 = 1 \quad \text{or} \quad n_3 = n_5 = 1\]
So if \(T\) is a Sylow 3-subgroup of \(G\) and \(F\) is a Sylow 5-subgroup, then at least one must be normal in \(G\).
So \(T \nrmsgp G\) or \(F \nrmsgp G\) or both.

Let \(H = TF\) and by Lagrange's Theorem, \(T \cap F = \bm{1}\), hence \(|H| = 15\) by Lemma~\ref{lem:setprodorder}.
We know from our classification of groups of order \(pq\) that \(H \cong C_{15}\).
Notice that a Sylow 2-subgroup \(K \leqslant G\) has order 2, so \(K \cong C_2\).
Let \(\langle t \rangle = K\) and \(\langle v \rangle = H\).
By the same argument as above, \(H \cap K = \bm{1}\) and \(|HK| = 30\).
Hence \(G = HK\).

Because \(|H| = 15 = \frac{30}{2}\), the index of \(H\) in \(G\) is 2, and we know a subgroup of index 2 is normal, so
\(H \nrmsgp G\).
Moreover, \(G = H \rtimes K\).

By Lemma~\ref{lem:aut}:
\[\Aut{C_{15}} = \units{15} \cong {(\Zn{3} \times \Zn{5})}^\times \cong \units{3} \times \units{5} \cong C_2 \times C_4\]

A homomorphism, \(\psi:C_2 \to C_2 \times C_4\) preserves element order and we know \(\psi\) is determined by it's effect on a generator.
So then \(t\psi\) has four possibilities: either the identity, or one of the three elements
of order 2.

Additionally, \(\psi\) preserves the Sylow subgroups of \(H\).
So write \(H = \langle v^3 \rangle \times \langle v^5 \rangle\), the direct product of its Sylow subgroups.

So the action of \(K\) on \(H\) is either trivial or by inversion on each of the Sylow subgroups of \(H\), giving us 4
possibilities:

\begin{enumerate}[\bfseries{Case} 1:]
    \item Trivial action on both Sylow subgroups.

        In this case, because the action is trivial on all of \(H\), we recover the direct product, \(G = H \times K
        \cong C_{30}\).

    \item Inversion on both Sylow subgroups.

        Here, \(K\) acts on all of \(H\), so we obtain:
        \[G = \langle\, v,\,t \mid v^{15} = t^2 = 1,\ t^{-1}vt = v^{-1}\,\rangle\]
        which we recognise as \(D_{30}\).

    \item Inversion on \(\langle v^5 \rangle\).

        We know already, from our classification of groups of order \(2p\), that \(C_3 \rtimes C_2 \cong D_6\).
        \[G = \langle v^3 \rangle \times (\langle v^5 \rangle \rtimes K) \cong C_5 \times D_6\]
        So then because the action on \(\langle v^3 \rangle\) is trivial:

    \item Inversion on \(\langle v^3 \rangle\).

        Similar to above, we obtain:
        \[G = \langle v^5 \rangle \times (\langle v^3 \rangle \rtimes K) \cong C_3 \times D_{10}\]
\end{enumerate}

Hence any group of order 30 is isomorphic to one of:
\[
    C_{30}, \quad%
    D_{15}, \quad%
    C_5 \times D_6, \quad \text{or} \quad%
    C_3 \times D_{10}
\]

