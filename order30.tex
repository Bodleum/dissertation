% ====   Order 30   ====
\section{Groups of Order 30}
Let \(G\) be a group of order \(30 = 2 \cdot 3 \cdot 5\), and let \(n_3\) and \(n_5\) denote the number of Sylow
3-subgroups and Sylow 5-subgroups of \(G\) respectively.
Then by Theorem~\ref{thm:sylow3}:
\[n_3 = 1 \ \text{or} \ 10 \quad \text{and} \quad n_5 = 1 \ \text{or} \ 6\]
If \(n_3 = 10\), then there are 20 elements of order 3, and if \(n_5 = 6\) then
there are 24 elements of order 5 in \(G\).
\(G\) only has 30 elements, so then either:
\[n_3 = 1 \ \text{and} \ n_5 = 6, \quad n_3 = 10 \ \text{and} \ n_5 = 1 \quad \text{or} \quad n_3 = n_5 = 1\]
So if \(T\) is a Sylow 3-subgroup of \(G\) and \(F\) is a Sylow 5-subgroup, then at least one must be normal in \(G\).
So \(T \nrmsgp G\) or \(F \nrmsgp G\) or both.

Let \(H = TF\) and by Lagrange's Theorem, \(T \cap F = \bm{1}\), hence \(|H| = 15\) by Lemma~\ref{lem:setprodorder}.
We know from our classification of groups of order \(pq\) that \(H \cong C_{15}\).
Notice that a Sylow 2-subgroup \(K \leqslant G\) has order 2, so \(K \cong C_2\).
Let \(\langle t \rangle = K\) and \(\langle v \rangle = H\).
By the same argument as above, \(H \cap K = \bm{1}\) and \(|HK| = 30\).
Hence \(G = HK\).

Because \(|H| = 15 = \frac{30}{2}\), the index of \(H\) in \(G\) is 2, and we know a subgroup of index 2 is normal, so
\(H \nrmsgp G\).
Moreover, \(G = H \rtimes K\).

By Lemma~\ref{lem:aut}:
\[\Aut{C_{15}} = \units{15} \cong {(\Zn{3} \times \Zn{5})}^\times \cong \units{3} \times \units{5} \cong C_2 \times C_4\]

Let \(\langle x, y \rangle = C_2 \times C_4\).
A homomorphism, \(\psi:C_2 \to C_2 \times C_4\) preserves element order, and there are 3 elements of order 2 in \(C_2
\times C_4\): \((x, 1)\), \((1, y^2)\) and \((x, y^2)\).
We know \(\psi\) is determined by it's effect on a generator, so if \(\langle z \rangle = K\) then \(z\psi\) has four
possibilities:
\[z\psi = (1,\,1),\ (x,\,1),\ (1,\,y^2) \quad \text{or} \quad (x,\,x^2)\]
So we know there are 4 groups of order 30 up to isomorphism, but which 4?

If \(z\psi = (1,\,1)\) then the action of \(K\) on \(H\) is trivial, so we obtain the direct product:
\[G \cong C_{15} \times C_2 \cong C_{30}\]
All other possibilities for \(z\psi\) have order 2, and we have seen before that the only possible action of order 2 is
inversion.

Notice that \((x,\,y^2)\) has both non-trivial elements of order 2, thus \(K\) acts on all of \(H\):
\[G \cong \langle\, v, t \mid v^{15} = t^2 = 1,\ t^{-1}vt = v^{-1}\,\rangle\]
which we recognise as \(D_{15}\).

For the final two cases, it is helpful to consider \(H\) as a direct product.
Without loss of generality, take \(H \cong C_5 \times C_3\).

If \(z\psi = (1,\,y^2)\), then it acts trivially on \(C_5\), and with inversion on \(C_3\).
So we can write:
\[G \cong C_5 \times (C_3 \rtimes C_2)\]
We have seen already that this semidirect product \(C_3 \rtimes C_2\), is isomorphic to \(D_6\), so we can conclude:
\[G \cong C_5 \times D_6\]

If \(z\psi = (x,\,1)\), then it acts with inversion on \(C_5\), and trivially on \(C_3\).
Hence:
\[G \cong C_3 \times (C_5 \rtimes C_2)\]
Once again, this semidirect product is isomorphic to \(D_{10}\), so we have:
\[G \cong C_3 \times D_{10}\]

Hence any group of order 30 is isomorphic to one of:
\[
    C_{30}, \quad%
    D_{15}, \quad%
    C_5 \times D_6, \quad \text{or} \quad%
    C_3 \times D_{10}
\]

