% ====   Order 30   ====
\section{Groups of Order 30}
Let \(G\) be a group of order \(30 = 2 \cdot 3 \cdot 5\), and let \(n_3\) and \(n_5\) denote the number of Sylow
3-subgroups and Sylow 5-subgroups of \(G\) respectively.
Then by Theorem~\ref{thm:sylow3}:
\[n_3 = 1 \ \text{or} \ 10 \quad \text{and} \quad n_5 = 1 \ \text{or} \ 6\]
If \(n_3 = 10\), then there are 20 elements of order 3, and if \(n_5 = 6\) then
there are 24 elements of order 5 in \(G\).
\(G\) only has 30 elements, so then either:
\[n_3 = 1 \ \text{and} \ n_5 = 6, \quad n_3 = 10 \ \text{and} \ n_5 = 1 \quad \text{or} \quad n_3 = n_5 = 1\]
So if \(T\) is a Sylow 3-subgroup of \(G\) and \(F\) is a Sylow 5-subgroup, then at least one must be normal in \(G\).
So \(T \nrmsgp G\) or \(F \nrmsgp G\) or both.

Let \(H = TF\) and by Lagrange's Theorem, \(T \cap F = \bm{1}\), hence \(|H| = 15\) by Lemma~\ref{lem:setprodorder}.
We know from our classification of groups of order \(pq\) that \(H \cong C_{15}\).
Notice that a Sylow 2-subgroup \(K \leqslant G\) has order 2, so \(K \cong C_2\).
By the same argument as above, \(H \cap K = \bm{1}\) and \(|HK| = 30\).
Hence \(G = HK\).

Because \(|H| = 15 = \frac{30}{2}\), the index of \(H\) in \(G\) is 2, and we know a subgroup of index 2 is normal, so
\(H \nrmsgp G\).
Moreover, \(G = H \rtimes K\).

By Lemma~\ref{lem:aut}:
\[\Aut{C_{15}} = \units{15} \cong {(\Zn{3} \times \Zn{5})}^\times \cong \units{3} \times \units{5} \cong C_2 \times C_4\]

Let \(\langle x, y \rangle = C_2 \times C_4\).
A homomorphism, \(\psi:C_2 \to C_2 \times C_4\) preserves element order, and there are 3 elements of order 2 in \(C_2
\times C_4\): \((x, 1)\), \((1, y^2)\) and \((x, y^2)\).
We know \(\psi\) is determined by it's effect on a generator, so if \(\langle z \rangle = K\) then \(z\psi\) has four
possibilities:

\begin{enumerate}[\bfseries{Case} 1:]
    \item \(z\psi = (1, 1)\).

        When \(z\psi = (1, 1)\), then \(\psi\) is the trivial homomorphism, and
        so we obtain:
        \[G \cong C_2 \times C_{15} \cong C_{30}\]

    \item \(z\psi = (x, 1)\).
    \item \(z\psi = (1, y^2)\).
    \item \(z\psi = (x, y^2)\).
\end{enumerate}

% TODO: Either classify these as given, or provide the groups and prove they
% are pairwise non-isom.

