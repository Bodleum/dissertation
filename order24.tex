% ====   Order 24   ====
\section{Groups of Order 24}
Let \(G\) be a group of order 24, and let \(H\) be a Sylow 3-subgroup of \(G\), so \(H \cong C_3\).
Let \(T\) by a Sylow 2-subgroup of \(G\), so \(T\) has order 8.
By Lagrange's Theorem, \(H \cap T = \bm{1}\) and then applying Lemma~\ref{lem:setprodorder}, \(|HT| = 24\).
Now let \(n_3\) denote the number of Sylow 3-subgroups, and by Sylow's~Theorems:
\[n_3 \equiv 1\ \text{mod 3} \quad \text{and} \quad n_3 \mid 8\]
Hence \(n_3\) is either 1 or 4.

If \(n_3 = 1\), then \(H\) is normal in \(G\).
Thus \(G = H \rtimes T\).
We'll want a homomorphism \(\psi: \Aut{T} \to H\)
From our classification of groups of order 8, we have 5 possibilities:

\begin{enumerate}[\bfseries{Case} 1:]
    \item \(T \cong C_8\) i.e. \(G \cong C_3 \rtimes C_8\)
        
        1 group
    \item \(T \cong (C_4 \times C_2)\) i.e. \(G \cong C_3 \rtimes (C_4 \times C_2)\)
    
        2 groups
    \item \(T \cong (C_2 \times C_2 \times C_2)\) i.e. \(G \cong C_3 \rtimes (C_2 \times C_2 \times C_2)\)

        1 group
    \item \(T \cong D_8\) i.e. \(G \cong C_3 \rtimes D_8\)
        
        2 groups
    \item \(T \cong Q_8\) i.e. \(G \cong C_3 \rtimes Q_8\)

        1 group --- binary dihedral
\end{enumerate}

If \(n_3 = 4\) then \(H\) is not normal.
Now let \(G\) act by conjugation on the set of its Sylow 3-subgroups, \(\Omega = \{\,H \mid H\ \text{is a Sylow
3-subgroup of}\ G\,\}\):
\[H^x = x^{-1}Hx = \{\,x^{-1}hx \mid h \in H\,\} \quad \text{for}\ x \in G\]
This is indeed a group action because for \(x,\,y \in G\):
\[{(H^x)}^y = {(x^{-1}Hx)}^y = (y^{-1}x^{-1})H(xy) = {(xy)}^{-1}H(xy) = H^{(xy)}\]
and:
\[H^1 = 1^{-1}H1 = H\]
Hence we obtain a homomorphism \(\rho:G \to S_4\).
The group \(G\) is acting on a set of 4 elements, so the kernel must have order dividing 6.
SOMETHING TO DO WITH NORMALISERS\@.

The kernel cannot be of order 3, because \(G\) has no normal subgroup of order 3 (because kernels of homomorphism are
normal in the domain group).
Likewise the kernel cannot be of order 6 because every group of order 6 has a unique Sylow 3-subgroup, which would be
normal in \(G\) as well, by Lemma~\ref{lem:char}.
Hence the kernel must have order 1 or 2.

If the kernel is of order 1, then \(\rho\) is actually an isomorphism, so \(G \cong S_4\).

If the kernel is of order 2, then we know that \(G/\ker{\rho} \cong \im{\rho}\), so then \(\im{\rho}\) must have order
12.
It also cannot have a normal Sylow 3-subgroup, so looking at our classification of groups of order 12, this must be
isomorphic to \(A_4\).
We know that \(A_4\) has a normal subgroup of order 4, and so by the Correspondence~Theorem, \(G\) must contain a normal
subgroup of order 8, say \(T\).
By Lagrange's~Theorem and Lemma~\ref{lem:setprodorder}, we can conclude that \(G = T \rtimes H\).
Again, we have 5 cases, but this time we'll exclude the trivial homomorphism, because that will just give us the direct
product which we have already seen:


\begin{enumerate}[\bfseries{Case} 1:]
    \item \(T \cong C_8\) i.e. \(G \cong C_8 \rtimes C_3\)

        An automorphism of \(T\), \(\varphi\), maps generators to generators, so say \(\langle x \rangle = T\).
        Then \(x\varphi\) could be \(x\), \(x^3\), \(x^5\) or \(x^7\).
        Notice that each of these, apart from \(\varphi:x \mapsto x\), has order 2.
        Hence, and Lagrange's~Theorem tells us that there are no non-trivial homomorphisms \(\psi:H \to \Aut{T}\).
        As a bonus: \(\Aut{C_8} \cong V_4\).

    \item \(T \cong (C_4 \times C_2)\) i.e. \(G \cong (C_4 \times C_2) \rtimes C_3\)

        An automorphism of \(T\), \(\psi\) preserves element order, so if \(\langle\,x,\,y \mid x^4 = y^2 = 1,\ xy =
        yx\,\rangle = T\), then \(x\psi\) must be of order 4, and \(y\psi\) must be of order 2.
        Moreover, \(y\psi\) cannot be in \(\langle\,x\psi\,\rangle\) because \(\psi\) is injective.

        So we are reduced to 2 possible choices for \(y\psi\), and 4 possible choices for \(x\psi\).
        Because an automorphism is determined by it's effect on generators, this gives us 8 possible automorphisms.
        Hence \(|\Aut{T}| = 8\).
        Lagrange's~Theorem tells us that there are no non-trivial homomorphisms \(\psi:H \to \Aut{T}\).

    \item \(T \cong (C_2 \times C_2 \times C_2)\) i.e. \(G \cong (C_2 \times C_2 \times C_2) \rtimes C_3\)

        To determine \(\Aut{T}\) it is helpful to think of \(C_2\) as the finite field with two elements.
        Then \(T\) is isomorphic a 3 dimensional vector space over two elements.
        So an automorphism of that vector space is just any linear map, with non-zero determinant.
        Thus, \(\Aut{T} \cong \GL_3(2)\).

        We can determine that \(|\GL_3(2)| = 168 = 2^3 \cdot 3 \cdot 7\), so \(\Aut T\) has a Sylow 3-subgroup of order
        3, isomorphic to \(C_3\).

        Sylow's~Theorems tells us that all subgroups of order 3 are conjugate, so Lemma~\ref{lem:semiisom} tells us
        there is only one unique action (up to isomorphism) of \(H\) on \(T\).
        So let's pick a homomorphism, \(\psi\), which will let us easily classify the resulting semidirect product.

        Write \(T = A \times B\) where \(A \cong C_2\) and \(B \cong C_2 \times C_2\).
        Then let \(\psi\) map \(H\) to the subgroup generated by the automorphism which fixes \(A\) and permutes the
        non-identity elements of \(B\) in a 3-cycle.
        This automorphism has order 3 by construction, so we can write:
        \[G \cong C_2 \times (V_4 \rtimes C_2)\]
        We know already that \(V_4 \rtimes C_2 \cong A_4\) so \(G \cong C_2 \times A_4\).

    \item \(T \cong D_8\) i.e. \(G \cong D_8 \rtimes C_3\)

        If we say \(\langle\,s,\,r \mid s^2 = r^4 = 1,\ s^{-1}rs = r^{-1}\,\rangle = T\), then consider two
        automorphisms, 

    \item \(T \cong Q_8\) i.e. \(G \cong Q_8 \rtimes C_3\)

        Firstly, because of the multiplication structure of the quaternions, the image of \(k\) is determined by the
        images of \(i\) and \(j\); it is forced.
        This reduces the possibilities for an automorphism.
        Additionally, \(\pm 1\) are fixed by an automorphism, because they are the only elements of their order.
        So an automorphism could send \(i\) to any of the remaining 6 elements of order 4.
        The image of \(j\) cannot be in the subgroup generated by the image of \(i\), otherwise we wouldn't have an
        automorphism.
        Thus there are 4 choices for the image of \(j\), giving us 24 possible automorphisms altogether.

        So \(\Aut{T}\) will have a Sylow subgroup of order 3.

        1 group --- binary tetrahedral
\end{enumerate}


