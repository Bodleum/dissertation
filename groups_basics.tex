% ====   Groups Basics   ====
\section{Basics of Groups}

\begin{definition}
    A \emph{permutation} of a set \(X\) is a bijection from \(X\) to \(X\).
    The \emph{symmetric group} \(X\) is the set of all permutations of \(X\) under composition.
    We write \(\Sym{X}\) to denote this.
    It is easy to show \(\Sym{X}\) is a group.
\end{definition}

\begin{definition}
    If \(G\) is a group, and \(H \subseteq G\), then \(H\) is a \emph{subgroup} of \(G\) if it is a group in its own right with
    the multiplication from \(G\).
    We write \(H \leqslant G\) to mean \(H\) is a subgroup of \(G\).

    If \(H\) is closed under \emph{conjugation}, i.e.\ for all \(g \in G\) and \(h \in H\), \(g^{-1}hg \in H\), then we
    say \(H\) is a \emph{normal subgroup} of \(G\).
    We write \(H \nrmsgp G\) to mean \(H\) is a normal subgroup of \(G\).
\end{definition}

\begin{definition}
    If \(G\) is a group and \(X \subseteq G\), then the \emph{subgroup generated by \(X\)} is the intersection of all
    subgroups of \(G\) containing \(X\).
    This in denoted \(\langle X \rangle\).
    The proof that \(\langle X \rangle\) is a subgroup of \(G\) is omitted.
    The elements of \(X\) are called \emph{generators} of \(G\).
\end{definition}

\begin{definition}
    If \(G\) is a group with subgroup \(H\) then the \emph{right coset} of \(H\) in \(G\) with representative \(g \in
    G\) is:
    \[Hg = \{\,hg \mid h \in H\,\}\]
\end{definition}

\begin{definition}
    The \emph{order} of a group, \(G\), is the number of elements in \(G\), denoted \(|G|\).
    The \emph{order} of an element \(g \in G\) is the smallest \(i \in \N\) such that \(g^i = 1\).
\end{definition}


% ====   Theorems   ====
\subsection{Theorems}

This collection of theorems is extremely useful for describing group structures.
Hopefully these ring some bells.
We will use them without proof.

\begin{theorem}[Lagrange's Theorem for Finite Groups]
    Let \(G\) be a finite group with subgroup \(H\).
    Then \(|H|\) divides \(|G|\).
    In particular, the order of an element of \(G\) must divide \(|G|\).
\end{theorem}

For the Sylow Theorems, let \(G\) be a group of order \(p^n m\) where \(p\) is a prime and \(p\nmid m\).
\begin{theorem}[\nth{1} Sylow Theorem]\label{thm:sylow1}
    \(G\) has a Sylow \(p\)-subgroup, i.e.\ a subgroup of order \(p^n\).
\end{theorem}
\begin{theorem}[\nth{2} Sylow Theorem]\label{thm:sylow2}
    All Sylow \(p\)-subgroups of \(G\) are conjugate to each other.
    In particular, if \(G\) has a unique Sylow \(p\)-subgroup, then it is a normal subgroup.
\end{theorem}
\begin{theorem}[\nth{3} Sylow Theorem]\label{thm:sylow3}
    Let \(n_p\) denote the number of Sylow \(p\)-subgroups of \(G\).
    Then:
    \begin{enumerate}[(i)]
        \item \(n_p \mid m\)
        \item \(n_p\equiv 1\) \mod{p}\)
    \end{enumerate}
\end{theorem}

\begin{theorem}[\nth{1} Isomorphism Theorem]\label{thm:iso1}
    For groups \(G\) and \(H\), and a homomorphism \(\psi:G \to H\):
    \[G / \ker{\psi} \cong \im{\psi}\]
\end{theorem}

\begin{theorem}[\nth{2} Isomorphism Theorem]\label{thm:iso2}
    Let \(G\) be a group, with subgroup \(H\) and normal subgroup \(N\).
    Then:
    \begin{enumerate}[(i)]
        \item \(H \cap N\) is a normal subgroup of \(G\)
        \item \(HN\) is a subgroup of \(G\) 
        \item \(H / (H \cap N) \cong (HN) / N\)
    \end{enumerate}
\end{theorem}

\begin{theorem}[\nth{3} Isomorphism Theorem]\label{thm:iso3}
    Let \(G\) be a group, with normal subgroups \(H\) and \(N\), such that \(H \leqslant N \leqslant G\).
    Then:
    \begin{enumerate}[(i)]
        \item \((N / H)\) is a normal subgroup of \(G / H\)
        \item \((G / H) / (N / H) \cong (G / H)\)
    \end{enumerate}
\end{theorem}
