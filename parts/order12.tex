% ====   Order 12   ====
\section{Groups of order 12}
We have seen that groups of order 12 have slightly different behaviour to groups of order \(4q\) in general, and we will
need this classification in order to classify groups of order 24.

Let \(G\) be a group of order \(12 = 2^2 \cdot 3\), and \(n_3\) and \(n_2\) denote the number of Sylow 3-subgroups and
Sylow 2-subgroups of G respectively.
By Sylow's~Theorems:
\[n_2 \equiv 1 \pmod{2} \quad \text{and} \quad n_2 \mid 3\]
\[n_3 \equiv 1 \pmod{3} \quad \text{and} \quad n_3 \mid 4\]
Hence:
\[n_2 = 1 \ \text{or} \ 3\]
\[n_3 = 1 \ \text{or} \ 4\]
Let \(H\) be a Sylow 2-subgroup and \(K\) be a Sylow 3-subgroup of \(G\), generated by \(x\).

Lagrange's Theorem tells us \(H\) has elements of order 1, 2, and 4, and \(K\) has elements of order 1 and 3.
Hence \(H \cap K = \bm{1}\).
Lemma~\ref{lem:setprodorder} tells us:
\[|HK| = |H| \cdot |K| = 12\]
Hence \(G = HK\), \(H \nrmsgp G\), and \(H \cap K = \bm{1}\).

Since an automorphism, \(\varphi\), must map generators to generators, \(\Aut{C_4} \cong C_2\) because \(C_4\) has two
generators.
An automorphism of \(V_4\) corresponds to a permutation of the three
non-identity elements, hence \(\Aut{V_4} \cong S_3\).

If we consider \(G\) where \(K \nrmsgp G\), i.e. \(G = K \rtimes H\), then we have two cases:

\begin{enumerate}[\bfseries{Case} 1:]
    \item \(H \cong C_4\) i.e. \(G \cong C_3 \rtimes C_4\).

        Let \(H = \gen{y}\).

        We know \(\Aut{C_3} \cong C_2\) so a homomorphism \(\psi\) maps \(H\) to the trivial group or to \(\gen{\beta:x
        \mapsto x^{-1}}\).

        If \(H\psi = \bm{1}\) then \(G = K \times H \cong C_3 \times C_4 \cong C_{12}\).

        If \(H\psi = \gen{\beta}\) then we have:
        \[G = \gen{x, y \mid x^3 = y^4 = 1,\ y^{-1}xy = x^{-1}}\]
        Now let \(a = xy^2\).
        And remember, \(y^{-1}xy = x^{-1}\) means \(x\) commutes with \(y^2\).
        So now:
        \[a^3 = xy^2xy^2xy^2 = x^3y^6 = y^2\]
        and:
        \[y^{-1}ay = y^{-1}xy^2y = (y^{-1}xy)y^2 = x^{-1}y^2 = y^2x^{-1} = a^{-1}\]
        So:
        \[G = \gen{a,\,y \mid a^6 = 1,\ a^3 = y^2,\ y^{-1}ay = a{-1}}\]
        which we recognise as \(\Dic_{12}\).
        This group is also sometimes denoted by \(T\).

    \item \(H \cong V_4\) i.e. \(G \cong C_3 \rtimes (C_2 \times C_2)\).

        If \(\psi:H \to \Aut{K}\) is trivial then we obtain the direct product again.
        We know \(\Aut{K} \cong C_2\), so there are 3 choices of elements in \(H\) to send to it, but they are all
        equivalent up to isomorphism, by Lemma~\ref{lem:semiisom}, taking \(\alpha\) to be the identity map.

        We know that \(H / \im{\psi} \cong \ker{\psi}\), so \(\ker{\psi}\) must be isomorphic to \(C_2\).
        Pick \(z\) so that it generates the kernel, and so the remaining generator, \(y\) is not in the kernel.
        Then:
        \[G = \gen{x,\,y,\,z \mid x^3 = y^2 = z^2 = 1,\ yz = zy,\ xz = zx,\ y^{-1}xy = x}\]
        Let \(a = xz\).
        So:
        \[a^3 = x^3z^3 = z\]
        and:
        \[ y^{-1}ay = y^{-1}xzy = y^{-1}xyz = x^{-1}z = x^2z = a^{-1}\]
        Hence:
        \[G = \gen{y,\,a \mid a^6 = y^2 = 1,\ y^{-1}ay = a^{-1}} \cong D_{12}\]
\end{enumerate}

Instead, if \(G\) has 4 Sylow 3-subgroups, then there are 8 elements of order 3 in \(G\).
So the remaining 4 must form the Sylow 2-subgroup, hence it is normal.

\begin{enumerate}[\bfseries{Case} 1:]
    \item \(H \cong C_4\) i.e. \(G \cong C_4 \rtimes C_3\).

        Let \(H = \gen{y}\).

        A homomorphism \(\psi:K \to \Aut{H} \cong C_2\), preserves order and together with Lagrange's Theorem means that
        the only possibility for \(\psi\) is trivial, i.e. \(K\psi = \bm{1}\).

        Hence \(G \cong C_4 \times C_3 \cong C_{12}\).

    \item \(H \cong V_4\) i.e. \(G \cong (C_2 \times C_2) \rtimes C_3\).

        Let \(H = \gen{y,\,z}\).

        A trivial homomorphism \(K\psi = \bm{1}\) yields the direct product.
        What non-trivial homomorphisms are there?
        The automorphism group, \(\Aut{H} \cong S_3\) is of order 6, and so has a unique subgroup of order 3, by
        Sylow's~Theorems.
        We know that a homomorphism \(\psi: K \to \Aut{H}\) is determined by where it sends the generator
        \(x\), so for \(\psi\) to be non-trivial, it must send \(x\) to an element of order 3 in \(\Aut{H}\).

        There are 2 such elements.
        Because \(\Aut{H} \cong S_3\), we will think of them as the permutations of order 3 of the set \(\{1, 2, 3\}\).
        Denote them \(a = (1\ 2\ 3)\) and \(b = (1\ 3\ 2)\).
        Notice that \(b = a^{-1}\), so we have homomorphisms:
        \[\psi_1:x \mapsto a \quad \text{and} \quad \psi_2:x \mapsto a^{-1}\]
        It appears we have 2 choices, but this is not the case.
        The inverse map, \(\beta:x \mapsto x^{-1}\), is an automorphism of K, and so by Lemma~\ref{lem:semiisom}, the
        corresponding semidirect products of \(\psi_1\) and \(\psi_2\) are isomorphic.
        Hence (up to isomorphism) there is one non-trivial homomorphism \(\psi:K \to \Aut{H}\).
        So \(x \in K\) acts by permuting the 3 non-identity elements of \(H\).

        We will show that in this case, \(G \cong A_4\).
        First, let's check \(A_4\) has the same subgroup structure as \(G\).
        There is a subgroup isomorphic to \(C_3\) in \(A_4\), generated by the 3-cycle \((1\ 2\ 3)\):
        \[\bar{K} = \gen{(1\ 2\ 3)}\]
        We can also find a subgroup isomorphic to \(V_4\):
        \[\bar{H} = \{\,1,\, (1\ 2)(3\ 4),\, (1\ 3)(2\ 4),\, (1\ 4)(2\ 3)\,\}\]
        Indeed, we can check that \(\bar{H}\) is normal in \(A_4\).
        We can see that \(\bar{H} \cap \bar{K} = \bm{1}\) because \(\bar{H}\) contains no 3-cycles, and that
        \(\bar{H}\bar{K} = A_4\).
        So we can conclude that \(A_4 = \bar{H} \rtimes \bar{K}\).

        Let's investigate how conjugation behaves.
        If we let \(\alpha = (1\ 2)(3\ 4)\), \(\beta = (1\ 4)(2\ 3)\) and \(\gamma = (1\ 2\ 3)\), then we can write an
        element of \(A_4\) as \(\alpha^i\beta^j\gamma^k\) for some \(i\), \(j\) and \(k\).
        Define \(\phi:A_4 \to G\) by \(\phi:\alpha^i\beta^j\gamma^k \mapsto x^{i}y^{j}z^{k}\).
        Then:
        \[\beta\phi = (\gamma^{-1}\alpha\gamma)\phi = c^{-1}ac = b\]
        So conjugation acts in the same way.
        Hence we can conclude that \(G \cong A_4\).
\end{enumerate}


So a group \(G\) of order 12 is isomorphic to one of:
\[
    C_{12}, \quad%
    C_2 \times C_6, \quad%
    A_4, \quad%
    D_{12}, \quad \text{or} \quad%
    \Dic_{12}
\]

