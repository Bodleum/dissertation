% ====   Order 24   ====
\section{Groups of Order 24}
\begin{theorem}\label{thm:24}
    Any group of order 24 is isomorphic to one of:
    \begin{displaymath}
    \begin{array}{c@{\hskip 3em}c@{\hskip 3em}c}
        C_{24} & C_3{\rtimes}C_8 & C_{12}{\times}C_2 \\
        \Dic_{12}{\times}C_2 & S_3{\times}C_4 & C_3{\times}C_2{\times}C_2{\times}C_2 \\
        S_3{\times}C_2{\times}C_2 & C_3{\times}D_8 & D_{24} \\
        C_3{\rtimes_{V_4}}D_8 & C_3{\times}Q_8 & \Dic_{24} \\
        S_4 & C_2{\times}A_4 & \SL_2(3)
    \end{array}
    \end{displaymath}
\end{theorem}

In the rest of this section, we will prove this result.
Let \(G\) be a group of order 24, and let \(H\) be a Sylow 3-subgroup of \(G\), so \(H \cong C_3\), and let \(h\)
generate \(H\).
Let \(T\) by a Sylow 2-subgroup of \(G\), so \(T\) has order 8.
By Lagrange's Theorem, \(H \cap T = \bm{1}\) and then applying Lemma~\ref{lem:setprodorder}, \(|HT| = 24\).
Now let \(n_3\) denote the number of Sylow 3-subgroups, and by Sylow's~Theorems:
\[n_3 \equiv 1\ \mod{3} \quad \text{and} \quad n_3 \mid 8\]
Hence \(n_3\) is either 1 or 4.

If \(n_3 = 1\), then \(H\) is normal in \(G\).
Thus \(G = H \rtimes T\).
We'll want a homomorphism \(\psi: T \to \Aut{H}\).
We know \(\Aut{H} \cong C_2\), and from our classification of groups of order 8, we have 5 possibilities.
An action of \(T\) on \(H\) will have image isomorphic to \(C_2\), and a kernel isomorphic to a group of order 4.
We can classify the possible actions by considering the kernel.

\begin{lemma}\label{lem:24_first}
    Let \(G\) be a group of order 24, with a normal Sylow 3-subgroup, and a Sylow 2-subgroup isomorphic to \(C_8\).
    Then \(G\) is isomorphic to one of:
    \[
        C_{24} \quad \text{or} \quad%
        C_3 \rtimes C_{8}
    \]
\end{lemma}

\begin{proof}
    We have \(G = H \rtimes T\), for the Sylow 3-subgroup \(H\), and a Sylow 2-subgroup \(T = \gen{t}\), with map
    \(\psi:T \to \Aut{H}\).
    If \(\psi\) is trivial, then:
    \[G = T \times H \cong C_{24}\]
    For a non-trivial \(\psi\), the kernel is a subgroup of order 4.
    There is a unique subgroup of order 4 in \(T\), generated by \(t^2\).
    Hence \(\gen{t^2}\) is the kernel of a non-trivial \(\psi\), so \(\psi\) must send \(t\) to the inversion map.
    Hence a non-trivial action of \(T\) on \(H\) is unique.
    We obtain:
    \[G = \gen{h,\,t \mid h^3 = t^8 = 1,\ h^{-1}th = t^{-1}} \cong C_3 \rtimes C_8\]
\end{proof}

\begin{lemma}
    Let \(G\) be a group of order 24, with a normal Sylow 3-subgroup, and a Sylow 2-subgroup isomorphic to \(C_{4}
    \times C_2\).
    Then \(G\) is isomorphic to one of:
    \[
        C_3 \times C_4 \times C_2, \quad%
        \Dic_{12} \times C_2 \quad \text{or} \quad%
        S_3 \times C_4%
    \]
\end{lemma}

\begin{proof}
    We have \(G = H \rtimes T\), for the Sylow 3-subgroup \(H\), and a Sylow 2-subgroup \(T \cong C_4 \times C_2\), with
    map \(\psi:T \to \Aut{H}\).
    In this case, \(T\) has subgroups isomorphic to both \(C_4\) and \(C_2 \times C_2\), so we have more
    possibilities for \(\psi\).
    Firstly, if \(\psi\) is trivial, then we obtain the direct product:
    \[G \cong C_3 \times C_4 \times C_2\]
    Let \(T\) be generated by \(x\) and \(y\), where \(x^4 = y^2 = 1\), and consider non-trivial \(\psi\).
    Say the kernel of \(\psi\) is isomorphic to \(C_2 \times C_2\).
    So it must be generated by the elements of order 2 in \(T\), \(x^2\) and \(y\).
    Then \(\psi\) must map \(x\) to the non-identity element in \(\Aut{H}\): inversion.
    Hence \(\gen{x}\) acts by inversion on \(H\), giving:
    \begin{equation*}
    \begin{aligned}
        G &= (H \rtimes \gen{x}) \times \gen{y} \\
        &\cong (C_3 \rtimes C_4) \times C_2 \\
        &\cong \Dic_{12} \times C_2
    \end{aligned}
    \end{equation*}
    If instead the kernel is isomorphic to \(C_4\), then it must be generated by an element of order 4 from \(T\).
    However, all elements of order 4 are automorphic, and so by Lemma~\ref{lem:semiisom}, we can pick \(x\) to
    generate the kernel, without loss of generality.
    So then \(\psi\) must map \(y\) to inversion.
    Hence \(\gen{x}\) acts trivially on \(H\), and \(\gen{y}\) acts by inversion.
    Thus:
    \begin{equation*}
    \begin{aligned}
        G &= (H \rtimes \gen{y}) \times \gen{x} \\
        &\cong (C_3 \rtimes C_2) \times C_4 \\
        &\cong S_3 \times C_4
    \end{aligned}
    \end{equation*}
\end{proof}

\begin{lemma}
    Let \(G\) be a group of order 24, with a normal Sylow 3-subgroup, and a Sylow 2-subgroup isomorphic to \(C_2 \times
    C_2 \times C_2\).
    Then \(G\) is isomorphic to one of:
    \[
        C_3 \times C_2 \times C_2 \times C_2 \quad \text{or} \quad%
        S_3 \times C_2 \times C_2
    \]
\end{lemma}

\begin{proof}
    We have \(G = H \rtimes T\), for the Sylow 3-subgroup \(H\), and the specified Sylow 2-subgroup \(T\), with map
    \(\psi:T \to \Aut{H}\).

    Let \(\gen{a,\,b,\,c} = T\).
    All elements in \(T\) have order 1 or 2, so cannot have subgroups isomorphic to \(C_4\).
    However, \(T\) does have subgroups isomorphic to \(C_2 \times C_2\), which can be generated by 2 of the 3
    generators of \(T\).
    This gives us 3 subgroups, but permuting the generators \(a,\,b\) and \(c\) is an automorphism of \(T\), so
    Lemma~\ref{lem:semiisom} tells us the resulting semidirect products are isomorphic.
    So choose \(\psi\) such that \(b\) and \(c\) are in the kernel.
    Then \(a\psi\) is either the identity map or the inversion map.
    If \(\psi\) is trivial, then we obtain the direct product:
    \[G \cong C_3 \times C_2 \times C_2 \times C_2\]
    If \(a\psi\) is inversion, then:
    \[G = (H \rtimes \gen{a}) \times \gen{b} \times \gen{c} \cong S_3 \times C_2 \times C_2\]
\end{proof}


\begin{lemma}
    Let \(G\) be a group of order 24, with a normal Sylow 3-subgroup, and a Sylow 2-subgroup isomorphic to \(D_8\).
    Then \(G\) is isomorphic to one of:
    \[
        C_3 \times D_8, \quad%
        D_{24} \quad \text{or} \quad%
        C_3 \rtimes_{V_4} D_8
    \]
\end{lemma}

\begin{proof}
    We have \(G = H \rtimes T\), for the Sylow 3-subgroup \(H\), and the specified Sylow 2-subgroup \(T\), with map
    \(\psi:T \to \Aut{H}\).

    Let \(r\) and \(s\) generate \(T\) with \(r^4 = s^2 = 1\).
    A trivial homomorphism will yield the direct product:
    \[G \cong C_3 \times D_8\]
    So for a non trivial homomorphism, firstly assume \(\ker\psi \cong C_4\).
    There is a unique subgroup in \(T\) isomorphic to \(C_4\), so its generated by an element of order 4.
    However the choice of generator is the same up to an isomorphism of \(T\), so Lemma~\ref{lem:semiisom} lets us
    pick \(r\) to be the generator, without loss of generality.
    Hence \(s\) cannot be in the kernel, and so \(s\psi\) is the inversion map.
    We obtain the presentation:
    \[G = \gen{h,\,r,\,s \mid h^3 = r^4 = s^2 = 1,\ hr = rh,\ s^{-1}rs = r^{-1},\ s^{-1}hs = h^{-1}}\]
    Let \(a = hr\), and consider:
    \[s^{-1}as = s^{-1}hrs = s^{-1}hrs^2 s^{-1} = (s^{-1}hs)(srs^{-1}) = h^{-1}r^{-1} = r^{-1}h^{-1} = a^{-1}\]
    So we have:
    \[G = \gen{a,\,s \mid a^{12} = s^2 = 1,\ s^{-1}as = a^{-1}}\]
    Which we recognise as \(D_{24}\), the dihedral group of order 24.

    If instead we consider \(\psi\) with kernel isomorphic to \(C_2 \times C_2\), then the kernel is generated by
    two elements of order 2.
    However, \(T\) only has two elements of order 2, \(r^2\) and \(s\), so they must generate the kernel.
    So then \(\psi\) must map \(r\) to inversion.
    Hence this action is fully specified.
    So:
    \[G \cong C_3 \rtimes_{V_4} D_8\]
    We will use the above notation to mean the unique action with kernel isomorphic to \(V_4\).
\end{proof}

\begin{lemma}
    Let \(G\) be a group of order 24, with a normal Sylow 3-subgroup, and a Sylow 2-subgroup isomorphic to \(Q_8\).
    Then \(G\) is isomorphic to one of:
    \[
        C_3 \times Q_8, \quad \text{or} \quad%
        \Dic_{24}
    \]
\end{lemma}

\begin{proof}
    We have \(G = H \rtimes T\), for the Sylow 3-subgroup \(H\), and a Sylow 2-subgroup \(T \cong Q_8\), with map
    \(\psi:T \to \Aut{H}\).

    Let \(T\) be generated by \(i\) and \(j\), with the product denoted by \(k\).
    That is:
    \[T = \gen{i,\,j \mid i^4 = j^4 = 1,\ i^2 = j^2,\ j^{-1}ij = i^{-1}}\]
    There is a single element of order 2 in \(T\), hence \(T\) has no subgroup isomorphic to \(C_2 \times C_2\).
    The elements \(i,\,j\) and \(k\) each generate a cyclic subgroup in \(T\).
    So \(\psi\) will send one of them to the kernel.
    We know that permuting these is an automorphism of \(T\), so Lemma~\ref{lem:semiisom} tells us the choice
    results in isomorphic semidirect products.

    So take \(i \in \ker{\psi}\).
    Indeed \(\gen{i} = \ker\psi\).
    Then for a non-trivial homomorphism, we must have \(j \notin \ker{\psi}\).
    Otherwise:
    \[i\psi\ j\psi = (ij)\psi = k\psi \in \ker{\psi}\]
    making \(\psi\) trivial.

    Thus either \(\psi\) is trivial and we obtain:
    \[G \cong C_3 \times Q_8\]
    or \(\psi\) maps \(j\) to the inversion map and we obtain the presentation:
    \[G = \gen{h,\,i,\,j \mid h^3 = i^4 = j^4 = 1,\ hi = ih,\ i^2 = j^2,\ j^{-1}hj = h^{-1},\ j^{-1}ij = i^{-1}}\]
    Now let \(a = hi\).
    So:
    \[a^6 = h^6 i^6 = i^2 = j^2\]
    And:
    \[j^{-1}aj = j^{-1}hij = j^{-1}hji^{-1} = h^{-1}i^{-1} = i^{-1}h^{-1} = a^{-1}\]
    Hence:
    \[G = \gen{a,\,j \mid a^{12} = 1,\ a^6 = j^2,\ j^{-1}aj = a^{-1}}\]
    We recognise this as the dicyclic group of order 24, \(\Dic_{24}\).
\end{proof}

Now we will consider groups of order 24 which don't have a normal Sylow 3-subgroup.

\begin{lemma}\label{lem:order_24_2_subgroup}
    Let \(G\) be a group of order 24, with a non-normal Sylow 3-subgroup.
    Then \(G\) either has a normal Sylow 2-subgroup, or is isomorphic to \(S_{4}\).

    If \(G\) does have a normal Sylow 2-subgroup, \(T\), then it can be expressed as a semidirect product with a Sylow
    3-subgroup, \(H\), \(G = T \rtimes H\).
\end{lemma}

\begin{proof}
    If \(n_3 = 4\) then \(H\) is not normal.
    We will proceed to show the result in a similar way to Borcherds\footcite{order24}.

    The normaliser of \(H\), \(\Norm_G(H)\) has index 4.
    Now let \(G\) act on the set of the cosets of \(\Norm_G(H)\) by conjugation.
    Hence we obtain a homomorphism \(\rho:G \to S_4\).
    The kernel is a subgroup of \(\Norm_G(H)\) so must have order dividing 6 by Lagrange's~Theorem.

    The kernel cannot be of order 3, because \(G\) has no normal subgroup of order 3 (because kernels of homomorphism are
    normal in the domain group).
    Likewise the kernel cannot be of order 6 because every group of order 6 has a unique Sylow 3-subgroup, which is
    characteristic.
    So by Lemma~\ref{lem:char}, it would be normal in \(G\).
    Hence the kernel must have order 1 or 2.

    If the kernel is of order 1, then \(\rho\) is an isomorphism, so \(G \cong S_4\).

    If the kernel is of order 2, then we know that \(G/\ker{\rho} \cong \im{\rho}\), so then \(\im{\rho}\) must have order
    12.
    It also cannot have a normal Sylow 3-subgroup, so looking at our classification of groups of order 12, this must be
    isomorphic to \(A_4\).
    We know that \(A_4\) has a normal subgroup of order 4, and so by the Correspondence~Theorem, \(G\) must contain a normal
    subgroup of order 8, say \(T\).

    Considering now the remaining groups of order 24 which will have a normal Sylow 2-subgroup, by Lagrange's~Theorem and
    Lemma~\ref{lem:setprodorder}, we can conclude that \(G = T \rtimes H\).
\end{proof}

Again, we have 5 cases which we will prove in 5 sub-lemmas, but this time we'll exclude the trivial homomorphism in
the proofs, because that will just give us the direct
product which we have already seen.

\begin{lemma}
    Let \(G\) be a group of order 24, with a normal Sylow 2-subgroup, isomorphic to \(C_8\).
    Then \(G\) is isomorphic to \(C_8 \times C_3\).
\end{lemma}

\begin{proof}
    By Lemma~\ref{lem:order_24_2_subgroup}, \(G = T \rtimes H\).
    An automorphism of \(T\), \(\varphi\), maps generators to generators, so say \(\langle x \rangle = T\).
    Then \(x\varphi\) could be \(x\), \(x^3\), \(x^5\) or \(x^7\).
    Hence, and Lagrange's~Theorem tells us that there are no non-trivial homomorphisms \(\psi:H \to \Aut{T}\).
    As a bonus: notice that each of these, apart from the identity, has order 2, so \(\Aut{C_8} \cong V_4\).
\end{proof}

\begin{lemma}
    Let \(G\) be a group of order 24, with a normal Sylow 2-subgroup, isomorphic to \(C_4 \times C_{2}\).
    Then \(G\) is isomorphic to \(C_4 \times C_2 \times C_3\).
\end{lemma}

\begin{proof}
    By Lemma~\ref{lem:order_24_2_subgroup}, \(G = T \rtimes H\).
    An automorphism of \(T\), say \(\phi\), preserves element order.
    So if:
    \[T = \langle\,x,\,y \mid x^4 = y^2 = 1,\ xy = yx\,\rangle\]
    then \(x\phi\) must be of order 4, and \(y\phi\) must be of order 2.
    Moreover, \(y\phi\) cannot be in \(\gen{x\phi}\) because \(\phi\) is injective.

    So we are reduced to 2 possible choices for \(y\phi\), and 4 possible choices for \(x\phi\).
    Because an automorphism is determined by its effect on generators, this gives us 8 possible automorphisms.
    Hence \(|\Aut{T}| = 8\), and Lagrange's~Theorem tells us that there are no non-trivial homomorphisms \(\psi:H
    \to \Aut{T}\).
\end{proof}

\begin{lemma}
    Let \(G\) be a group of order 24, with a normal Sylow 2-subgroup, isomorphic to \(C_{2} \times C_2 \times C_{2}\).
    Then \(G\) is isomorphic to one of:
    \[
        C_2 \times C_2 \times C_2 \times C_3 \quad \text{or} \quad%
        C_2 \times A_4
    \]
\end{lemma}

\begin{proof}
    By Lemma~\ref{lem:order_24_2_subgroup}, \(G = T \rtimes H\).
    To determine \(\Aut{T}\) it is helpful to think of \(C_2\) as the finite field with two elements.
    Then \(T\) is isomorphic a 3 dimensional vector space over two elements.
    So an automorphism of that vector space is just any linear map, with non-zero determinant.
    Thus, \(\Aut{T} \cong \GL_3(2)\).

    We can determine that \(|\GL_3(2)| = 168 = 2^3 \cdot 3 \cdot 7\), so \(\Aut T\) has a Sylow 3-subgroup of order
    3, isomorphic to \(C_3\).
    Sylow's~Theorems tells us that all subgroups of order 3 are conjugate, so Lemma~\ref{lem:semiisom} tells us
    there is only one unique action (up to isomorphism) of \(H\) on \(T\).
    As before, pick a homomorphism, \(\psi\), which will let us easily classify the resulting semidirect product.

    Write \(T = A \times B\) where \(A \cong C_2\) and \(B \cong C_2 \times C_2\).
    Then let \(\psi\) map \(H\) to the subgroup generated by the automorphism which fixes \(A\) and permutes the
    non-identity elements of \(B\) in a 3-cycle.
    This automorphism has order 3 by construction, so we can write:
    \[G \cong C_2 \times (V_4 \rtimes C_3)\]
    We know already that \(V_4 \rtimes C_3 \cong A_4\), so \(G \cong C_2 \times A_4\).
\end{proof}

\begin{lemma}
    Let \(G\) be a group of order 24, with a normal Sylow 2-subgroup, isomorphic to \(D_8\).
    Then \(G\) is isomorphic to \(D_8 \times C_3\).
\end{lemma}

\begin{proof}
    By Lemma~\ref{lem:order_24_2_subgroup}, \(G = T \rtimes H\).
    Let \(\gen{s,\,r \mid s^2 = r^4 = 1,\ s^{-1}rs = r^{-1}} = T\).
    An automorphism, \(\psi\), of \(T\) preserves element order, so for \(r\psi\) we have two choices, \(r\) or
    \(r^{-1}\).
    We can send \(s\psi\) to any element of order 2 which is not in \(\langle r\psi \rangle\).
    This leaves only reflections, of which there are 4: \(s,\,rs,\,r^2s\) and \(r^3s\).
    Hence there are 8 possible automorphisms of \(D_8\), so \(|\Aut{D_8}| = 8\).
    Lagrange's~Theorem tells us that there are no non-trivial homomorphisms \(\psi:H \to \Aut{T}\).
\end{proof}

\begin{lemma}\label{lem:24_last}
    Let \(G\) be a group of order 24, with a normal Sylow 2-subgroup, isomorphic to \(Q_8\).
    Then \(G\) is isomorphic to one of:
    \[
        Q_8 \times C_3 \quad \text{or} \quad%
        \SL_2(3)
    \]
\end{lemma}

\begin{proof}
    By Lemma~\ref{lem:order_24_2_subgroup}, \(G = T \rtimes H\).
    Firstly, for an automorphism, \(\phi\), because:
    \[i\phi j\phi = (ij)\phi\]
    the image of \(k\) under \(\phi\) is determined by the images of \(i\) and \(j\).
    This reduces the possibilities for an automorphism.
    Additionally, \(\pm 1\) are fixed by an automorphism, because they are the only elements of their order.
    So an automorphism could send \(i\) to any of the 6 elements of order 4.
    The image of \(j\) cannot be in the subgroup generated by the image of \(i\), otherwise we wouldn't have an
    automorphism.
    Thus there are 4 choices for the image of \(j\), giving us at most 24 possible automorphisms.

    An element \(g \in T\) can be written as \(g = i^l j^u\) for some \(0 \leqslant l,\, u \leqslant 3\).
    So then:
    \begin{align*}
        g\phi &= (i^l j^u) \phi \\
        &= (i^l \phi)(j^u \phi) \\
        &= {(i\phi)}^l {(j\phi)}^u
    \end{align*}
    Hence \(g\phi\) is determined by \(i\phi\) and \(j\phi\), and each of the 24 possible combinations of images extends
    to a valid automorphism of \(T\).
    Therefore, \(|\Aut{T}| = 24\).

    So \(\Aut{T}\) will have a Sylow subgroup of order 3, and Sylow tells us all subgroups of order 3 are conjugate.
    If our homomorphism for the semidirect product is \(\psi:H \to \Aut{T}\), then Lemma~\ref{lem:semiisom} tells us
    that if \(\psi\) maps \(H\) to a given subgroup in \(\Aut{T}\), then mapping \(H\) to a conjugate subgroup produces
    an isomorphic semidirect product.
    Thus the choice of which subgroup of \(\Aut{T}\) (if there is any choice) to map into is equivalent up to
    isomorphism.

    If \(t\) generates a subgroup of order 3 in \(\Aut{T}\), then \(\psi\) can map \(t\) to either \(h\) or \(h^{-1}\).
    But again, Lemma~\ref{lem:semiisom} tells us that by applying \(\beta:x \mapsto x^{-1}\) if necessary, the two
    choices result in isomorphic semidirect products.
    Hence \(T \rtimes H\) is uniquely specified.

    We will show that \(\SL_2(3)\) is in that isomorphism class.
    For the sake of brevity, not every fact will be explicitly shown, and we will indicate where the reader may verify
    assertions made.
    Firstly, let \(e_1\) and \(e_2\) be the basis vectors for the vector space.
    We can find an element of order 3 in \(\SL_2(3)\):
    \[\bar{h} = \twomat{1}{1}{0}{1}\]
    And so \(\bar{h}\) generates a cyclic subgroup of order 3 inside \(\SL_2(3)\), say \(\bar{H}\).
    It can be shown that the centre of \(\SL_2(3)\) (denote by \(\bar{Z}\)) is:
    \[ \bar{Z} = \left\{\,\twomat{\lambda}{0}{0}{\lambda} \mid \lambda^2 = 1\,\right\} = \{\,\pm I\,\} \]
    Now let's find 3 matrices which correspond to \(i\), \(j\) and \(k\) from \(Q_8\).
    We need matrices, \(A\), \(B\) and \(C\) such that \(M^2 = -1\).
    So the minimal polynomial of \(M\) is \(x^2 + 1\).
    Without too much trouble, we can find:
    \[ A = \twomat{0}{1}{-1}{0} \]
    and check it's minimal polynomial.
    Conjugating by \(\bar{h}\) will give us another:
    \[ B = \twomat{1}{-1}{0}{1}\twomat{0}{1}{-1}{0}\twomat{1}{1}{0}{1} = \twomat{1}{-1}{-1}{-1} \]
    So then their product will give us \(C\):
    \[ C = AB = \twomat{0}{1}{-1}{0}\twomat{1}{-1}{-1}{-1} = \twomat{-1}{-1}{-1}{1} \]
    These calculations can be verified by the reader.
    With \(A\), \(B\) and \(C\) in hand, let's quotient \(\SL_2(3)\) by its centre and find what group we get.
    It will turn out that \(\SL_2(3)/\bar{Z}\) is isomorphic to \(A_4\), and we will use this to find a normal copy of
    \(Q_8\) inside \(\SL_2(3)\).

    In \(\SL_2(3)/\bar{Z}\), we have 4 1-dimensional subspaces given by:
    \begin{align*}
        U_1 &= \Sp(e_1) \\
        U_2 &= \Sp(e_2) \\
        U_3 &= \Sp(e_1 + e_2) \\
        U_4 &= \Sp(e_1 - e_2)
    \end{align*}
    Let's investigate how the quotient acts on the set of these subspaces.
    The element \(A\) sets \(e_1\) to \(e_2\) and \(e_2\) to \(-e_1\), so swaps \(U_1\) and \(U_2\).
    And so \(e_1 + e_2\) becomes \(e_1 - e_2\), and \(e_1 - e_2\) becomes \(-e_1 - e_2 = -(e_1 + e_2)\).
    Thus \(A\) swaps \(U_{3}\) and \(U_4\) and so it corresponds to the permutation \((1\ 2)(3\ 4)\).

    By a similar argument, we can determine that \(B\) corresponds to \((1\ 4)(2\ 3)\), and \(C\) corresponds to
    \((1\ 3)(2\ 4)\).
    Again, these calculations can be checked.
    So we can conclude that:
    \[\SL_2(3)/\bar{Z} \cong A_4\]
    We know that \(A_4\) has a normal subgroup isomorphic to \(C_2 {\times} C_{2}\), and so by applying the
    Correspondence Theorem, \(\SL_2(3)\) has a normal subgroup isomorphic to \(Q_8\), which contains the centre, and
    elements \(A\), \(B\) and \(C\).

    We can check that \(\bar{H} \cap \bar{T} = \bm{1}\) and so \(\SL_2(3) = T \rtimes H\).
    Hence, \(Q_8 {\rtimes} C_3 \cong \SL_2(3)\).
\end{proof}

Together, Lemmas~\ref{lem:24_first}~to~\ref{lem:24_last} prove Theorem~\ref{thm:24}.

