% ====   Order 30   ====
\section{Groups of Order 30}
\begin{theorem}\label{thm:30}
    Any group of order 30 is isomorphic to one of:
    \[
        C_{30}, \quad%
        D_{15}, \quad%
        C_5 \times D_6 \quad \text{or} \quad%
        C_3 \times D_{10}
    \]
\end{theorem}

We will devote the rest of this section to the proof.
This classification is based on the one given in the cited Stack Exchange post\footcite{order30}.
Let \(G\) be a group of order \(30 = 2 \cdot 3 \cdot 5\).
We will first show that \(G\) can be written as one of 4 semidirect products, then we will identify what those products
are isomorphic to.

\begin{lemma}\label{lem:order_30_actions}
    Any group of order 30 can be expressed as a semidirect product of the subgroup of order 15 with a Sylow 2-subgroup.
    Moreover, there are precisely 4 possible actions for such semidirect products.
\end{lemma}

\begin{proof}
    Sylow's Theorems tell us \(G\) has a Sylow 3-subgroup, \(T\), and a Sylow 5-subgroup, \(F\).
    Let \(H = TF\) and by Lagrange's Theorem, \(T \cap F = \bm{1}\), hence \(|H| = 15\) by Lemma~\ref{lem:setprodorder}.
    We know from our classification of groups of order \(pq\) that \(H \cong C_{15}\).
    Because \(|H| = 15 = \frac{30}{2}\), the index of \(H\) in \(G\) is 2, and we know a subgroup of index 2 is normal,
    so \(H \nrmsgp G\).

    A Sylow 2-subgroup \(K \leqslant G\) has order 2, so \(K \cong C_2\).
    Let \(\langle k \rangle = K\) and \(\langle h \rangle = H\).
    By the same argument as above, \(H \cap K = \bm{1}\) and \(|HK| = 30\).
    Hence \(G = HK\).
    Moreover, \(G = H \rtimes K\).

    By Lemma~\ref{lem:aut}:
    \[\Aut{C_{15}} = \units{15} \cong {(\Zn{3} \times \Zn{5})}^\times \cong \units{3} \times \units{5} \cong C_2 \times C_4\]
    A homomorphism, \(\psi:C_2 \to C_2 \times C_4\) preserves element order and we know \(\psi\) is determined by its
    effect on a generator.
    So then \(k\psi\) has four possibilities: either the identity, or one of the three elements
    of order 2.

    Additionally, \(\psi\) preserves the Sylow subgroups of \(H\).
    So write \(H = \langle h^3 \rangle \times \langle h^5 \rangle\), the direct product of its Sylow subgroups.
    So the action of \(K\) on \(H\) is either trivial or by inversion on each of the Sylow subgroups of \(H\), giving us 4
    possibilities.
\end{proof}

\begin{lemma}
    The semidirect products specified by Lemma~\ref{lem:order_30_actions} are isomorphic to:
    \[
        C_{30}, \quad%
        D_{15}, \quad%
        C_5 \times D_6 \quad \text{or} \quad%
        C_3 \times D_{10}
    \]
\end{lemma}

\begin{proof}
    \mbox{}
    \begin{enumerate}[\bfseries{Case} 1:]
        \item Trivial action on both Sylow subgroups.

            In this case, because the action is trivial on all of \(H\), we recover the direct product, \(G = H \times K
            \cong C_{30}\).
        \item Inversion on both Sylow subgroups.

            Here, \(K\) acts on all of \(H\), so we obtain:
            \[G = \langle\, h,\,k \mid h^{15} = k^2 = 1,\ k^{-1}hk = h^{-1}\,\rangle\]
            which we recognise as \(D_{30}\).
        \item Inversion on \(\langle h^5 \rangle\).

            We know already, from our classification of groups of order \(2p\), that \(C_3 \rtimes C_2 \cong D_6\).
            So then because the action on \(\langle h^3 \rangle\) is trivial:
            \[G = \langle h^3 \rangle \times (\langle h^5 \rangle \rtimes K) \cong C_5 \times D_6\]
        \item Inversion on \(\langle h^3 \rangle\).

            Similar to above, we obtain:
            \[G = \langle h^5 \rangle \times (\langle h^3 \rangle \rtimes K) \cong C_3 \times D_{10}\]
    \end{enumerate}
\end{proof}

These two lemmas prove Theorem~\ref{thm:30}.

