% ====   Groups of order p^2   ====
\section{Groups of Order \(p^2\)}
Slightly harder, are groups of order \(p^2\).
Sylow's Theorems, which we will make extensive use of later, are not so useful here.
Nonetheless, we have:

\begin{theorem}
    For a prime \(p\), any group of order \(p^2\) is isomorphic to one of:
    \[C_{p^2} \quad \text{or} \quad C_p \times C_p\]
\end{theorem}

\begin{proof}
    Let \(G\) be a group of order \(p^2\).
    By Lagrange's Theorem, the elements of \(G\) have order 1, \(p\) or \(p^2\).
    We will consider these possible cases in turn:

    \begin{case}
    \item \(G\) has an element of order \(p^2\).

        If \(x \in G\) has order \(p^2\), then \(x\) generates \(G\) so \(G \cong C_{p^2}\).

    \item \(G\) has no element of order \(p^2\).

        If \(G\) does not have an element of order \(p^2\) then all elements, except the identity, have order \(p\).
        We know that \(G\) must have a subgroup of order \(p\), \(P\), and because \(p\) is prime, \(P \cong C_p\). Pick
        a generator for \(P\), say \(x\) and an element \(y \in G\) such that \(y \notin P\).

        The intersection is a subgroup of both \(P\) and \(\bar{P}\).
        It must be a proper subgroup because we chose \(y \notin P\).
        Lagrange's~Theorem tells us it must be trivial and Lemma~\ref{lem:index_p_normal} tells us that both \(P\) and
        \(\bar{P}\) are normal, and by Lemma~\ref{lem:setprodorder}, \(|P\bar{P}| = p^2 = |G|\), so:
        \[G = P \times \bar{P} \cong C_p \times C_p\]
    \end{case}
\end{proof}


