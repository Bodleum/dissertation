% ====   Groups of order p^2   ====
\section{Groups of Order \(p^2\)}
Let \(G\) be a group of order \(p^2\).
By Lagrange's Theorem, the elements of \(G\) have order 1, \(p\) or \(p^2\).
If \(x \in G\) has order \(p^2\), then \(x\) generates \(G\) so \(G \cong C_{p^2}\).

If \(G\) does not have an element of order \(p^2\) then all elements, except the identity, have order \(p\).
We know that \(G\) must have a subgroup of order \(p\), \(P\), and because \(p\) is prime, \(P \cong C_p\).
Pick a generator for \(P\), say \(x\) and an element \(y \in G\) such that \(y \notin P\).
Then \(y \neq x^i\) for any \(i\).

If \(y^j = x^i\) for some \(i\) and \(j\), then:
\[{(y^j)}^{1-j} = {(x^i)}^{1-j} = y^{j-j+1} = y = x^{i(1-j)} = x^k \quad \text{for some} \ k\text{,} \ \text{a
contradiction.}\]
So no power of \(y\) is equal to any power of \(x\).
Because \(y\) has order \(p\), it generates a subgroup of order \(p\), \(\bar{P}\), with \(P \cap \bar{P} = \bm{1}\).
Lemma~\ref{lem:index_p_normal} tells us that both \(P\) and \(\bar{P}\) are normal, and by Lemma~\ref{lem:setprodorder},
\(|P\bar{P}| = p^2 = |G|\), so:
\[G = P \times \bar{P} \cong C_p \times C_p\]

If \(G\) has no elements of order \(p\) or \(p^2\), then it only has elements of order 1, which is the trivial group.

Hence any group of order \(p^2\) is isomorphic to one of:
\[C_{p^2} \quad \text{or} \quad C_p \times C_p\]

