% ====   Order pq   ====
\section{Groups of Order \(pq\)}
\begin{theorem}\label{thm:pq}
    For distinct primes \(p\) and \(q\), any group of order \(pq\) is isomorphic to one of:
    \[C_{pq}\]
    \[C_{p} \rtimes C_{q} = \gen{x, y \mid x^p = y^q = 1,\ y^{-1}xy = x^a} \tag{additionally, if \(q \mid p-1\)}\]
    where \(a\) is a generator for the subgroup of order \(q\) in \(\units{p}\).
\end{theorem}

We will devote the rest of the section to the proof of this Theorem.
Let \(G\) be a group of order \(pq\) where \(p\), \(q\) are prime numbers with \(p > q\), and let \(n_p\) and \(n_q\)
denote the number of Sylow p-subgroups and Sylow q-subgroups of \(G\) respectively.
Then by Sylow's~Theorems:
\[n_p \equiv 1 \ \pmod{p} \quad \text{and} \quad n_p \mid q\]
So \(G\) has a unique Sylow \(p\)-subgroup, say \(P \nrmsgp G\), and a Sylow \(q\)-subgroup, \(Q \leqslant G\).
Because \(p\) and \(q\) are prime numbers, \(P \cong C_p\) and \(Q \cong C_q\).
Pick generators for each, say \(x\) and \(y\).
So:
\[
    P = \gen{x} \cong C_{p} \quad \text{and} \quad Q = \gen{y} \cong C_q
\]

We have 2 possibilities for \(n_q\): \(p-1\) is a multiple of \(q\) or 1.

\begin{lemma}
    For distinct primes \(p\) and \(q\), with \(q \nmid p-1\), any group of order \(pq\) is isomorphic to \(C_{pq}\).
\end{lemma}

\begin{proof}
    If \(p-1\) is not a multiple of \(q\), then \(n_q = 1\) and \(Q \nrmsgp G\), hence:
    \[G = P \times Q \cong C_{pq}\]
\end{proof}

\begin{lemma}
    For distinct primes \(p\) and \(q\), with \(q \mid p-1\), any group of order \(pq\) is isomorphic to one of:
    \[C_{pq} \quad \text{or}\]
    \[C_{p} \rtimes C_{q} = \gen{x, y \mid x^p = y^q = 1,\ y^{-1}xy = x^a}\]
    where \(a\) is a generator for the subgroup of order \(q\) in \(\units{p}\).
\end{lemma}

\begin{proof}
    If \(p-1\) is a multiple of \(q\), then \(n_q\) could be \(p\) as well as 1.
    So \(Q\) is not necessarily normal in \(G\).
    Therefore we have another group as well as the direct product.
    Concentrating on this group, Lagrange's Theorem tell us \(P \cap Q = \bm{1}\) and by Lemma~\ref{lem:setprodorder}, \(|PQ| = pq\).
    Hence we have \(G = P \rtimes Q\), some non-trivial semidirect product.

    By Lemma~\ref{lem:aut}, \(\Aut{C_p} \cong \units{p} \cong C_{p-1}\).
    So if \(\nu \in \units{p}\), then \(x \mapsto x^\nu\) is an automorphism.
    We know also that \(C_{p-1}\) has a unique subgroup of order \(q\),
    hence \(G\) has the presentation:
    \[G = \langle \, x, y, \mid x^p = y^q = 1,\ y^{-1}xy = x^a \, \rangle\]
    where \(a\) is a generator for the subgroup of order \(q\) in \(\units{p}\).

    Notice that picking different generators are equivalent up to isomorphism because the composition of two
    isomorphisms is an isomorphism.
\end{proof}

Thus Theorem~\ref{thm:pq} is proved.
