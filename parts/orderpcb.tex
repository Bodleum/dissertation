% ====   Order p^3   ====
\section{Groups of Order \(p^3\)}
This classification is based on the one found on the Groupprops subwiki\footcite{pcubed}.
Let \(G\) be a group of order \(p^3\), where \(p\) is a prime number.
We will first gain a handle on \(G\) by describing its centre, and quotient by it.
If \(G\) is abelian, we know by the Fundamental Theorem of Finite Abelian Groups that it is isomorphic to one of:
\[ C_{p^3}, \quad C_{p^2} \times C_p \quad \text{or} \quad C_p \times C_p \times C_p\]
So from now on, we will focus on the non-abelian groups.

Denote the centre of \(G\) by \(Z\) and consider its order.
Lagrange's~Theorem tells us \(Z\) must have order dividing \(p^3\).
It cannot be \(p^3\) because \(G\) is non-abelian, and Lemma~\ref{lem:z_non_trivial} tells us that it cannot be 1.
If \(|Z| = p^2\), then \(|G/Z| = p\), so \(G/Z \cong C_p\).
However Lemma~\ref{lem:cyclic_over_centre} says that then \(G\) must be abelian, so then \(|Z|\) must be \(p\).
Then by our previous classification, \(G/Z\) is isomorphic to either \(C_{p^2}\) or \(C_p \times C_p\).
Lemma~\ref{lem:cyclic_over_centre} tells us that it must by the latter.

This gives us a handle to start investigating the structure of \(G\).
Another useful tool will be \emph{commutators}, which we will denote by \(\com{a}{b} = \comex{a}{b}\).
The \emph{derived subgroup} of \(G\), \(G' = \gen{\com{x}{y} \mid x,\,y \in G}\), is the smallest normal
subgroup such that \(G/G'\) is abelian.
We saw that \(G/Z\) is abelian, so \(G' \leqslant Z\), but because \(G'\) is non-trivial, we must have equality.
Now we will prove a useful lemma which holds in \(G\).

\begin{lemma}\label{lem:comm}
    Suppose \(G\) is a group such that \(G' \leqslant \z(G)\).
    Then for elements \(a,\,b,\,c \in G\):
    \[\com{a}{bc} = \com{a}{b}\com{a}{c}\]
\end{lemma}

\begin{proof}
    First we note that:
    \[abc = bca\com{a}{bc}\]
    Then moving one term at a time:
    \[abc = ba\com{a}{b}c = bac\com{a}{b}^c = bca\com{a}{c}\com{a}{b}^c\]
    Hence:
    \[bca\com{a}{bc} = bca\com{a}{c}\com{a}{b}^c\]
    Now by multiplying on the left by \(a^{-1}c^{-1}b^{-1}\) gives:
    \[\com{a}{bc} = \com{a}{c}\com{a}{b}^c\]
    Because \(G' \leqslant \z(G)\), conjugation by \(c\) has no effect.
    Additionally, the two commutators commute, giving:
    \[\com{a}{bc} = \com{a}{b}\com{a}{c}\]
    as required.
\end{proof}

So far, we know \(G/Z \cong C_p \times C_p\), and that \(G' = Z\), as well as a useful lemma.
Now pick two elements, \(a\) and \(b\) so that \(aZ\) and \(bZ\) generate \(G/Z\).
So then \(G = \gen{Z,\,a,\,b}\).

Let \(z = \com{a}{b}\).
If \(z = 1\) then that means \(a\) and \(b\) commute.
And by definition, \(a\) commutes with \(Z\), so \(a \in Z\), which contradicts our choice of \(a\) as a generator of
\(G/Z\).
Hence \(z \neq 1\), and in particular, \(a\) and \(b\) do not commute.
Now we know \(G' = Z\) which has order \(p\), so \(Z \cong C_p\).
Moreover, \(z \in Z\), and \(z \neq 1\) so we can conclude that \(\gen{z} = Z\).
We can see that although \(a\) and \(b\) are not in \(Z\), \(a^p\) and \(b^p\) are, because \(aZ\) and \(bZ\) have order
\(p\) in \(G/Z\).
Considering the orders of \(a\) and \(b\) we have 3 cases:

\begin{enumerate}[\bfseries{Case} 1:]
    \item Both \(a\) and \(b\) have order \(p\).

        The above descriptions give the presentation:
        \[G = \gen{z,\,a,\,b \mid z^p = a^p = b^p = 1,\ az = za,\ bz = zb,\ \com{a}{b} = z}\]
        We can write an arbitrary \(g \in G\) as \(a^i b^j z^k\) for integers \(i\), \(j\) and \(k\) taken mod \(p\).
        Hence this presentation has order at most \(p^3\).

        Now consider the set:
        \[\left\{\,\unimat{\alpha}{\beta}{\gamma} \mid \alpha,\,\beta,\,\gamma \in \F_p\,\right\}\]
        It can be shown that this is a group under the usual matrix multiplication, and is known as the
        \emph{unitriangular group}\footcite{unitriangular}, denoted \(\UT_3(p)\).
        Taking:
        \[z = \unimat{0}{1}{0}, \quad a = \unimat{1}{0}{0} \quad \text{and} \quad b = \unimat{0}{0}{1}\]
        we can see that \(\UT_3(p)\) satisfies this presentation for \(p > 2\).
        (Indeed, the above presentation is the standard presentation definition for \(\UT_3(p)\)).
        Thus there is a single isomorphism class for this case.

        The group behaves differently when \(p = 2\) because we know that a group whose elements all have order either 1
        or 2 is abelian.
        So the elements cannot have order only 1 or 2.
        In particular:
        \[\unimat{1}{0}{1} \quad \text{and} \quad \unimat{1}{1}{1}\]
        have order 4.
        We can check that all other non-identity elements have order 2.
        Thus \(\UT_2(2) \cong D_8\).

    \item One element of each order \(p\) and \(p^2\).

        The roles of \(a\) and \(b\) are interchangeable, so we can take \(a\) to have order \(p^2\) and \(b\) to have
        order \(p\) without loss of generality.
        So \(\gen{a} \cong C_{p^2}\), and has index \(p\), so by Lemma~\ref{lem:index_p_normal} is normal.
        We noted that \(a^p \neq 1\) is in \(Z\), and so \(a^p\) is some power of \(z\), say \(a^p = z^r\), mod \(p\).
        Consider the element \(c = b^r\).
        Note that \(c\) is not the identity, because \(0 < r < p\), hence \(\gen{c} \cong C_p\).
        Now we will show that \(\gen{a} \cap \gen{c} = \bm{1}\).
        If \(\gen{a} \cap \gen{c} \neq \bm{1}\), then for some \(i\), \(j\):
        \[a^i = c^j = {(b^r)}^j = b^{rj} \mod{p}\]
        which is a contradiction, because \(G = \gen{z,\,a,\,b}\).
        The order of \(c\) is \(p\), because \(\gcd(r,\,p) = 1\).
        So we can conclude that \(G = \gen{a} \rtimes \gen{c}\).
        How does \(c\) conjugate \(a\)?
        Consider:
        \[\comex{a}{c} = \com{a}{c} = \com{a}{b^r}\]
        By Lemma~\ref{lem:comm}:
        \[\com{a}{b^r} = \com{a}{b}^r = z^r = a^p\]
        So:
        \[c^{-1}ac = a^{p + 1}\]
        Thus:
        \[G \cong C_{p^2} \rtimes C_p\]
        With presentation:
        \[\gen{a,\,c \mid a^{p^2} = c^p = 1,\ c^{-1}ac = a^{p + 1}}\]

        When \(p = 2\), this reduces to the dihedral group, \(D_8\).
        We can see that for \(p > 2\), this is not isomorphic to the previous case, because \(\UT_3(p)\) has no element
        of order \(p^2\).

    \item Both \(a\) and \(b\) have order \(p^2\).

\end{enumerate}

