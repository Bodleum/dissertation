% ==== Order p^2q   ====
\section{Some Groups of Order \(p^2q\)}
Let \(p\) and \(q\) be distinct prime numbers, and \(G\) be a group of order \(p^2q\).
To classify \(G\) in full generality is beyond this report, so we will focus on the cases when \(p = 2\) and when \(q =
2\).

\subsection{\(4q\)}
Let \(G\) be a group of order \(4p\), and require \(p > 3\).
And let \(n_q\) denote the number of Sylow \(q\)-subgroups.
The \(n_q\) must divide 4, so could be 1, 2 or 4, and must be congruent to 1 mod \(q\).
If \(q \leqslant 3\) then \(G\) could have 4 or 2 Sylow \(q\)-subgroups, but we have already classified those orders,
which is why we took \(q > 3\).
Hence \(G\) has a normal Sylow \(q\)-subgroup, \(Q \cong C_q\).
Let \(x\) generate \(Q\).

Lagrange's Theorem, together with Lemma~\ref{lem:setprodorder}, tell us that a Sylow 2-subgroup, \(T\), intersects
trivially with \(Q\), and \(|QT| = |G|\).
Hence, \(G = Q \rtimes T\).

We know by Lemma~\ref{lem:aut}, that \(\Aut{Q} \cong C_{q-1}\).
So we have two cases:

\begin{enumerate}[\bfseries{Case} 1:]
    \item \(T \cong V_4\) i.e. \(G \cong C_q \rtimes V_4\).

    We saw in our classification of groups of order \(2p\), that \(\units{q}\) has a unique element of order 2,
    corresponding to the inversion map.
    So Lemma~\ref{lem:semiisom} tells us that there is only a single non-trivial homomorphism \(\psi:T \to \Aut{Q}\).

    If \(\psi\) is trivial, then we obtain the product:
    \[G \cong C_q \times V_4 \cong C_{2q} \times C_2\]
    If \(\psi\) is non-trivial, it maps \(T\) to the subgroup generated by the inversion map, isomorphic to \(C_2\).
    Therefore the kernel is isomorphic to \(C_2\), so pick \(z\) such that it generates the kernel.
    Denote the other generator of \(T\) by \(y\), then we obtain the following presentation:
    \[G = \langle\, x,\,y,\,z \mid x^q = y^2 = z^2 = 1,\ yz = zy,\ xz = zx,\ y^{-1}xy = x^{-1}\,\rangle\]
    Now let \(a = xz\), and in a similar calculation to when we classified groups of order 12, we will show that \(G
    \cong D_{4p}\).

    Firstly, notice that the order of \(a\) is \(4q\), and:
    \[a^q = x^q z^q = z \quad \text{and} \quad a^{q-1} = x^{q-1} z^{q-1} = x^{q-1}\]
    Now consider:
    \[y^{-1}ay = y^{-1}xzy = y^{-1}xyz = x^{-1}z = a^{q-1} a^q = a^{2q-1} = a^{-1}\]
    Hence:
    \[G = \langle\, a,\,y \mid a^{2q} = y^2 = 1,\ y^{-1}ay = a^{-1}\,\rangle\]
    which we recognise as \(D_{4p}\).

    \item \(T \cong C_4\) i.e. \(G \cong C_q \rtimes C_4\).

        Let \(t\) generate \(T\).
        Assume \(4 \nmid q-1\), which means \(q \equiv 3 \mod{4}\).
        So then \(\Aut{Q}\) has no subgroup of order 4, and a homomorphism, \(\psi\) must map \(T\) to either the
        trivial group, or the group generated by the inverse automorphism.

        If \(T\psi\) is trivial, then we recover the direct product, \(C_q \times C_4 \cong C_{4q}\).

        If \(T\psi\) is non-trivial, then \(G\) has the presentation:
        \[G = \langle\, x,\,t \mid x^q = t^4 = 1,\ t^{-1}xt = x^{-1}\,\rangle\]
        Let \(a = xt^2\).
        Then:
        \[a^q = xt^2\dots xt^2 = x^q t^{2q} = t^{2q}\]
        We know \(q \equiv 3 \mod{4}\), so for some \(n\), \(q = 4n + 3\).
        Thus \(2q = 8n + 6 = 4(2n + 1) + 2\).
        So then:
        \[a^q = t^{4(2n + 1) + 2} = t^2\]
        Additionally:
        \[t^{-1}at = t^{-1}xt^2t = (t^{-1}xt)t^2 = x^{-1}t^2 = t^2 x^{-1} = a^{-1}\]
        Hence:
        \[G = \langle\, a,\,t \mid a^{2q} = 1,\ a^q = t^2,\ t^{-1}at = a^{-1}\,\rangle\]
        which is the dicyclic group of order \(4q\), \(\Dic_{4q}\).

        If \(4 \mid q-1\), i.e. \(q \equiv 1 \mod{4}\), then \(\Aut{Q}\) contains a unique element of order 4, and so
        has a unique subgroup generated by it.
        We know by Lemma~\ref{lem:aut}, that \(\Aut{Q} \cong \units{q}\), so say \(\alpha\) is the generator of the
        subgroup of order 4 in \(\units{q}\).
        Our homomorphism can map \(T\) to this subgroup, and we get a group with the presentation:
        \[G = \langle\, x,\,t \mid x^q = t^4 = 1,\ t^{-1}xt = x^{\alpha}\,\rangle\]

\end{enumerate}

\subsection{\(2p^2\)}
Let \(G\) be a group of order \(2p^2\), with \(p > 2\).
Denote the number of Sylow \(p\)-subgroups by \(n_p\).
By Sylow's~Theorems, \(n_p\) divides 2, and is congruent to 1 mod \(p\), so must be 1.
Hence, \(G\) has a normal Sylow \(p\)-subgroup, \(P\) of order \(p^2\).

If \(T\) is a Sylow 2-subgroup, then by applying Lagrange's~Theorem, and Lemma~\ref{lem:setprodorder}, we can conclude
that \(G = P \rtimes T\).
From our classification of groups of order \(p^2\), we have 2 choices for \(P\):

\begin{enumerate}[\bfseries{Case} 1:]
    \item \(P \cong C_{p^2}\) i.e. \(G \cong C_{p^2} \rtimes C_2\).

        From Lemma~\ref{lem:aut}, we know \(|\Aut{P}| = p^2 - p = p(p-1)\).
        Because \(p\) is prime, \(2 \nmid p\), but \( 2 \mid p - 1\), so \(\Aut{P}\) has a unique element of order 2.
        Hence, in addition to the direct product, \(G \cong C{2p^2}\), we have \(G \cong C_{p^2} \rtimes C_2 \quad \),
        with \(C_2\) acting by inversion.
        If \(x\) generates \(P\), and \(y\) generates \(T\), we have the presentation:
        \[G = \langle\, x,\,y \mid x^{p^2} = y^2 = 1,\ y^{-1}xy = x^{-1}\,\rangle\]
        which we recognise as \(D_{2p^2}\), the dihedral group of order \(2p^2\).

    \item \(P \cong C_p \times C_p\) i.e. \(G \cong C_p \times C_p \rtimes C_2\).

        Consider \(P\) as the product of the subgroups generated by \(a\) and \(b\), i.e. \(P = \langle a \rangle \times
        \langle b \rangle\).
        Then the action of \(T\) on \(P\) can either be trivial on both subgroups, invert one, or invert both.

        If the action is trivial on both subgroups, then we recover the direct product \(G \cong C_p \times C_{2p}\).

        If the action is non-trivial on just one of the subgroups, then we can consider only one case.
        This is because they are equivalent up to an isomorphism of \(T\), and Lemma~\ref{lem:semiisom} tells us the
        resulting semidirect products are isomorphic.
        So we have:
        \[G = \langle a \rangle \times (\langle b \rangle \rtimes T) \cong C_p \times D_{2p}\]
        
        Finally, if we choose to invert both subgroups, then we act on all of \(P\) by inversion.
        So if \(a\) and \(b\) generate \(P\), then:
        \[G = \langle\, a,\,b,\,x \mid a^p = b^p = x^2 = 1,\ ab = ba,\ x^{-1}ax = a^{-1},\ x^{-1}bx = b^{-1}\,\rangle\]
        Because \(C_p\) has all elements of order p, excluding 1, and they are all \emph{automorphic} to each other (meaning
        that some automorphism maps one to the other), \(x^{-1}gx = g^{-1}\) for all \(g \in P\).
        Hence:
        \[G = \langle\, P,\,x \mid x^2 = 1,\ x^{-1}gx = g^{-1}\ \forall g \in P \,\rangle\]
        which is known as the generalised dihedral group for \(C_p\), denoted \(\Dih(C_p)\).

\end{enumerate}
