% ==== Order p^2q   ====
\section{Order \(4q\)}
As discussed at the beginning of this chapter, we will only classify certain groups of order \(p^2 q\).
First, we will classify groups of order \(4q\) for a prime \(q > 3\), in the next section, groups of order \(2p^2\),
then in the next chapter, return to the special case of groups of order 12.

\begin{theorem}\label{thm:4q}
    For a prime \(q > 3\), any group of order \(4p\) is isomorphic to one of:
    \[%
        C_{4q}, \quad%
        C_{2q} \times C_2, \quad%
        D_{4q} \quad \text{or} \quad%
        \Dic_{4q}
    \]
    \[
        \gen{x,\, t \mid x^q = t^4 = 1,\, t^{-1}xt = x^\alpha} \tag{additionally, if \(4 \mid q-1\)}
    \]
    where \(\alpha\) is the generator of the subgroup of order 4 in \(\units{q}\).
\end{theorem}

We will prove Theorem~\ref{thm:4q} in the rest of this section.
Let \(q > 3\) be a prime number, and \(G\) be a group of order \(4q\).
Denote the number of Sylow \(q\)-subgroups by \(n_q\).
Then \(n_q\) must divide 4, so could be 1, 2 or 4, and must be congruent to 1 mod~\(q\).

If \(q = 3\), then \(G\) could have 4 Sylow \(q\)-subgroups, so we will classify groups of order 12 later.
If \(q = 2\), then we have a group of order \(p^3\), which we have already classified.
This is why we take \(q > 3\).
So \(G\) has a normal Sylow \(q\)-subgroup, \(Q \cong C_q\).
Let \(x\) generate \(Q\).

Lagrange's Theorem, together with Lemma~\ref{lem:setprodorder}, tell us that a Sylow 2-subgroup, \(T\), intersects
trivially with \(Q\), and \(|QT| = |G|\).
Hence, \(G = Q \rtimes T\).

We know by Lemma~\ref{lem:aut}, that \(\Aut{Q} \cong C_{q-1}\).
So we have two cases: \(T \cong V_4\) or \(T \cong C_4\).

\begin{lemma}
    For a prime \(q > 3\), let \(G\) be a group of order \(4q\).
    Then if \(G\) has a Sylow 2-subgroup isomorphic to \(V_{4}\), then \(G\) is isomorphic to one of:
    \[%
        C_{2q} \times C_2 \quad \text{or} \quad%%
        D_{4q}%
    \]
\end{lemma}

\begin{proof}
    We saw in our classification of groups of order \(2p\), that \(\units{q}\) has a unique element of order 2,
    corresponding to the inversion map.
    So Lemma~\ref{lem:semiisom} tells us that there is a single non-trivial homomorphism \(\psi:T \to \Aut{Q}\).

    If \(\psi\) is trivial, then we obtain the product:
    \[G \cong C_q \times V_4 \cong C_{2q} \times C_2\]
    If \(\psi\) is non-trivial, it maps \(T\) to the subgroup generated by the inversion map, isomorphic to \(C_2\).
    Therefore the kernel is isomorphic to \(C_2\), so pick \(z\) such that it generates the kernel.
    Denote the other generator of \(T\) by \(y\), then we obtain the following presentation:
    \[G = \langle\, x,\,y,\,z \mid x^q = y^2 = z^2 = 1,\ yz = zy,\ xz = zx,\ y^{-1}xy = x^{-1}\,\rangle\]
    Now let \(a = xz\), and we will show that \(G \cong D_{4p}\).

    Firstly, notice that the order of \(a\) is \(4q\), and:
    \[a^q = x^q z^q = z \quad \text{and} \quad a^{q-1} = x^{q-1} z^{q-1} = x^{q-1}\]
    Now consider:
    \[y^{-1}ay = y^{-1}xzy = y^{-1}xyz = x^{-1}z = a^{q-1} a^q = a^{2q-1} = a^{-1}\]
    Hence:
    \[G = \langle\, a,\,y \mid a^{2q} = y^2 = 1,\ y^{-1}ay = a^{-1}\,\rangle\]
    which we recognise as \(D_{4q}\).
\end{proof}

\begin{lemma}
    For a prime \(q > 3\) and \(4 \nmid q-1\), let \(G\) be a group of order \(4q\).
    Then if \(G\) has a Sylow 2-subgroup isomorphic to \(C_{4}\), then \(G\) is isomorphic to one of:
    \[%
        C_{4q} \quad \text{or} \quad%
        \Dic_{4q}
    \]
\end{lemma}

\begin{proof}
    Let \(t\) generate \(T\).
    If \(4 \nmid q-1\), then \(q \equiv 3 \mod{4}\).
    So then \(\Aut{Q}\) has no subgroup of order 4, and a homomorphism, \(\psi\) must map \(T\) to either the
    trivial group, or the group generated by the inverse automorphism.

    If \(T\psi\) is trivial, then we recover the direct product, \(C_q \times C_4 \cong C_{4q}\).

    If \(T\psi\) is non-trivial, then \(G\) has the presentation:
    \[G = \langle\, x,\,t \mid x^q = t^4 = 1,\ t^{-1}xt = x^{-1}\,\rangle\]
    Let \(a = xt^2\).
    Then:
    \[a^q = xt^2\dots xt^2 = x^q t^{2q} = t^{2q}\]
    We know \(q \equiv 3 \mod{4}\), so for some \(n\), \(q = 4n + 3\).
    Thus \(2q = 8n + 6 = 4(2n + 1) + 2\).
    So then:
    \[a^q = t^{4(2n + 1) + 2} = t^2\]
    Additionally:
    \[t^{-1}at = t^{-1}xt^2t = (t^{-1}xt)t^2 = x^{-1}t^2 = t^2 x^{-1} = a^{-1}\]
    Hence:
    \[G = \langle\, a,\,t \mid a^{2q} = 1,\ a^q = t^2,\ t^{-1}at = a^{-1}\,\rangle\]
    This is known as the \emph{binary dihedral} or \emph{dicyclic group}\footcite{dicyclic}, denoted \(\Dic_{4q}\).
\end{proof}

\begin{lemma}
    For a prime \(q > 3\) and \(4 \mid q-1\), let \(G\) be a group of order \(4q\).
    Then if \(G\) has a Sylow 2-subgroup isomorphic to \(C_{4}\), then \(G\) is isomorphic to one of:
    \[%
        C_{4q}, \quad%
        \Dic_{4q} \quad \text{or} \quad%
        \gen{x,\, t \mid x^q = t^4 = 1,\, t^{-1}xt = x^\alpha}
    \]
\end{lemma}

\begin{proof}
    Our classification for when \(4 \nmid q-1\) still holds, so we could be isomorphic to:
    \[
        C_{4q} \quad \text{or} \quad%
        \Dic_{4q}%
    \]
    If \(4 \mid q-1\), i.e. \(q \equiv 1 \mod{4}\), then \(\Aut{Q}\) contains a unique element of order 4, and so
    has a unique subgroup generated by it.
    We know by Lemma~\ref{lem:aut}, that \(\Aut{Q} \cong \units{q}\), so say \(\alpha\) is the generator of the
    subgroup of order 4 in \(\units{q}\).
    So we obtain another homomorphism, mapping \(T\) to this subgroup, and we get a group with the presentation:
    \[G = \langle\, x,\,t \mid x^q = t^4 = 1,\ t^{-1}xt = x^{\alpha}\,\rangle\]
\end{proof}

Hence we have proved our result.

\section{Order \(2p^2\)}
\begin{theorem}\label{thm:2p_squared}
    For an odd prime \(p\), any group of order \(2p^2\) is isomorphic to one of:
    \[%
        C_{2p^2}, \quad%
        D_{2p^2}, \quad%
        C_p \times C_{2p}, \quad%
        C_p \times D_{2p} \quad \text{or} \quad%
        \Dih(C_p {\times} C_p)
    \]
\end{theorem}

Again, the remainder of this section will prove this.
Let \(G\) be a group of order \(2p^2\), with \(p > 2\).
Denote the number of Sylow \(p\)-subgroups by \(n_p\).
By Sylow's~Theorems, \(n_p\) divides 2, and is congruent to 1 mod \(p\), so must be 1.
Hence, \(G\) has a normal Sylow \(p\)-subgroup, \(P\) of order \(p^2\).

If \(T\) is a Sylow 2-subgroup, then by applying Lagrange's~Theorem, and Lemma~\ref{lem:setprodorder}, we can conclude
that \(G = P \rtimes T\).
From our classification of groups of order \(p^2\), we have 2 choices for \(P\): \(C_{p^2}\) or \(C_{p} \times C_{p}\).

\begin{lemma}
    For an odd prime \(p\), let \(G\) be a group of order \(2p^2\). Then if \(G\) has a Sylow \(p\)-subgroup isomorphic
    to \(C_{p^2}\), \(G\) is isomorphic to one of:
    \[%
        C_{p^2} \quad \text{or} \quad%
        D_{2p^2} \quad%
    \]
\end{lemma}

\begin{proof}
        From Lemma~\ref{lem:aut}, we know \(|\Aut{P}| = p^2 - p = p(p-1)\).
        Because \(p\) is prime, \(2 \nmid p\), but \( 2 \mid p - 1\), so \(\Aut{P}\) has a unique element of order 2.
        Hence, a homomorphism \(\psi:T \to \Aut{P}\) has image isomorphic to either \(\bm{1}\) or \(C_{2}\).

        So in addition to the direct product, \(G \cong C_{2p^2}\), for a trivial homomorphism, we have \(G \cong
        C_{p^2} \rtimes C_2\), with \(C_2\) acting by inversion.
        If \(x\) generates \(P\), and \(y\) generates \(T\), we have the presentation:
        \[G = \langle\, x,\,y \mid x^{p^2} = y^2 = 1,\ y^{-1}xy = x^{-1}\,\rangle\]
        which we recognise as \(D_{2p^2}\), the dihedral group of order \(2p^2\).
\end{proof}

\begin{lemma}
    For an odd prime \(p\), let \(G\) be a group of order \(2p^2\). Then if \(G\) has a Sylow \(p\)-subgroup isomorphic
    to \(C_p \times C_{p}\), \(G\) is isomorphic to one of:
    \[%
        C_p \times C_{2p}, \quad%
        C_p \times D_{2p} \quad \text{or} \quad%
        \Dih(C_p {\times} C_p)
    \]
\end{lemma}

\begin{proof}
        Consider \(P\) as the product of the subgroups generated by \(a\) and \(b\), i.e. \(P = \langle a \rangle \times
        \langle b \rangle\).
        Then the action of \(T\) on \(P\) can either be trivial on both subgroups, invert one, or invert both.

        If the action is trivial on both subgroups, then we recover the direct product \(G \cong C_p \times C_{2p}\).

        If the action is non-trivial on just one of the subgroups, then we can consider only one case.
        This is because they are equivalent up to an isomorphism of \(T\), and Lemma~\ref{lem:semiisom} tells us the
        resulting semidirect products are isomorphic.
        So we have:
        \[G = \langle a \rangle \times (\langle b \rangle \rtimes T) \cong C_p \times D_{2p}\]
        Finally, if we choose to invert both subgroups, then we act on all of \(P\) by inversion.
        So if \(a\) and \(b\) generate \(P\), then:
        \[G = \langle\, a,\,b,\,x \mid a^p = b^p = x^2 = 1,\ ab = ba,\ x^{-1}ax = a^{-1},\ x^{-1}bx = b^{-1}\,\rangle\]
        Because \(C_p\) has all elements of order p, excluding 1, and they are all \emph{automorphic} to each other (meaning
        that some automorphism maps one to the other), \(x^{-1}gx = g^{-1}\) for all \(g \in P\).
        Hence:
        \[G = \langle\, P,\,x \mid x^2 = 1,\ x^{-1}gx = g^{-1}\ \forall g \in P \,\rangle\]
        which is known as the \emph{generalised dihedral group}\footcite{gendihedral} for \(P\), denoted \(\Dih(P)\).
\end{proof}

And so Theorem~\ref{thm:2p_squared} is proven.

