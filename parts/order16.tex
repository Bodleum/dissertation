% ====   Order 16   ====
\section{Groups of Order 16}
The following classification is based of the one by Wild\footcite{order16}.
For brevity, denote \(C_2 \times C_2 \times C_2 \times C_2\) by \({(C_2)}^4\).
We begin by proving a lemma:

\begin{lemma}\label{lem:order_16_subgp}
    For a group, \(G\), with order 16, not isomorphic to \({(C_2)}^4\), then \(G\) has a
    normal subgroup isomorphic to either \(C_8\) or \(C_4 \times C_2\).
\end{lemma}

\begin{proof}
    Firstly, we know that a subgroup with index 2 is normal, so any subgroup of order 8 in \(G\) is normal.
    So it remains to show \(G\) possesses such subgroups.
    If some \(x \in G\) has order 8, then \(\gen{x} \cong C_8\) and we are done.
    So assume \(G\) has no element of order 8.

    Because \(G \ncong {(C_2)}^4\), there is at least one element of order 4 in \(G\), say \(y\).
    By Lemma~\ref{lem:z_non_trivial}, there is some element \(z \in \z(G)\) which has order 2.
    Let \(H = \gen{z}\).
    Moreover, \(H \nrmsgp G\) because \(z\) is in the centre.

    If \(y^2 \neq z\), then \(\gen{y} \cap H = \bm{1}\), so \(\gen{y,\,z} \cong C_4 \times C_2\).

    If then all elements of order 4 in \(G\) have \(y^2 = z\), all elements in \(G/H\) have order 2.
    Thus \(G/H \cong {(C_2)}^3\), and in particular, is abelian.
    So the conjugacy class of \(y\), \(y^G\) is a subgroup of \(yH\).
    Hence the centraliser of \(y\), \(\C_G(y)\), has order 8.

    Let \(g \in \C_G(y) \setminus \gen{y}\).
    If \(g\) has order 2, then \(\gen{y,\,g} \cong C_4 \times C_2\).
    If \(g\) has order 4, then:
    \[g^2 = z \quad \text{and} \quad {(yg)}^2 = y^2 g^2 = z^2 = 1\]
    We can see that \(yg \notin \gen{y}\) because then, \(yg = y^2\) which gives the contradiction \(y = g\).
    So \(\gen{y,\,yg} \cong C_4 \times C_2\).
\end{proof}

Since we know already that \({(C_2)}^4\) is a group of order 16 by the Fundamental Theorem of Finite Abelian Groups, we
will concentrate on when \(G\) is not isomorphic to \({(C_2)}^4\).
If \(N\) is the subgroup of \(G\) specified by Lemma~\ref{lem:order_16_subgp}, we will classify the groups by extending
\(N\) in different ways.
Pick some element \(a \in G \setminus N\).
Then the order of \(aN\) is two.
(It must be either 1 or 2, and cannot be 1 because that contradicts our choice of \(a\).)
Let \(v = a^2\).
Notice that \(v \in N\). % TODO: Why?

Let \(t_a:x \mapsto a^{-1}xa\) denote conjugation by \(a\).
So now restricting \(t_a\) to \(N\), we obtain an automorphism of \(N\).
Denote this by \(\tau\).
Notice:
\[v\tau = aa^2 a^{-1} = a^2 = v \quad \text{and} \quad x\tau^2 = a^{-1}(a^{-1}xa)a = v^{-1}xv = xt_v \tag{for all \(x \in
N\)}\]
We know already what the possible automorphisms of \(C_8\) and \(C_4 \times C_2\) are from our classification of groups
of order 24.
So now by considering choices for the order of \(a\) and automorphisms of \(N\), we will classify \(G\).
Begin with \(\gen{x} = N \cong C_8\).

\begin{enumerate}[\bfseries{Case} 1:]
    \item Order 2.

        If \(a\) has order 2, then \(v\) must be 1.
        So then if \(\gen{a} = H\), then \(N \cap H = \bm{1}\), so \(G = N \rtimes H\).
        We have 4 possible semidirect products, because each of the automorphisms of \(N\) have order 2 (excluding the
        identity map).
        Hence:
        \begin{align*}
        G &= \gen{x,\,a \mid x^8 = a^2 = 1,\ a^{-1}xa = a} \cong C_8 \times C_2 \\
        G &= \gen{x,\,a \mid x^8 = a^2 = 1,\ a^{-1}xa = a^{3}} \cong \SD_{16}\\
        G &= \gen{x,\,a \mid x^8 = a^2 = 1,\ a^{-1}xa = a^{5}} \cong \M_{16}\\
        G &= \gen{x,\,a \mid x^8 = a^2 = 1,\ a^{-1}xa = a^{-1}} \cong D_{16}
        \end{align*}
        The group \(\SD_{16}\) is known as the \emph{semidihedral group}, of order 16\footcite{semidihedral}, and
        \(\M_{16}\) is called the \emph{modular group} of order 16\footcite{order16names}.

    \item Order 4.

        Here, \(v\) must be an element of order 2 in \(N\), which can only be \(x^4\).
        Lemma~\ref{lem:semiisom} lets us assume that in this case, all elements in \(G\setminus N\) have order at least
        4.
        How does conjugation behave?
        In particular, what is \(a^-1ga\) for an element \(g \in G\)?
        I claim the only possibility is the automorphism \(x\mapsto x^{-1}\).
        It cannot be \(x\mapsto x^3\) because then:
        \[(xa)(xa) = x(axa^{-1})a^2 = x(x^3) a^2 = x^4 a^2 = 1\]
        Likewise, it cannot be \(\iota:x\mapsto x\) or \(\psi:x\mapsto x^5\) because then \(x^2 a\) will have order 2:
        \[{(x^2 a)}^2 = x^2 (ax^2 a^{-1})a^2 = x^2 {(x^2)}^5 a^2 = x^4 a^2 = 1\]
        All of which contradict our assumption.
        Hence the only posibility for the effect of conjugation by \(a\) is the map \(x\mapsto x^7 = x^{-1}\).
        So we obtain the presentation:
        \[G = \gen{x,\,a \mid x^8 = a^4 = 1,\ a^2 = x^4,\ a^{-1}xa = x^{-1}}\]
        Which we recognise as the dicyclic group, \(\Dic_{24}\).

    \item Order 8.

        Here, \(v\) has order 4, so could be either \(x^2\) or \(x^6\).
        It turns out, we obtain no new groups in this case.
        Because \(x^2\) and \(x^6\) are automorphic, we only need to consider one case, say \(v = x^2\), and apply
        Lemma~\ref{lem:semiisom}.
        If conjugation by \(a\) is either \(x\mapsto x^3\) or \(x\mapsto x^7\) then we obtain a contradiction:
        \[a^2 = a^{-1}a^2 a = a^{-1}x^2 a = x^6\]
        \[a^2 = a^{-1}a^2 a = a^{-1}x^2 a = x^{14} = x^6\]
        Hence the only possibilities for conjugation by \(a\) are the maps \(x\mapsto x\) and \(x \mapsto x^5\).
        For the first:
        \[(x^3 a)(x^3 a) = x^3(ax^3 a^{-1})a^2 = x^3 x^3 a^2 = x^8 = 1\]
        a contradiction.
        And the second:
        \[(xa)(xa) = x(axa^{-1})a^2 = x(x^5)a^2 = x^8 = 1\]
        another contradiction.
        Hence we obtain no new groups.

    \item Order 16.

        The only possibility here is \(G \cong C_{16}\), generated by \(a\).
\end{enumerate}

Now let's move on to consider when \(\gen{x} \cong C_4 \times C_2\).
