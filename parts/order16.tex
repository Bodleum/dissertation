% ====   Order 16   ====
\section{Groups of Order 16}
And finally, all that remains is to classify groups of order 16.
This classification is often not included; \textit{MT4003} classifies groups up to order 15 for example, however it
appears in Burnside's \textit{Theory of Groups}\footcite[p145]{burnside1911} on page 145.
So to cap of this report, let's tackle one of the hardest cases!
For brevity, denote \(C_2 {\times} C_2 {\times} C_2 {\times} C_2\) by \({(C_2)}^4\).


\begin{theorem}
    Any group of order 16 is isomorphic to one of:
    \begin{displaymath}
    \begin{array}{c@{\hskip 3em}c@{\hskip 3em}c}
        {(C_2)}^4 & C_8 {\times} C_2 & \SD_{16} \\
        \M_{16} & D_{16} & \Dic_{16} \\
        C_{16} & C_4 {\times} C_2 {\times} C_2 & D_8 {\times} C_2 \\
        (C_4 {\times} C_2) {\rtimes} C_2 & \text{Pauli Group} & Q_8 {\times} C_2 \\
        C_4 {\rtimes} C_4 & C_4 {\times} C_4 & \\
    \end{array}
    \end{displaymath}
\end{theorem}

The rest of this section is the proof of this classification, and is based of the one by Wild\footcite{order16}.
We begin by proving a lemma:

\begin{lemma}\label{lem:order_16_subgp}
    \raggedright
    Let \(G\) be a group of order 16, not isomorphic to \({(C_2)}^4\), then \(G\) has a
    normal subgroup isomorphic to either \(C_8\) or \(C_4 \times C_2\).
\end{lemma}

\begin{proof}
    Firstly, we know that a subgroup with index 2 is normal, so any subgroup of order 8 in \(G\) is normal.
    So it remains to show \(G\) possesses such subgroups.
    If some \(x \in G\) has order 8, then \(\gen{x} \cong C_8\) and we are done.
    So assume \(G\) has no element of order 8.

    Because \(G \ncong {(C_2)}^4\), there is at least one element of order 4 in \(G\), say \(y\).
    By Lemma~\ref{lem:z_non_trivial}, there is some element \(z \in \z(G)\) which has order 2.
    Let \(H = \gen{z}\).
    Moreover, \(H \nrmsgp G\) because \(z\) is in the centre.

    If \(y^2 \neq z\), then \(\gen{y} \cap H = \bm{1}\), so \(\gen{y,\,z} \cong C_4 \times C_2\).

    If then all elements of order 4 in \(G\) have \(y^2 = z\), all elements in \(G/H\) have order 2.
    Thus \(G/H \cong {(C_2)}^3\), and in particular, is abelian.
    So the conjugacy class of \(y\) is a subgroup of \(yH\).
    Hence the centraliser of \(y\), \(\C_G(y)\), has order 8.
    Let \(g \in \C_G(y) \setminus \gen{y}\).
    If \(g\) has order 2, then \(\gen{y,\,g} \cong C_4 \times C_2\).
    If \(g\) has order 4, then:
    \[g^2 = z \quad \text{and} \quad {(yg)}^2 = y^2 g^2 = z^2 = 1\]
    We can see that \(yg \notin \gen{y}\) because then, \(yg = y^2\) which gives the contradiction \(y = g\).
    So \(\gen{y,\,yg} \cong C_4 \times C_2\).
\end{proof}

Since we know already that \({(C_2)}^4\) is a group of order 16 by the Fundamental Theorem of Finite Abelian Groups, we
will concentrate on when \(G\) is not isomorphic to \({(C_2)}^4\).
If \(N\) is the subgroup of \(G\) specified by Lemma~\ref{lem:order_16_subgp}, we will classify the groups by extending
\(N\) in different ways.
Pick some element \(a \in G \setminus N\).
Then the order of \(aN\) is two, (it must be either 1 or 2, and cannot be 1 because that contradicts our choice of
\(a\)) and \(a^2 {\in} N\).

We know already what the possible automorphisms of \(C_8\) and \(C_4 \times C_2\) are from our classification of groups
of order 24.
They are summarised in Tables~\ref{tab:aut_C8}~and~\ref{tab:aut_C4xC2} in the appendix.
So now by considering choices for the order of \(a\) and automorphisms of \(N\), we will classify \(G\).

\begin{lemma}
    Let \(G\) be a group of order 16, not isomorphic to \({(C_{2})}^4\).
    Suppose \(G\) has a normal subgroup \(N \cong C_8\), and element \(a \in G\setminus N\) of order 2.
    Then \(G\) is isomorphic to one of:
    \[
        C_8 \times C_2, \quad%
        \SD_{16}, \quad%
        \M_{16} \quad \text{or} \quad%
        D_{16}
    \]
\end{lemma}

\begin{proof}
    If \(a\) has order 2, then \(a^2\) must be 1.
    So then \(N \cap \gen{a} = \bm{1}\), and \(G = N \rtimes \gen{a}\).
    We have 4 possible semidirect products, because each automorphism of \(N\) has order 2 (excluding the
    identity map).
    Hence we obtain the following 4 groups:
    \begin{equation*}
    \begin{aligned}
        G_1 &= \gen{x,\,a \mid x^8 = a^2 = 1,\ a^{-1}xa \overset{\phi_1}{=} x} \\
            &= \gen{x} \times \gen{a} \\
            &\cong C_8 \times C_2 \\
            & \\
        G_2 &= \gen{x,\,a \mid x^8 = a^2 = 1,\ a^{-1}xa \overset{\phi_2}{=} x^{3}} \\
            &= \gen{x} \rtimes_3 \gen{a} \\
            &\cong \SD_{16}\\
            & \\
        G_3 &= \gen{x,\,a \mid x^8 = a^2 = 1,\ a^{-1}xa \overset{\phi_3}{=} x^{5}} \\
            &= \gen{x} \rtimes_5 \gen{a} \\
            &\cong \M_{16}\\
            & \\
        G_4 &= \gen{x,\,a \mid x^8 = a^2 = 1,\ a^{-1}xa \overset{\phi_4}{=} x^7 = x^{-1}} \\
            &= \gen{x} \rtimes_7 \gen{a} \\
            &\cong D_{16}
    \end{aligned}
    \end{equation*}
    We are subscripting \(\rtimes\) with the power of \(x\) it gets sent to upon conjugation with \(a\).
    The group \(\SD_{16}\) is known as the \emph{semidihedral group}\footcite{semidihedral}, of order 16, and
    \(\M_{16}\) is called the \emph{modular group}\footcite{order16names} of order 16.
\end{proof}

\begin{lemma}
    Let \(G\) be a group of order 16, not isomorphic to \({(C_{2})}^4\).
    Suppose \(G\) has a normal subgroup \(N \cong C_8\), and element \(a \in G\setminus N\) of order 4.
    Then \(G\) is isomorphic to \(\Dic_{16}\).
\end{lemma}

\begin{proof}
    Assume all elements in \(G\setminus N\) have order at least 4, otherwise we are back in the previous lemma.
    How does conjugation behave?
    In particular, what is \(a^{-1}ga\) for an element \(g \in G\)?
    I claim the only possibility is the automorphism \(\phi_4:x\mapsto x^{-1}\).
    It cannot be \(\phi_2:x\mapsto x^3\) because then:
    \[(xa)(xa) = x(axa^{-1})a^2 \overset{\phi_2}{=} x(x^3) a^2 = x^4 a^2 = 1\]
    Likewise, it cannot be \(\phi_1:x\mapsto x\) or \(\phi_3:x\mapsto x^5\) because then \(x^2 a\) will have order 2:
    \[{(x^2 a)}^2 = x^2 (ax^2 a^{-1})a^2 \overset{\phi_1}{=} x^4 a^2 = 1\]
    \[{(x^2 a)}^2 = x^2 (ax^2 a^{-1})a^2 \overset{\phi_3}{=} x^2 {(x^2)}^5 a^2 = x^4 a^2 = 1\]
    All of which contradict our assumption.
    Hence the only posibility for the effect of conjugation by \(a\) is the map \(\phi_4:x\mapsto x^7 = x^{-1}\).
    So we obtain the presentation:
    \begin{align*}
        G_5 &= \gen{x,\,a \mid x^8 = a^4 = 1,\ a^2 = x^4,\ a^{-1}xa \overset{\phi_4}{=} x^{-1}}
    \end{align*}
    Which we recognise as the dicyclic group, \(\Dic_{16}\).
\end{proof}

\begin{lemma}
    Let \(G\) be a group of order 16, not isomorphic to \({(C_{2})}^4\).
    There are no groups \(G\) which have a normal subgroup \(N \cong C_8\), and element \(a \in G\setminus N\) of order 8.
\end{lemma}

\begin{proof}
    We have two choices for \(a\): either \(a^2 = x^2\) or \(a^2 = x^6\).
    Because \(x^2\) and \(x^6\) are automorphic, we only need to consider one case, say \(a^2 = x^2\), and apply
    Lemma~\ref{lem:semiisom}.
    If conjugation by \(a\) is either \(\phi_2:x\mapsto x^3\) or \(\phi_4:x\mapsto x^7\) then we obtain a
    contradiction:
    \[a^2 = a^{-1}a^2 a = a^{-1}x^2 a \overset{\phi_2}{=} x^6\]
    \[a^2 = a^{-1}a^2 a = a^{-1}x^2 a \overset{\phi_4}{=} x^{14} = x^6\]
    Hence the only possibilities for conjugation by \(a\) are the maps \(\phi_1:x\mapsto x\) and \(\phi_3:x \mapsto
    x^5\).
    For the first:
    \[(x^3 a)(x^3 a) = x^3(ax^3 a^{-1})a^2 \overset{\phi_1}{=} x^3 x^3 a^2 = x^8 = 1\]
    a contradiction.
    And the second:
    \[(xa)(xa) = x(axa^{-1})a^2 \overset{\phi_3}{=} x(x^5)a^2 = x^8 = 1\]
    another contradiction.
    Hence we obtain no new groups.
\end{proof}

\begin{lemma}
    Let \(G\) be a group of order 16, not isomorphic to \({(C_{2})}^4\).
    Suppose \(G\) has a normal subgroup \(N \cong C_8\), and element \(a \in G\setminus N\) of order 16.
    Then \(G\) is isomorphic to \(C_{16}\).
\end{lemma}

\begin{proof}
    The only possibility here is \(G_6 \cong C_{16}\), generated by \(a\).
\end{proof}


Now let's move on to consider when \(\gen{x,\,y} = N \cong C_4 \times C_2\).
In particular, all elements of \(G\) have order less than 8.

\begin{lemma}
    Let \(G\) be a group of order 16, not isomorphic to \({(C_{2})}^4\).
    Suppose \(G\) has a normal subgroup \(N \cong C_4 \times C_2\), and element \(a \in G\setminus N\).
    There are no groups \(G\) such that conjugation by \(a\) is of order 4.
\end{lemma}

\begin{proof}
    The automorphisms of order 4 are \(\psi_2\) and \(\psi_4\).
    Consider an element \(g \in N\).
    We can write \(g = x^i y^j\) for some \(0 \leqslant i \leqslant 3\) and \(0 \leqslant j \leqslant 1\).
    Then on the one hand:
    \[a^{-2}ga^2 = g\]
    Because \(N\) is abelian.
    On the other:
    \begin{align*}
        a^{-2}ga^2 &= a^{-1}(a^{-1}x^i\,y^j a)a \\
        &= a^{-1}(x^i\psi_2\ y^j\psi_2)a \\
        &= a^{-1}(x^{-i}y^i\ x^{2j}y^j)a \\
        &= a^{-1}(x^{2j-i}\ y^{i+j})a \\
        &= x^{i-2j}y^{2j-i}\ x^{2i+2j}y^{i+j} \\
        &= x^{3i}y^{3j} \\
        &= y^j x^{-i}
    \end{align*}
    Which is \(g^{-1}\).
    Hence we have \(g = g^{-1}\) which is a contradiction, because \(x \in N\) has order 4.
    A similar contradiction can be shown for \(\psi_4\).
\end{proof}

\begin{lemma}
    Let \(G\) be a group of order 16, not isomorphic to \({(C_{2})}^4\).
    Suppose \(G\) has a normal subgroup \(N \cong C_4 \times C_2\), and element \(a \in G\setminus N\) of order 2.
    Then \(G\) is isomorphic to one of:
    \[
        C_4 \times C_2 \times C_2, \quad%
        D_8 \times C_2, \quad%
        (C_4 \times C_2) \rtimes C_2 \quad \text{or} \quad%
        \text{the Pauli Group}
    \]
    The Pauli Group has presentition:
    \[
        \gen{x,\,y,\,a \mid x^4 = y^2 = a^2 = 1,\ xy = yx,\ xa = ax,\ a^{-1}ya = x^2 y}
    \]
\end{lemma}

\begin{proof}
    If \(a\) has order 2, then \(a^2 = 1\), so \(\gen{a} \cap N = \bm{1}\).
    Hence \(G = N \rtimes \gen{a}\).
    So we can apply Lemma~\ref{lem:semiisom}, and only consider one representative from each conjugacy class.
    We have the following presentations:
    \begin{align*}
        G_7 &= \gen{x,\,y,\,a \mid x^4 = y^2 = a^2 = 1,\ xy = yx,\ a^{-1}xa \overset{\psi_1}{=} x,\ a^{-1}ya \overset{\psi_1}{=} y} \\
            &= \gen{x} \times \gen{y} \times \gen{a} \\
            &\cong C_4 \times C_2 \times C_2 \\
            & \\
        G_8 &= \gen{x,\,y,\,a \mid x^4 = y^2 = a^2 = 1,\ xy = yx,\ a^{-1}xa \overset{\psi_3}{=} x^{-1},\ a^{-1}ya \overset{\psi_3}{=} y} \\
            &= (\gen{x} \rtimes \gen{a}) \times \gen{y} \\
            &\cong D_8 \times C_2 \\
            & \\
        G_9 &= \gen{x,\,y,\,a \mid x^4 = y^2 = a^2 = 1,\ xy = yx,\ a^{-1}xa \overset{\psi_5}{=} xy,\ a^{-1}ya \overset{\psi_5}{=} y} \\
            &= (\gen{x} \times \gen{y}) \rtimes \gen{a} \\
            &\cong (C_4 \times C_2) \rtimes C_2 \tag{nameless} \\
            & \\
        G_{10} &= \gen{x,\,y,\,a \mid x^4 = y^2 = a^2 = 1,\ xy = yx,\ a^{-1}xa \overset{\psi_8}{=} x,\ a^{-1}ya \overset{\psi_8}{=} x^2 y}\\
    \end{align*}
    The final group, \(G_{10}\), is sometimes called the \emph{Pauli Group}\footcite{order16names}, because it
    consists of the Pauli Matrices from quantum mechanics, and their products with powers of \(i\).

    The group \(G_9\) doesn't appear to have any significant name.
\end{proof}

\begin{lemma}
    Let \(G\) be a group of order 16, not isomorphic to \({(C_{2})}^4\).
    Suppose \(G\) has a normal subgroup \(N \cong C_4 \times C_2\), generated by \(x\) of order 4, and \(y\) of order 2.
    Further suppose that \(a \in G\setminus N\) has order 4 with \(a^2 = x^2\).
    Then \(G\) is isomorphic to one of:
    \[
        Q_8 \times C_2 \quad \text{or} \quad%
        C_4 \rtimes C_{4}
    \]
\end{lemma}

\begin{proof}
    Firstly, let's show what conjugation by \(a\) cannot be.
    If it's \(\psi_1\):
    \[(xa)(xa) = xa^2(a^{-1}xa) \overset{\psi_1}{=} xa^2 x = 1\]
    If it's \(\psi_6\):
    \[(xya)(xya) = xya^2(a^{-1}xya) \overset{\psi_6}{=} xya^2(x^{-1}x^2y) = x^4 y^2 = 1\]
    If it's \(\psi_8\):
    \[(x^2 ya)(x^2 ya) = x^2 ya^2 (a^{-1}x^2 ya) \overset{\psi_8}{=} x^2 ya^2(x^2 x^2 y) = x^8 y^2 = 1\]
    All of which are contradictions to our assumption.
    So the remaining possibilities are:
    \begin{align*}
        G_{11} &= \gen{x,\,y,\,a \mid x^4 = y^2 = a^4 = 1,\ xy = yx,\ a^{-1}xa \overset{\psi_3}{=} x^{-1},\ a^{-1}ya \overset{\psi_3}{=} y} \\
            &= \gen{x,\,a} \times \gen{y} \\
            &\cong Q_8 \times C_2 \\
            & \\
        G_{12a} &= \gen{x,\,y,\,a \mid x^4 = y^2 = a^4 = 1,\ xy = yx,\ a^{-1}xa \overset{\psi_5}{=} xy,\ a^{-1}ya \overset{\psi_5}{=} y} \\
        G_{12b} &= \gen{x,\,y,\,a \mid x^4 = y^2 = a^4 = 1,\ xy = yx,\ a^{-1}xa \overset{\psi_7}{=} x^{-1}y,\ a^{-1}ya \overset{\psi_7}{=} y}
    \end{align*}
    Concentrating on the latter two groups, we will show they are both isomorphic to \(C_4 \rtimes C_4\),
    with the inversion action.
    First, \(G_{12a}\).
    Consider the element \(ax\), and so \(\gen{ax} = \{\,1,\,ax,\,y,\,axy\,\}\).
    By inspection, \(\gen{x} \cap \gen{ax} = \bm{1}\).
    So then:
    \[x^{-1}(ax)x = x^{-1}xaxy = axy = {(ax)}^{-1}\]
    Hence, \(G_{12a} = \gen{x} \rtimes \gen{ax} \cong C_4 \rtimes C_4\).

    Likewise for \(G_{12b}\), \(\gen{ax} = \{\,1,\,ax,\,a^2 y,\,x^{-1}a^{-1}\,\}\), and again \(\gen{x} \cap
    \gen{ax} = \bm{1}\).
    Additionally:
    \[x^{-1}(ax)x = ax^3 y = ax^{-1}y = xa\]
    Multiplying by \(x^4 = 1\) gives:
    \[x^4 xa = x^{-1}x^2 a = x^{-1}a^2 a = x^{-1}a^{-1}\]
    Hence \(G_{12b}\) is isomorphic to the same semidirect product, \(C_4 \rtimes C_4\).

    We can check that it is indeed valid to write \(C_4 \rtimes C_4\).
    We know \(\Aut{C_4} \cong C_2\) and so a homomorphism \(\varphi:C_4 \to \Aut{C_4}\) can map the
    generator to either the identity map or the inverse map.
    Hence we have only one non-trivial semidirect product.
\end{proof}

\begin{lemma}
    Let \(G\) be a group of order 16, not isomorphic to \({(C_{2})}^4\).
    Suppose \(G\) has a normal subgroup \(N \cong C_4 \times C_2\), generated by \(x\) of order 4, and \(y\) of order 2.
    Further suppose that \(a \in G\setminus N\) has order 4 with \(a^2 = y\).
    Then \(G\) is isomorphic to \(C_4 \times C_4\).
\end{lemma}

\begin{proof}
    Conjugation by \(a\) cannot be \(\psi_6\) or \(\psi_8\) because then:
    \[a^2 = a^{-1}a^2 a = a^{-1}ya \overset{\psi_{6,8}}{=} x^2 y\]
    a contradiction.
    If it's \(\psi_5\) then let \(\bar{a} = xa\) and so:
    \[(xa)(xa) = x(axa^{-1})a^2 \overset{\psi_5}{=} x^2 y^2 = x^2\]
    and we are in the previous case.
    Likewise if conjugation by \(a\) is \(\psi_3\) then:
    \[(x^2 a)(x^2 a) = x^2(ax^2 a^{-1})a^2 \overset{\psi_3}{=} x^2 y^2 = x^2\]
    Finally, \(\psi_7\) gives:
    \[(xa)(xa) = xa^2(a^{-1}xa) \overset{\psi_7}{=} xa^2 x^3 y = x^4 y^2 = 1\]
    a contradiction, leaving only \(\psi_1\) as the final standing possibility, giving:
    \[G_{13} = \gen{x,\,y,\,a \mid x^4 = y^2 = a^4 = 1,\ xy = yx,\ ax \overset{\psi_1}{=} xa,\ ay \overset{\psi_1}{=} ya}\]
    Letting \(b = xy\):
    \[G_{13} = \gen{b,\,a \mid b^4 = a^4 = 1,\ ab = ba} \cong C_4 \times C_4\]
\end{proof}

\begin{lemma}
    Let \(G\) be a group of order 16, not isomorphic to \({(C_{2})}^4\).
    Suppose \(G\) has a normal subgroup \(N \cong C_4 \times C_2\), generated by \(x\) of order 4, and \(y\) of order 2.
    Further suppose that \(a \in G\setminus N\) has order 4 with \(a^2 = x^2 y\).
    Then \(G\) will be isomorphic to \(C_4 \times C_4\), as in the previous lemma.
\end{lemma}

\begin{proof}
    For any group in this case, if we apply the automorphism \(\psi_8\), then we have:
    \[a^2\psi_8 = (x^2 y)\psi_8 = x^2 x^2 y = y\]
    So it will be isomorphic to a group from the previous subcase.
\end{proof}

So we prove our classification with these lemmas.
Thus we finish not only the classification of groups of order 16, but this report as a whole.
