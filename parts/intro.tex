% ====   Introduction   ====
The study of groups is an important area of mathematics, and as such, it's quite useful to have quick examples of groups
`in your back pocket' so to speak; even more so, to have an exhaustive list of the possible behaviours of groups.
We do this by the notion of an isomorphism class, a kind of equivalence between groups which are essentially the same.
However as groups increase in size, this becomes increasingly hard to do, especially for so called `\(p\)-groups':
groups which have order the power of a prime.
For example, up to isomorphism, there are 51 possible groups of order 32 --- wow!

This report will cover the classification, and proof thereof, of groups of order up to, and including, 31.
In doing so, we will further our understanding of constructing groups, and where possible, find out the names given to
the groups we come across by the wider world of group theory.

To start, let's solidify the notation used in this report.
We shall denote groups and sets with capital letters, like \(G\), \(H\), and elements of those groups with lower case
letters, like \(g\), \(h\).
Greek letters shall denote mappings, generally \(\phi\), \(\psi\), etc.\ with \(\iota\) reserved for the identity map,
and we will write mappings on the right.

To denote the cyclic group of order \(n\) we will use \(C_n\), \(D_{2n}\) to denote the dihedral group of order \(2n\),
\(A_n\) to denote the alternating group over \(n\) elements, \(S_n\) to denote the symmetric group over \(n\)
elements, and \(Q_8\) to denote the quaternion group.
The trivial group, \(\{\, 1\, \}\) is denoted by \(\bm{1}\).
We will meet other groups as we go on our journey of classification!

This report consists of 4 main movements: first we will review and expand our knowledge of group theory, and then in the
subsequent 3 chapters, we will classify our groups.
We will first tackle \(p\)-groups up to \(p^3\), then groups which have composite order, specifically \(pq\), \(4q\) and
\(2p^2\), before filling in the gaps we missed, classifying groups of orders 12, 24, 30 and 16 in the final chapter.

