% ====   Introduction   ====
\section{Abstract}
Table~\ref{tab:abstract} shows the classification for all groups up to order 31.
Note: not all presentations are included, specifically long ones, and those not especially relevant.

\begin{table}
    \caption{Classification of Groups up to Order 31}\label{tab:abstract}
    \begin{center}
        \begin{displaymath}
\begin{array}{c|c|l}
    \toprule
    \text{Order} & \text{Classification} & \text{Presentation} \\
    \midrule
    1 & \bm{1} & \gen{1} \\
    \midrule
    2 & C_2 & \gen{x \mid x^2 = 1} \\
    \midrule
    3 & C_3 & \gen{x \mid x^3 = 1} \\
    \midrule
    4 & C_4 & \gen{x \mid x^4 = 1} \\
      & C_2 \times C_2 & \gen{x,\,y \mid x^2 = y^2 = 1,\, xy = yx} \\
    \midrule
    5 & C_5 & \gen{x \mid x^5 = 1} \\
    \midrule
    6 & D_6 \cong S_3 & \gen{x,\,y \mid x^3 = y^2 = 1,\, y^{-1}xy = x^{-1}} \\
      & C_6 & \gen{x \mid x^6 = 1} \\
    \midrule
    7 & C_7 & \gen{x \mid x^7 = 1} \\
    \midrule
    8 & C_8 & \gen{x \mid x^8 = 1} \\
      & C_4 \times C_2 & \gen{x,\,y \mid x^4 = y^2 = 1,\, xy = yx} \\
      & D_8 & \gen{x,\,y \mid x^4 = y^2 = 1,\, y^{-1}xy = x^{-1}} \\
      & Q_8 & \gen{x,\,y \mid x^4 = y^4 = 1,\, x^2 = y^2,\, y^{-1}xy = x^{-1}} \\
      & C_2 \times C_2 \times C_2 & \gen{x,\,y,\,z \mid x^2 = y^2 = z^2 = 1,\, xy = yx,\, xz = zx,\, yz = zy} \\
    \midrule
    9 & C_9 & \gen{x \mid x^9 = 1} \\
      & C_3 \times C_3 & \gen{x,\,y \mid x^3 = y^3 = 1,\, xy = yx} \\
    \midrule
    10 & D_{10} & \gen{x,\,y \mid x^5 = y^2 = 1,\, y^{-1}xy = x^{-1}} \\
       & C_{10} & \gen{x \mid x^{10} = 1} \\
    \midrule
    11 & C_{11} & \gen{x \mid x^{11} = 1} \\
    \midrule
    12 & \Dic_{12} & \gen{x,\,y \mid x^6 = 1,\, x^3 = y^2,\, y^{-1}xy = x^{-1}} \\
       & C_{12} & \gen{x \mid x^{12} = 1} \\
       & A_4 & {\omitpres} \\ %\gen{x,\,y,\,z \mid x^3 = y^2 = z^2 = 1,\, yz = zy,\, x^{-1}yx = yz,\, x^{-1}zx = y} \\
       & D_{12} & \gen{x,\,y \mid x^6 = y^2 = 1,\, y^{-1}xy = x^{-1}} \\
       & C_6 \times C_2 & \gen{x,\,y \mid x^6 = y^2 = 1,\, xy = yx} \\
    \midrule
    13 & C_{13} & \gen{x \mid x^{13} = 1} \\
    \midrule
    14 & D_{14} & \gen{x,\,y \mid x^7 = y^2 = 1,\, y^{-1}xy = x^{-1}} \\
       & C_{14} & \gen{x \mid x^{14} = 1} \\
    \midrule
    15 & C_{15} & \gen{x \mid x^{15} = 1} \\
    \midrule
    16 & C_{16} & \gen{x \mid x^{16} = 1} \\
       & C_4 \times C_4 & \gen{x,\,y \mid x^4 = y^4,\, xy = yx} \\
       & (C_4 \times C_2) \rtimes C_2 & \gen{x,\,y,\,z \mid x^4 = y^2 = z^2 = 1,\ xy = yx,\ z^{-1}xz = xy,\ z^{-1}yz = y} \\
       & C_4 \rtimes C_4 & \gen{x,\,y \mid x^4 = y^4 = 1,\, x^{-1}yx = y^{-1}} \\
       & C_8 \times C_2 & \gen{x,\,y \mid x^8 = y^2 = 1,\, xy = yx} \\
       & \M_{16} & \gen{x,\,y \mid x^8 = y^2 = 1,\ y^{-1}xy = x^{5}} \\
       & D_{16} & \gen{x,\,y \mid x^8 = y^2 = 1,\, y^{-1}xy = x^{-1}} \\
       & \SD_{16} & \gen{x,\,y \mid x^8 = y^2 = 1,\, y^{-1}xy = x^3} \\
       & \Dic_{16} & \gen{x,\,y \mid x^8 = 1,\, x^4 = y^2,\, y^{-1}xy = x^{-1}} \\
       & C_4 \times C_2 \times C_2 & \gen{x,\,y,\,z \mid x^4 = y^2 = z^2 = 1,\, xy = yx,\, xz = zx,\, yz = zy} \\
       & D_8 \times C_2 & \gen{x,\,y,\,z \mid x^4 = y^2 = z^2 = 1,\, xy = yx,\, yz = zy,\, z^{-1}xz = x^{-1}} \\
       & Q_8 \times C_2 & \gen{x,\,y,\,z \mid x^4 = y^4 = z^2 = 1,\, xz = zx,\, yz = zy,\, y^{-1}xy = x^{-1}} \\
       & \text{Pauli Group} & \gen{x,\,y,\,z \mid x^4 = y^2 = z^2 = 1,\, xy = yx,\, z^{-1}xz = x^{-1},\, z^{-1}yz = x^2 y}\\
       & {(C_2)}^4 & {\omitpres} \\

       \bottomrule
\end{array}
        \end{displaymath}


        % \begin{tabular}{c|c}
        %     \toprule
        %     Order & Classification \\
        %     \midrule
        %     1 & \(\bm{1}\) \\
        %     \midrule
        %     2 & \(C_2\) \\
        %     \midrule
        %     3 & \(C_3\) \\
        %     \midrule
        %     4 & \(C_4\) \\
        %       & \(C_2 \times C_2\) \\
        %     \midrule
        %     5 & \(C_5\) \\
        %     \midrule
        %     6 & \(D_6 \cong S_3\) \\
        %       & \(C_6\) \\
        %     \midrule
        %     7 & \(C_7\) \\
        %     \midrule
        %     8 & \(C_8\) \\
        %       & \(C_4 \times C_2\) \\
        %       & \(D_8\) \\
        %       & \(Q_8\) \\
        %       & \(C_2 \times C_2 \times C_2\) \\
        %     \midrule
        %     9 & \(C_9\) \\
        %       & \(C_3 \times C_3\) \\
        %     \midrule
        %     10 & \(D_{10}\) \\
        %        & \(C_{10}\) \\
        %     \midrule
        %     11 & \(C_{11}\) \\
        %     \midrule
        %     12 & \(\Dic_{12}\) \\
        %        & \(C_{12}\) \\
        %        & \(A_4\) \\
        %        & \(D_{12}\) \\
        %        & \(C_6 \times C_2\) \\
        %     \midrule
        %     13 & \(C_{13}\) \\
        %     \midrule
        %     14 & \(D_{14}\) \\
        %        & \(C_{14}\) \\
        %     \midrule
        %     15 & \(C_{15}\) \\
        %     \midrule
        %     16 & \(C_{16}\) \\
        %        & \(C_4 \times C_4\) \\
        %        & \((C_4 \times C_2) \rtimes C_2\) \\
        %        & \(C_4 \rtimes C_4\) \\
        %        & \(C_8 \times C_2\) \\
        %        & \(\M_{16}\) \\
        %   \bottomrule
        % \end{tabular}
    \end{center}
\end{table}


\section{Introduction}
The study of groups is an important area of mathematics, and as such, it's quite useful to have quick examples of groups
`in your back pocket' so to speak; even more so, to have an exhaustive list of the possible behaviours of groups.
We do this by the notion of an isomorphism class, a kind of equivalence between groups which are essentially the same.
However as groups increase in size, this becomes increasingly hard to do, especially for so called `\(p\)-groups':
groups which have order the power of a prime.
For example, up to isomorphism, there are 51 possible groups of order 32 --- wow!

This report will cover the classification, and proof thereof, of groups of order up to, and including, 31.
In doing so, we will further our understanding of constructing groups, and where possible, find out the names given to
the groups we come across by the wider world of group theory.

To start, let's solidify the notation used in this report.
We shall denote groups and sets with capital letters, like \(G\), \(H\), and elements of those groups with lower case
letters, like \(g\), \(h\).
Greek letters shall denote mappings, generally \(\phi\), \(\psi\), etc.\ with \(\iota\) reserved for the identity map,
and we will write mappings on the right.

We will use \(\N\) to denote the natural numbers (not including 0), \(\Z\) to denote the integers, and \(\R\) to denote
the real numbers.

To denote the cyclic group of order \(n\) we will use \(C_n\), \(D_{2n}\) to denote the dihedral group of order \(2n\),
\(A_n\) to denote the alternating group over \(n\) elements, \(S_n\) to denote the symmetric group over \(n\)
elements, and \(Q_8\) to denote the quaternion group.
The trivial group, \(\{\, 1\, \}\) is denoted by \(\bm{1}\).
We will meet other groups as we go on our journey of classification!

