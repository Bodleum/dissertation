% ====   Order 2p   ====
\section{Groups of Order \(2p\)}
To illustrate an example of groups of order \(pq\), let's take \(q = 2\).
\begin{theorem}
    Any group of order \(2p\) is isomorphic to one of:
    \[C_{2p} \quad \text{or} \quad D_{2p}\]
\end{theorem}

\begin{proof}
Because every prime greater than 2 is odd, \(p - 1\) is an even number, and so \(2 \mid p - 1\).

An element \(a \in \units{p}\) of order 2 satisfies \(a^2 = 1\), hence \(a = 1\) or \(-1\).
But 1 has order 1, so \(a\) can only be \(-1\).
Side-note: from the proof of Lemma~\ref{lem:aut}, this corresponds to the inverse map.

So, in addition to \(C_{2p}\), we have:
\[G \cong \langle\, x, y \mid x^p = y^2 = 1,\ y^{-1}xy = x^{-1}\,\rangle\]
Which is the presentation for the dihedral group of order \(2p\), \(D_{2p}\).
\end{proof}
