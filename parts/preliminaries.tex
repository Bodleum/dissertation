% ====   Preliminaries   ====
\section{Preliminaries}
To start, let's solidify the notation used in this report.

We shall denote groups and sets with capital letters, like \(G\), \(H\), and elements of those groups with lower case
letters, like \(g\), \(h\).
Greek letters shall denote mappings, generally \(\phi\), \(\psi\), etc.\ with \(\iota\) reserved for the identity map,
and we will write mappings on the right.

We will use \(\N\) to denote the natural numbers (not including 0), \(\Z\) to denote the integers, and \(\R\) to denote
the real numbers.

To denote the cyclic group of order \(n\) we will use \(C_n\), \(D_{2n}\) to denote the cyclic group of order \(2n\),
\(A_n\) to denote the alternating group over \(n\) elements, \(S_n\) to denote the symmetric group over \(n\)
elements, and \(Q_8\) to denote the quaternion group.
The trivial group, \(\{\, 1\, \}\) is denoted by \(\bm{1}\).

Let's move on to review some facts and theorems which will be valuable later on, as well as introduce some new concepts
and prove some new results!


\begin{definition}
    If \(G\) and \(H\) are groups with elements \(g_1,\, g_2 \in G\), then a map:
    \[\phi:G \to H\]
    is a \emph{homomorphism} if:
    \[(g_1 g_2)\phi = (g_1\phi)(g_2\phi)\]
    If \(\phi\) is bijective, then we call it an \emph{isomorphism}, with \(G \cong H\) denoting that \(G\) is
    isomorphic to \(H\).
    And if \(\phi\) is an isomorphism from \(G\) to itself, then we call it an \emph{automorphism} of \(G\).
\end{definition}

\begin{lemma}
    The set of all automorphisms of a group \(G\) form a group under composition.
    Indeed, this is called the \emph{automorphism group} of \(G\), denoted \(\Aut{G}\).
\end{lemma}

\begin{proof}
    Let \(A = \Aut{G} = \{\,\phi:G \to G \mid \phi\ \text{is an isomorphism}\,\}\), and let \(\phi \in A\).
    Denote an element of \(G\) by \(g\).

    We know already that the composition of two isomorphisms is an isomorphism, so \(A\) is closed under composition.

    The identity map, \(\iota:g \mapsto g\), is certainly an automorphism of \(G\) and so \(A \neq \varnothing\).

    Indeed, \(\iota:g \mapsto g\) is the identity of \(A\), since:
    \[g\phi\iota = (g\phi)\iota = g\phi \quad \text{and} \quad g\iota\phi = (g\iota)\phi = g\phi\]

    And inverses clearly exists, because automorphisms are bijections, and bijections are invertible.
    Hence \(A = \Aut{G}\) is a group.
\end{proof}

\begin{lemma}\label{lem:aut}
    The automorphism group of \(C_n\) is isomorphic to the multiplicative group
    of integers mod \(n\).

    i.e. \(\Aut{C_n} \cong \units{n}\)
\end{lemma}

\begin{proof}
    Let \(C_n = \langle x \rangle\).
    Any automorphism, \(\varphi\) of \(C_n\) has the property:
    \[(x^i)\varphi = {(x\varphi)}^i\]
    Hence \(\varphi\) is determined by it's effect on a generator, \(x\), and preserves element
    order.
    In particular, \(\varphi\) sends generators to generators.
    So for \(\varphi\) to be an automorphism, it must send \(x\) to another generator, say \(x^k\).
    An element \(x^k\) generates \(C_n\) if \(x^k\) has order \(n\), i.e.\ when \(k\) and \(n\) are co-prime.
    Denote the automorphism sending \(x\) to \(x^k\) by \(\varphi_k\).

    Let's now investigate how these automorphisms behave.
    Let \(\varphi_k, \varphi_l \in \Aut{C_n}\), and consider:
    \[x\varphi_k\varphi_l = (x^k)\varphi_l = {(x^k)}^l = x^{(kl)} = x\varphi_{kl} \quad \mod{n}\]
    Because multiplication modulo \(n\) is commutative, \(x^{kl} = x^{lk}\), so \(\Aut{C_n}\) is abelian.

    Now consider \(\theta:\Aut{C_n} \to \units{n}\) defined by \(\varphi_k\theta = k\).
    We will show \(\theta\) is an isomorphism.
    Every \(k \in \units{n}\) is co-prime to \(n\) and so \(x^k\) is a generator of \(C_n\), hence there is some \(\varphi_k
    \in \Aut{C_n}\) such that \(\varphi_k\theta = k\).
    So \(\theta\) is surjective.
    If \(\varphi_k\theta = \varphi_l\theta\) then \(k = l\), so \(\theta\) is also injective.
    Finally, \(\theta\) is a homomorphism because:
    \[(\varphi_k\varphi_l)\theta = \varphi_{kl}\theta = kl = (\varphi_k\theta)(\varphi_l\theta)\]
    So \(\theta:\Aut{C_n} \to \units{n}\) is an isomorphism.
\end{proof}

\begin{lemma}\label{lem:char}
    Let \(G\) be a group with normal subgroup \(H\).
    And let \(K\) be the unique subgroup of its order in \(H\) (we call \(K\) \emph{characteristic} in \(H\)).
    Then \(K\) is a normal subgroup of \(G\).
\end{lemma}

\begin{proof}
    Consider the map \(\varphi_g:G \to G\) defined by \(\varphi_g:x \mapsto g^{-1}xg\) for elements \(x,\,g \in G\).
    We will show that this is an automorphism of \(G\).
    For \(x,\,y \in G\):
    \[x\varphi_g y\varphi_g = (g^{-1}xg)(g^{-1}yg) = g^{-1}(xy)g = (xy)\varphi_g\]
    Hence \(\varphi_g\) is a homomorphism.
    Moreover, \(\varphi_g\) is invertible with inverse \(\varphi_{g^{-1}}\).
    So \(\varphi_g\) is indeed an automorphism of G.

    Because \(H\) is normal, \(H\varphi_g = H\).
    So \(\varphi_g\) is an automorphism of \(H\) too.
    And so \(\varphi_g\) maps \(K\) to itself, because it is the unique subgroup of its order.
    Hence the set
    \[\{\,g^{-1}kg \mid k \in K\,\} = K\]
    So \(K\) is normal in \(G\).
\end{proof}

% ====   Semidirect Product   ====
\subsection{Semidirect Product}
We already know about the direct product:

\begin{definition}[Direct Product]
    For groups \(N\) and \(H\), the \emph{direct product}, \(G = N\times H\) is a group of ordered pairs of elements
    \((n, h)\) where \(n \in N\) and \(h \in H\) with the operation:
    \[(n_1, h_1)(n_2, h_2) = (n_1n_2, h_1h_2)\]

    Moreover, if \(\bar{N} = N \times \bm{1}\) and \(\bar{H} = \bm{1} \times H\), then:
    \begin{enumerate}[(i)]
        \item \(\bar{N} \nrmsgp G\) and \(\bar{H} \nrmsgp G\)
        \item \(\bar{N} \cap \bar{H} = \bm{1}\)
        \item \(\bar{N}\bar{H} = \{\,nh \mid n \in N,\ h \in H\,\} = G\)
    \end{enumerate}
\end{definition}

Now let's seek a slightly more general way to combine groups, by relaxing that \(H\) must be normal.
So we have:
\[N \nrmsgp G,\ H \leqslant G,\ NH = G, \quad \text{and} \quad N \cap H = \bm{1}\]
Consider the \emph{set}, (not the direct product):
\[N \times H = \{\,(n,\,h) \mid n \in N,\ h \in H\,\}\]
and a map
\[\phi:N \times H \to G \quad \text{defined by} \quad (n,\, h) \mapsto nh\]
We want \(\phi\) to be an isomorphism.

To show \(\phi\) is injective, take \(n_1,\, n_2 \in N\) and \(h_1,\, h_2 \in H\), and assume \(n_1 h_1 = n_2 h_2\).
Then multiplying on the left by \(n_2^{-1}\) and on the right by \(h_1^{-1}\) gives:
\[n_2^{-1} n_1 = h_2 h_1^{-1}\]
On the left we have an element of \(N\) and on the right, an element of \(H\), so \(n_2^{-1} n_1 = h_2 h_1^{-1} \in N
\cap H\).
But \(N \cap H = \bm{1}\) so then \(n_2^{-1} n_1 = h_2 h_1^{-1} = 1\).
Hence:
\[n_1 = n_2 \quad \text{and} \quad h_1 = h_2\]

To show \(\phi\) is surjective, consider the image, \(\im{\phi} = \{\,nh \mid n \in N,\ h \in H\,\}\).
This is by definition \(NH = G\), so \(\phi\) is surjective, and hence a bijection.

For \(\phi\) to be a homomorphism, we need:
\begin{equation*}
\begin{aligned}
    \relax[(n_1,\,h_1)(n_2,\,h_2)]\phi &= (n_1,\,h_1)\phi\,(n_2,\,h_2)\phi \\
    &= n_1 h_1 n_2 h_2 \\
    &= n_1 h_1 n_2 h_1^{-1} h_1 h_2 \\
    &= (n_1 h_1 n_2 h_1^{-1})(h_1 h_2)
\end{aligned}
\end{equation*}
But \(N\) is normal in \(G\) so \(h_1 n_2 h_1^{-1}\) is just another element in \(N\), say \(n_3\).
So:
\[\relax[(n_1,\,h_1)(n_2,\,h_2)]\phi = (n_1 n_3)(h_1 h_2) = (n_1 n_3,\,h_1 h_2)\phi\]
We know that \(\phi\) is injective, so then:
\[(n_1,\,h_1)(n_2,\,h_2) = (n_1 n_3,\,h_1 h_2)\]
This tells us the multiplication that will make \(NH\) a group.
Because \(N \nrmsgp G\), the map
\[\varphi_{h_1}:n_2 \mapsto h_1 n_2 h_1^{-1} = n_3\]
is an automorphism of \(N\).
This gives rise to the definition:

\begin{definition}[Semidirect Product]
\mbox{}
\begin{enumerate}[(i)]
    \item

        For a group \(G\) with normal subgroup \(N\) and subgroup \(H\) with \(NH = G\) and \(N \cap H = \bm{1}\),
        \(G\) is the \emph{internal semidirect product} of \(N\) by \(H\), written \(G = N \rtimes H\).

    \item

        For groups \(N\) and \(H\), and a homomorphism \(\psi:H \to \Aut{N}\), the \emph{external semidirect product} of
        \(N\) by \(H\) via \(\psi\) is the set:
        \[N \times H = \{\,(n,\,h) \mid n \in N,\ h \in H\,\}\]
        with multiplication:
        \[(n_1,\,h_1)(n_2,\,h_2) = (n_1(n_2^{h_1\psi}),\,h_1 h_2)\]
        denoted:
        \[N \rtimes_{\psi} H\]

        We use the notation \(n_2^{h_1\psi}\) both because it indicates conjugation, and is clearer.

\end{enumerate}
\end{definition}

\begin{lemma}\label{lem:setprodorder}
    For a group \(G\) with \(N \leqslant G\) and \(H \leqslant G\), with \(N \cap H = \bm{1}\) then:
    \[|NH| = |\{nh \mid n \in N, h \in H\}| = |N| \cdot |H|\]
\end{lemma}

\begin{proof}
    We just saw above that for elements \(n \in N\) and \(h \in H\), the map:
    \[\phi:N \times H \to NH \quad \text{defined by} \quad (n,\,h) \mapsto nh\]
    is a bijection.
    The result follows immediately from this.
\end{proof}

\begin{lemma}\label{lem:semiisom}
    Let \(N\) and \(H\) be groups, and \(\alpha \in \Aut{H}\).
    Then the semidirect products via the homomorphism \(\psi\), \(N \rtimes_\psi H\), and via the homomorphism
    \(\alpha\psi\), \(N \rtimes_{\alpha\psi} H\), are isomorphic.

    That is, we can apply any automorphism to \(H\) and the resulting semidirect product remains in the same isomorphism
    class.
\end{lemma}

\begin{proof}
    Let \(G = N \rtimes_\psi H\) and \(\bar{G} = N \rtimes_{\alpha\psi} H\), and define:
    \[\vartheta:G \to \bar{G} \quad \text{by} \quad \vartheta:(n,\,h) \mapsto (n,\,h\alpha)\]
    We will show that \(\vartheta\) is an isomorphism.

    First, \(\vartheta^{-1}\) exists because \(\alpha^{-1}\) exists, and is given by:
    \[\vartheta^{-1}:(n,\,h) \mapsto (n,\,h\alpha^{-1})\]
    Hence \(\vartheta\) is a bijection.

    Now for two elements, \((n_1,\,h_1),\ (n_2,\,h_2) \in G\), consider:
    \begin{equation*}
    \begin{aligned}
        (n_1,\,h_1)\vartheta\ (n_2,\,h_2)\vartheta &= (n_1,\,h_1\alpha)(n_2,\,h_2\alpha) \\
        &= (n_1 n_2^{(h_1\alpha)\psi},\,h_1\alpha h_2\alpha) \\
        &= (n_1 n_2^{h_1(\alpha\psi)},\, (h_1 h_2)\alpha) \\
        &= ((n_1,\,h_1)(n_2,\,h_2))\vartheta
    \end{aligned}
    \end{equation*}
    So \(\vartheta\) is an isomorphism.
\end{proof}

\subsection{Group Actions}
Some snazzy introduction.

\begin{definition}
    Let \(G\) be a group, and \(\Omega\) be a set, with elements \(g \in G\) and \(\omega \in \Omega\).
    Consider a map \(\mu:\Omega \times G \to \Omega\), and write \(\omega^g\) for the image of \((\omega,\,g)\) under
    \(\mu\).
    So we have:
    \[\mu:\Omega \times G \to \Omega \quad \text{defined by} \quad (\omega,\,g) \mapsto \omega^g\]
    We say \(G\) \emph{acts on} \(\Omega\) if for all \(g_1,\,g_2 \in G\) and all \(\omega \in \Omega\):
    \begin{enumerate}[(i)]
        \item \({(\omega^{g_1})}^{g_2} = \omega^{(g_1 g_2)}\)
        \item \(\omega^1 = \omega\)
    \end{enumerate}
    We call \(\mu\) the \emph{group action} of \(G\) on \(\Omega\).
\end{definition}

This might remind you of a homomorphism.
Indeed we have a result:

\begin{lemma}\label{lem:actionhom}
    A group action induces a homomorphism.
    Specifically, let \(G\) be a group which acts on a set \(\Omega\), with \(g \in G\) and \(\omega \in \Omega\), and
    define:
    \[\rho_g:\Omega \to \Omega \quad \text{by} \quad \omega \mapsto \omega^g\]
    Then:
    \[\rho:G \to \Sym{\Omega} \quad \text{defined by} \quad g \mapsto \rho_g\]
    is a homomorphism.
\end{lemma}

\begin{proof}
    Firstly, \(\rho_g\) is indeed a permutation of \(\Omega\) because it is invertible (and therefore a bijection),
    with:
    \[{(\rho_g)}^{-1} = \rho_{g^{-1}}\]
    Consider \(g,\,h \in G\) and their corresponding maps, \(\rho_{g},\,\rho_{h} \in \Sym{\Omega}\).
    Then:
    \[\omega(g\rho)(h\rho) = \omega\rho_g\rho_h = {(\omega^g)}^h = \omega^{(gh)} = \omega\rho_{gh} = \omega(gh)\rho\]
    Thus \(\rho\) is a homomorphism.
\end{proof}

A group acting on the set its cosets will be very useful:

\begin{definition}
    For a group \(G\) with \(H \leqslant G\), let \(\Omega = \{\,Hg \mid g \in G\,\}\), i.e.\ the set of cosets of \(H\)
    in \(G\).
    If \(x \in G\), define a group action:
    \[\Omega \times G \to \Omega \quad \text{by} \quad (Hg, x) \mapsto Hgx\]
\end{definition}

\begin{lemma}
    The action above is \emph{well defined}, meaning the action is independent of our choice of representative \(g\).
\end{lemma}

\begin{proof}
    % TODO
\end{proof}




