\documentclass[a4paper, oneside, 12pt, final]{article}

\title{Classification of Finite Groups}
\author{Daniel Laing}
\date{\today}

% ====   Preamble   ====
\usepackage[left=1.5cm, right=1.5cm, top=1.5cm, bottom=2cm]{geometry}
\usepackage{amsmath, amssymb, amsthm}
\usepackage{bm} % Bold maths
\usepackage[T1]{fontenc}
\usepackage{hyperref}
\usepackage{multicol}
\usepackage{tabularx}
\usepackage[super]{nth}
\usepackage{enumerate}
\usepackage{lipsum}
\usepackage{stmaryrd}   % for \lightning

\renewcommand{\emph}{\underline}

% Maths in section headings
\makeatletter
\pdfstringdefDisableCommands{\let\(\fake@math}
\newcommand\fake@math{}% just for safety
\def\fake@math#1\){\detokenize{#1}}
\makeatother

\newtheorem{theorem}{Theorem}[section]
\newtheorem{corollary}{Corollary}[theorem]
\newtheorem{lemma}[theorem]{Lemma}
\theoremstyle{definition}
\newtheorem{definition}[theorem]{Definition}

\DeclareMathOperator{\Aut}{Aut}
\DeclareMathOperator{\Sym}{Sym}
\DeclareMathOperator{\order}{o}
\DeclareMathOperator{\lcm}{lcm}
\DeclareMathOperator{\im}{im}
\DeclareMathOperator{\z}{Z}
\DeclareMathOperator{\C}{C}

\newcommand{\nrmsgp}{\unlhd}
\newcommand{\Z}{\mathbb{Z}}
\newcommand{\Zn}[1]{\Z/#1\Z}
\newcommand{\units}[1]{(\Zn{#1})^\times}
\newcommand{\F}{\mathbb{F}}
\newcommand{\R}{\mathbb{R}}
\newcommand{\N}{\mathbb{N}}


% ====   Document   ====
\begin{document}
{\maketitle}
{\tableofcontents}

\part{Doing}

\section{Introduction}

% ====   Theorems and Lemmas   ====
\section{Preliminaries}
To start, let's solidify the notation used in this report.

We shall denote groups and sets with capital letters, like \(G\), \(H\), and elements of those groups with lower case
letters, like \(g\), \(h\).
Greek letters shall denote mappings, generally \(\phi\), \(\psi\), etc.\ with \(\iota\) reserved for the identity map,
and we will write mappings on the right.

We will use \(\N\) to denote the natural numbers (not including 0), \(\Z\) to denote the integers, and \(\R\) to denote
the real numbers.

To denote the cyclic group of order \(n\) we will use \(C_n\), \(D_{2n}\) to denote the cyclic group of order \(2n\),
\(A_n\) to denote the alternating group over \(n\) elements, \(S_n\) to denote the symmetric group over \(n\)
elements, and \(Q_8\) to denote the quaternion group.
The trivial group, \(\{\, 1\, \}\) is denoted by \(\bm{1}\).

Let's move on to review some facts and theorems which will be valuable later on, as well as introduce some new concepts
and prove some new results!

\begin{definition}
    A \emph{permutation} of a set \(X\) is a bijection from \(X\) to \(X\).
    The \emph{symmetric group} \(X\) is the set of all permutations of \(X\) under composition.
    We write \(\Sym{X}\) to denote this.
    It is easy to show \(\Sym{X}\) is a group.
\end{definition}



\begin{definition}
    If \(G\) is a group, and \(H \subseteq G\), then \(H\) is a \emph{subgroup} of \(G\) if it is a group in its own right with
    the multiplication from \(G\).
    We write \(H \leqslant G\) to mean \(H\) is a subgroup of \(G\).

    If \(H\) is closed under \emph{conjugation}, i.e.\ for all \(g \in G\) and \(h \in H\), \(g^{-1}hg \in H\), then we
    say \(H\) is a \emph{normal subgroup} of \(G\).
    We write \(H \nrmsgp G\) to mean \(H\) is a normal subgroup of \(G\).
\end{definition}

\begin{definition}
    If \(G\) is a group and \(X \subseteq G\), then the \emph{subgroup generated by \(X\)} is the intersection of all
    subgroups of \(G\) containing \(X\).
    This in denoted \(\langle X \rangle\).
    The proof that \(\langle X \rangle\) is a subgroup of \(G\) is omitted.
    The elements of \(X\) are called \emph{generators} of \(G\).
\end{definition}

\begin{definition}
    If \(G\) is a group with subgroup \(H\) then the \emph{right coset} of \(H\) in \(G\) with representative \(g \in
    G\) is:
    \[Hg = \{\,hg \mid h \in H\,\}\]
\end{definition}


\begin{definition}
    The \emph{order} of a group, \(G\), is the number of elements in \(G\), denoted \(|G|\).
    The \emph{order} of an element \(g \in G\) is the smallest \(i \in \N\) such that \(g^i = 1\).
\end{definition}

\begin{definition}
    If \(G\) and \(H\) are groups with elements \(g_1,\, g_2 \in G\), then a map:
    \[\phi:G \to H\]
    is a \emph{homomorphism} if:
    \[(g_1 g_2)\phi = (g_1\phi)(g_2\phi)\]
    If \(\phi\) is bijective, then we call it an \emph{isomorphism}, with \(G \cong H\) denoting that \(G\) is
    isomorphic to \(H\).
    And if \(\phi\) is an isomorphism from \(G\) to itself, then we call it an \emph{automorphism} of \(G\).
\end{definition}

\begin{lemma}
    The set of all automorphisms of a group \(G\) form a group under composition.
    Indeed, this is called the \emph{automorphism group} of \(G\), denoted \(\Aut{G}\).
\end{lemma}

\begin{proof}
    Let \(A = \Aut{G} = \{\,\phi:G \to G \mid \phi\ \text{is an isomorphism}\,\}\), and let \(\phi \in A\).
    Denote an element of \(G\) by \(g\).

    We know already that the composition of two isomorphisms is an isomorphism, so \(A\) is closed under composition.

    The identity map, \(\iota:g \mapsto g\), is certainly an automorphism of \(G\) and so \(A \neq \varnothing\).

    Indeed, \(\iota:g \mapsto g\) is the identity of \(A\), since:
    \[g\phi\iota = (g\phi)\iota = g\phi \quad \text{and} \quad g\iota\phi = (g\iota)\phi = g\phi\]

    And inverses clearly exists, because automorphisms are bijections, and bijections are invertible.
    Hence \(A = \Aut{G}\) is a group.
\end{proof}

\begin{lemma}\label{lem:aut}
    The automorphism group of \(C_n\) is isomorphic to the multiplicative group
    of integers mod \(n\).

    i.e. \(\Aut{C_n} \cong \units{n}\)
\end{lemma}

\begin{proof}
    Let \(C_n = \langle x \rangle\).
    Any automorphism, \(\varphi\) of \(C_n\) has the property:
    \[(x^i)\varphi = {(x\varphi)}^i\]
    Hence \(\varphi\) is determined by it's effect on a generator, \(x\), and preserves element
    order.
    In particular, \(\varphi\) sends generators to generators.
    So for \(\varphi\) to be an automorphism, it must send \(x\) to another generator, say \(x^k\).
    An element \(x^k\) generates \(C_n\) if \(x^k\) has order \(n\), i.e.\ when \(k\) and \(n\) are co-prime.
    Denote the automorphism sending \(x\) to \(x^k\) by \(\varphi_k\).

    Let's now investigate how these automorphisms behave.
    Let \(\varphi_k, \varphi_l \in \Aut{C_n}\), and consider:
    \[x\varphi_k\varphi_l = (x^k)\varphi_l = {(x^k)}^l = x^{(kl)} = x\varphi_{kl} \quad \text{(mod \(n\))}\]
    Because multiplication modulo \(n\) is commutative, \(x^{kl} = x^{lk}\), so \(\Aut{C_n}\) is abelian.

    Now consider \(\theta:\Aut{C_n} \to \units{n}\) defined by \(\varphi_k\theta = k\).
    We will show \(\theta\) is an isomorphism.
    Every \(k \in \units{n}\) is co-prime to \(n\) and so \(x^k\) is a generator of \(C_n\), hence there is some \(\varphi_k
    \in \Aut{C_n}\) such that \(\varphi_k\theta = k\).
    So \(\theta\) is surjective.
    If \(\varphi_k\theta = \varphi_l\theta\) then \(k = l\), so \(\theta\) is also injective.
    Finally, \(\theta\) is a homomorphism because:
    \[(\varphi_k\varphi_l)\theta = \varphi_{kl}\theta = kl = (\varphi_k\theta)(\varphi_l\theta)\]
    So \(\theta:\Aut{C_n} \to \units{n}\) is an isomorphism.
\end{proof}

\begin{theorem}[Lagrange's Theorem]

\end{theorem}

\subsection{Sylow Theorems}
This collection of theorems is extremely useful for describing subgroup structure.
Hopefully these ring some bells.
Let \(G\) be a group of order \(p^n m\) where \(p\) is a prime and \(p\nmid m\).
\begin{theorem}[\nth{1} Sylow Theorem]\label{thm:sylow1}
    \(G\) has a Sylow \(p\)-subgroup, i.e.\ a subgroup of order \(p^n\).
\end{theorem}
\begin{theorem}[\nth{2} Sylow Theorem]\label{thm:sylow2}
    All Sylow \(p\)-subgroups of \(G\) are conjugate to each other.
    In particular, if \(G\) has a unique Sylow \(p\)-subgroup, then it is a normal subgroup.
\end{theorem}
\begin{theorem}[\nth{3} Sylow Theorem]\label{thm:sylow3}
    Let \(n_p\) denote the number of Sylow \(p\)-subgroups of \(G\).
    Then:
    \begin{enumerate}[(i)]
        \item \(n_p \mid m\)
        \item \(n_p\equiv 1\) (mod \(p\))
    \end{enumerate}
\end{theorem}

% TODO: Write out isom theorems
\subsection{Isomorphism Theorems}
\begin{theorem}\label{thm:iso1}
\end{theorem}

\begin{theorem}\label{thm:iso2}
\end{theorem}

\begin{theorem}\label{thm:iso3}
\end{theorem}

\subsection{Semidirect Product}
We already know about the direct product:

\begin{definition}[Direct Product]
    For groups \(N\) and \(H\), the \emph{direct product}, \(G = N\times H\) is a group of ordered pairs of elements
    \((n, h)\) where \(n \in N\) and \(h \in H\) with the operation:
    \[(n_1, h_1)(n_2, h_2) = (n_1n_2, h_1h_2)\]

    Moreover, if \(\bar{N} = N \times \bm{1}\) and \(\bar{H} = \bm{1} \times H\), then:
    \begin{enumerate}[(i)]
        \item \(\bar{N} \nrmsgp G\) and \(\bar{H} \nrmsgp G\)
        \item \(\bar{N} \cap \bar{H} = \bm{1}\)
        \item \(\bar{N}\bar{H} = \{\,nh \mid n \in N,\ h \in H\,\} = G\)
    \end{enumerate}
\end{definition}

Now let's seek a slightly more general way to combine groups, by relaxing that \(H\) must be normal.
So we have:
\[N \nrmsgp G,\ H \leqslant G,\ NH = G, \quad \text{and} \quad N \cap H = \bm{1}\]
Consider the \emph{set}, (not the direct product):
\[N \times H = \{\,(n,\,h) \mid n \in N,\ h \in H\,\}\]
and a map
\[\phi:N \times H \to G \quad \text{defined by} \quad (n,\, h) \mapsto nh\]
We want \(\phi\) to be an isomorphism.

To show \(\phi\) is injective, take \(n_1,\, n_2 \in N\) and \(h_1,\, h_2 \in H\), and assume \(n_1 h_1 = n_2 h_2\).
Then multiplying on the left by \(n_2^{-1}\) and on the right by \(h_1^{-1}\) gives:
\[n_2^{-1} n_1 = h_2 h_1^{-1}\]
On the left we have an element of \(N\) and on the right, an element of \(H\), so \(n_2^{-1} n_1 = h_2 h_1^{-1} \in N
\cap H\).
But \(N \cap H = \bm{1}\) so then \(n_2^{-1} n_1 = h_2 h_1^{-1} = 1\).
Hence:
\[n_1 = n_2 \quad \text{and} \quad h_1 = h_2\]

To show \(\phi\) is surjective, consider the image, \(\im{\phi} = \{\,nh \mid n \in N,\ h \in H\,\}\).
This is by definition \(NH = G\), so \(\phi\) is surjective, and hence a bijection.

For \(\phi\) to be a homomorphism, we need:
\begin{equation*}
\begin{aligned}
    \relax[(n_1,\,h_1)(n_2,\,h_2)]\phi &= (n_1,\,h_1)\phi\,(n_2,\,h_2)\phi \\
    &= n_1 h_1 n_2 h_2 \\
    &= n_1 h_1 n_2 h_1^{-1} h_1 h_2 \\
    &= (n_1 h_1 n_2 h_1^{-1})(h_1 h_2)
\end{aligned}
\end{equation*}
But \(N\) is normal in \(G\) so \(h_1 n_2 h_1^{-1}\) is just another element in \(N\), say \(n_3\).
So:
\[\relax[(n_1,\,h_1)(n_2,\,h_2)]\phi = (n_1 n_3)(h_1 h_2) = (n_1 n_3,\,h_1 h_2)\phi\]
We know that \(\phi\) is injective, so then:
\[(n_1,\,h_1)(n_2,\,h_2) = (n_1 n_3,\,h_1 h_2)\]
This tells us the multiplication that will make \(NH\) a group.
Because \(N \nrmsgp G\), the map
\[\varphi_{h_1}:n_2 \mapsto h_1 n_2 h_1^{-1} = n_3\]
is an automorphism of \(N\).
This gives rise to the definition:

\begin{definition}[Semidirect Product]
\mbox{}
\begin{enumerate}[(i)]
    \item

        For a group \(G\) with normal subgroup \(N\) and subgroup \(H\) with \(NH = G\) and \(N \cap H = \bm{1}\),
        \(G\) is the \emph{internal semidirect product} of \(N\) by \(H\), written \(G = N \rtimes H\).

    \item

        For groups \(N\) and \(H\), and a homomorphism \(\psi:H \to \Aut{N}\), the \emph{external semidirect product} of
        \(N\) by \(H\) via \(\psi\) is the set:
        \[N \times H = \{\,(n,\,h) \mid n \in N,\ h \in H\,\}\]
        with multiplication:
        \[(n_1,\,h_1)(n_2,\,h_2) = (n_1(n_2\phi_{h_1}),\,h_1 h_2)\]
        denoted:
        \[N \rtimes_{\psi} H\]

\end{enumerate}
\end{definition}

\begin{lemma}\label{lem:setprodorder}
    For a group \(G\) with \(N \leqslant G\) and \(H \leqslant G\), with \(N \cap H = \bm{1}\) then:
    \[|NH| = |\{nh \mid n \in N, h \in H\}| = |N| \cdot |H|\]
\end{lemma}

\begin{proof}
    We just saw above that for elements \(n \in N\) and \(h \in H\), the map:
    \[\phi:N \times H \to NH \quad \text{defined by} \quad (n,\,h) \mapsto nh\]
    is a bijection.
    The result follows immediately from this.
\end{proof}

\subsection{Group Actions}
Some snazzy introduction.

\begin{definition}
    Let \(G\) be a group, and \(\Omega\) be a set, with elements \(g \in G\) and \(\omega \in \Omega\).
    Consider a map \(\mu:\Omega \times G \to \Omega\), and write \(\omega^g\) for the image of \((\omega,\,g)\) under
    \(\mu\).
    So we have:
    \[\mu:\Omega \times G \to \Omega \quad \text{defined by} \quad (\omega,\,g) \mapsto \omega^g\]
    We say \(G\) \emph{acts on} \(\Omega\) if for all \(g_1,\,g_2 \in G\) and all \(\omega \in \Omega\):
    \begin{enumerate}[(i)]
        \item \({(\omega^{g_1})}^{g_2} = \omega^{(g_1 g_2)}\)
        \item \(\omega^1 = \omega\)
    \end{enumerate}
    We call \(\mu\) the \emph{group action} of \(G\) on \(\Omega\).
\end{definition}

This might remind you of a homomorphism.
Indeed we have a result:

\begin{lemma}\label{lem:actionhom}
    A group action induces a homomorphism.
    Specifically, let \(G\) be a group which acts on a set \(\Omega\), with \(g \in G\) and \(\omega \in \Omega\), and
    define:
    \[\rho_g:\Omega \to \Omega \quad \text{by} \quad \omega \mapsto \omega^g\]
    Then:
    \[\rho:G \to \Sym{\Omega} \quad \text{defined by} \quad g \mapsto \rho_g\]
    is a homomorphism.
\end{lemma}

\begin{proof}
    Firstly, \(\rho_g\) is indeed a permutation of \(\Omega\) because it is invertible (and therefore a bijection),
    with:
    \[{(\rho_g)}^{-1} = \rho_{g^{-1}}\]
    Consider \(g,\,h \in G\) and their corresponding maps, \(\rho_{g},\,\rho_{h} \in \Sym{\Omega}\).
    Then:
    \[\omega(g\rho)(h\rho) = \omega\rho_g\rho_h = {(\omega^g)}^h = \omega^{(gh)} = \omega\rho_{gh} = \omega(gh)\rho\]
    Thus \(\rho\) is a homomorphism.
\end{proof}

A group acting on the set its cosets will be very useful:

\begin{definition}
    For a group \(G\) with \(H \leqslant G\), let \(\Omega = \{\,Hg \mid g \in G\,\}\), i.e.\ the set of cosets of \(H\)
    in \(G\).
    If \(x \in G\), define a group action:
    \[\Omega \times G \to \Omega \quad \text{by} \quad (Hg, x) \mapsto Hgx\]
\end{definition}

\begin{lemma}
    The action above is \emph{well defined}, meaning the action is independent of our choice of representative \(g\).
\end{lemma}

\begin{proof}
    % TODO
\end{proof}


% ====   Trivial and Prime   ====
\section{First Classifications}
Let's start with the easiest case: groups of order 1.
Any group \(G\) must have an identity element, and so that's all our possible elements used up!
All groups of order 1 are isomorphic to the trivial group, \(\bm{1}\).

What about groups of prime order?
Let \(G\) be a group of order \(p\), where \(p\) is a prime number.
Then Lagrange's Theorem tell us all elements must have order 1 or \(p\).
Pick some \(x \in G\) with \(x\) having order \(p\).
Then \(\langle x \rangle = G\) so G is cyclic and \(G \cong C_p\).

% ====   Groups of order 6   ====
\section{Groups of Order 6}
Let \(G\) be a group of order 6, and \(n_3\) denote the number of Sylow 3-subgroups of \(G\).
Then by Theorem~\ref{thm:sylow3}:
\[n_3 \equiv 1 \ \text{(mod 3)} \quad \text{and} \quad n_3 \mid 2 \implies n_3 = 1\]
So \(G\) has one Sylow 3-subgroup, \(N\), and because 3 is prime, it is isomorphic to \(C_3\).
Let \(N = \langle x \rangle\).
Any Sylow 2-subgroup, \(H \leqslant G\), will have order 2, and so \(H \cong C_2\).
Let \(H = \langle y \rangle\).
Lagrange's Theorem tells us that \(N\) has elements of orders 1 and 3, and \(H\) has elements of
order 1 and 2 hence:
\[N \cap H = \bm{1}\]

By Lemma~\ref{lem:setprodorder}:
\[|NH| = |N| \cdot |H| = 6\]
So \(G = NH\), \(N \nrmsgp G\) and \(N \cap H = \bm{1}\), which means \(G = N \rtimes H\), the
semidirect product of \(N\) by \(H\).

Now we need to determine \(\Aut{N}\).
An automorphism, \(\varphi\) of \(N\) preserves element order.
In particular, \(\varphi\) maps generators to generators.
Hence, \(x\varphi = x\) or \(x^2\) because they are the generators of \(N\).
So \(\Aut{N} \cong C_2\).

Now we want a homomorphism \(\psi:H \to \Aut{N}\).
If \(\psi\) is trivial, then it maps \(H\) to the trivial group, so every element of \(H\) gets sent
to the trivial automorphism.
If \(\psi\) is not trivial, then at least one element of \(H\) is not sent to the trivial
automorphism.
It cannot be 1 because then element order is not preserved, so it must be the generator, \(y\).
Hence we obtain 2 possibilities for \(G\):

\begin{enumerate}[\bfseries{Case} 1:]
    \item
        \begin{equation*}
        \begin{aligned}
            G &= \langle\, x, y \mid x^3 = y^2 = 1,\ y^{-1}xy = x \,\rangle \\
            &=\langle\, x, y \mid x^3 = y^2 = 1,\ xy = yx \,\rangle \\
            &= C_3 \times C_2 \cong C_6
        \end{aligned}
        \end{equation*}

    \item
        \begin{equation*}
        \begin{aligned}
            G &= \langle\, x, y \mid x^3 = y^2 = 1,\ y^{-1}xy = x^{-1}
            \,\rangle \\
            &\cong D_6
        \end{aligned}
        \end{equation*}
\end{enumerate}

These are clearly not isomorphic, because \(C_6\) is abelian, and \(D_6\) is not.

Hence \(G\) is isomorphic one of:
\[C_6 \quad \text{or} \quad D_6\]

% ====   Groups of order 2p   ====
\section{Generalisation to Groups of Order \(2p\)}
Now that we have seen groups of order 6, let's try and work towards a more general case: groups of
order 2 times a prime number.
So let \(G\) be a group of order \(2p\) where \(p\) is a prime number, and \(n_p\) denote the number
of Sylow p-subgroups of \(G\).
Then by Theorem~\ref{thm:sylow3}:
\[n_p \equiv 1 \ \text{(mod \(p\))} \quad \text{and} \quad n_p \mid 2 \implies n_p = 1\]
So \(G\) has one Sylow \(p\)-subgroup, say \(N\), and it is isomorphic to \(C_p\).
Let \(N = \langle x \rangle\).
A Sylow 2-subgroup, \(H \leqslant G\) will have order 2 so \(H \cong C_2\).
Let \(H = \langle y \rangle\).
Lagrange's Theorem tells us that \(N\) has elements of orders 1 and \(p\), and \(H\) has elements of
order 1 and 2 hence:
\[N \cap H = \bm{1}\]

By Lemma~\ref{lem:setprodorder}:
\[|NH| = |N| \cdot |H| = 2p\]
So \(G = N \rtimes H\) as before.

We know by Lemma~\ref{lem:aut} that \(\Aut{N} \cong \units{p}\), so let's look for the elements of
order 2.
An element \(x \in \units{p}\) of order 2 satisfies \(x^2 = 1\), hence \(x = 1\) or \(-1\).
But 1 has order 1, so \(x\) can only be \(-1\).
From the proof of Lemma~\ref{lem:aut}, this corresponds to the inverse map \(\beta:x \mapsto
x^{-1}\).

Now we want a homomorphism \(\psi: H \to \Aut{N}\).
By the same argument as for groups of order 6, we have two possibilities for \(G\):

\begin{enumerate}[\bfseries{Case} 1:]
    \item
        \begin{equation*}
        \begin{aligned}
            G &= \langle\, x, y \mid x^p = y^2 = 1,\ y^{-1}xy = x \,\rangle \\
            &= C_p \times C_2 \cong C_{2p}
        \end{aligned}
        \end{equation*}
    \item
        \begin{equation*}
        \begin{aligned}
            G &= \langle\, x, y \mid x^p = y^2 = 1,\ y^{-1}xy = x^{-1}
            \,\rangle \\
            &\cong D_{2p}
        \end{aligned}
        \end{equation*}
\end{enumerate}

Again, these are clearly not isomorphic, because \(C_{2p}\) is abelian, and \(D_{2p}\) is not.

Hence a group of order \(2p\) is isomorphic to one of:
\[C_{2p} \quad \text{or} \quad D_{2p}\]

% ====   Groups of order pq   ====
\section{Groups of Order \(pq\)}
Let \(G\) be a group of order \(pq\) where \(p\), \(q\) are prime numbers with \(p > q\), and let \(n_p\) and \(n_q\)
denote the number of Sylow p-subgroups and Sylow q-subgroups of \(G\) respectively.
Then by Theorem~\ref{thm:sylow3}:
\[n_p \equiv 1 \ \text{(mod \(p\))} \quad \text{and} \quad n_p \mid q \implies n_p = 1\]
\[n_q \equiv 1 \ \text{(mod \(q\))} \implies n_q = 1,\ q+1,\ 2q+1,\ \dots \quad \text{and} \quad n_q \mid p\]

So \(G\) has a unique Sylow \(p\)-subgroup, say \(P \nrmsgp G\), and a Sylow \(q\)-subgroup, \(Q \leqslant G\).
Because \(p\) and \(q\) are prime numbers, \(P \cong C_p\) and \(Q \cong C_q\).
Pick generators for each, say \(\rangle x \langle  = P\) and \(\rangle y \langle = Q\).
We have 2 possibilities for \(n_q\): \(p-1\) is a multiple of \(q\) or 1.

\begin{enumerate}[\bfseries{Case} 1:]
    \item \(q \nmid p - 1\).

        If \(p-1\) \emph{is not} a multiple of \(q\) then \(n_q = 1\) and \(Q \nrmsgp G\), hence:
        \[G = P \times Q \cong C_{pq}\]

    \item \( q \mid p - 1\).

        If \(p-1\) \emph{is} a multiple of \(q\) then \(n_q = p\) and so \(Q\) is \emph{not} normal in \(G\).
        By Lagrange's Theorem, \(P \cap Q = \bm{1}\) and by Lemma~\ref{lem:setprodorder}, \(|PQ| = pq\).
        Hence, as well as the direct product, we have \(G = P \rtimes Q\), some non-trivial semidirect product.

        By Lemma~\ref{lem:aut}, \(\Aut{C_p} \cong \units{p} \cong C_{p-1}\).
        So if \(\nu \in \units{p}\), then \(x \mapsto x^\nu\) is an automorphism.
        We know also that \(C_{p-1}\) has a unique subgroup of order \(q\),
        hence \(G\) has the presentation:

        \[G = \langle \, x, y, \mid x^p = y^q = 1,\ y^{-1}xy = x^a \, \rangle\]
        where \(a\) is a generator for the subgroup of order \(q\) in
        \(\units{p}\).

        Notice that picking different generators are equivalent up to isomorphism because the composition of two
        isomorphisms is an isomorphism.
\end{enumerate}

So any group of order \(pq\) is isomorphic to either:
\begin{equation*}
\begin{aligned}
    C_{pq} \quad \text{or} \quad \langle \, x, y \mid x^p = y^q = 1,\ y^{-1}xy
    = x^a \, \rangle \qquad &\text{if} \ q \mid p-1 \\
    C_{pq} \qquad &\text{if} \ q \nmid p-1
\end{aligned}
\end{equation*}


% ====   Groups of order 4   ====
\section{Groups of order 4}
The Sylow theorems are not so helpful here, because any Sylow 2-subgroup will be of order 4, which
is just G.
Lagrange's Theorem tells us every element of \(G\) has order 1, 2 or 4.

If \(x \in G\) has order 4, then \(x\) generates G so \(G \cong C_4\).

If instead there is no element of order 4 in \(G\), then every \(x \in G\) except the identity is of
order 2.
Consider \(a, b \in G\) with \(a \neq b\), and their product, \(ab\).
It must be that \(ab\) is the third element of order 2, otherwise we reach a contradiction.
So it is easy to see that \(G \cong C_2 \times C_2\).

So any group of order 4 is isomorphic to one of:
\[C_4 \quad \text{or} \quad C_2 \times C_2\]

% ====   Groups of order p^2   ====
\section{Generalisation to Groups of Order \(p^2\)}
Let \(G\) be a group of order \(p^2\).
First, we will prove a useful lemma:

\begin{lemma}
    If \(G\) is a \(p\)-group (i.e.\ a group of prime power order), then every subgroup of index \(p\) is normal.
\end{lemma}

\begin{proof}
    Let \(H\) be a subgroup of G, with index \(p\), and let \(\Omega\) be the set of all cosets of \(H\).
    So by definition, \(|\Omega| = p\).
    By Lemma~\ref{lem:actionhom}, there is a homomorphism:
    \[\rho:G \to S_p\]
    If we have \(x \in \ker{\rho}\), then:
    \[(H1)x = H1 = H\]
    which means \(x \in H\).
    So the kernel of \(\rho\) is just \(H\), which means the quotient \(\frac{G}{\ker{\rho}}\) is of order \(p\).
    By Theorem~\ref{thm:iso1}, this means the image of \(\rho\) is a subgroup of \(S_p\) with order \(p\).
    Because kernels are normal subgroups, \(H \nrmsgp G\).  % TODO: Add this as a fact
\end{proof}


By Lagrange's Theorem, the elements of \(G\) have order 1, \(p\) or \(p^2\).

If \(x \in G\) has order \(p^2\), then \(x\) generates \(G\) so \(G \cong C_{p^2}\).

If \(G\) does not have an element of order \(p^2\) then all elements, except the identity, have order \(p\).
We know that \(G\) must have a subgroup of order \(p\), \(P\), and because \(p\) is prime, \(P \cong C_p\).
Pick a generator for \(P\), say \(x\) and an element \(y \in G\) such that \(y \notin P\).
Then \(y \neq x^i\) for any \(i\).

If \(y^j = x^i\) for some \(i\) and \(j\), then:
\[{(y^j)}^{1-j} = {(x^i)}^{1-j} = y^{j-j+1} = y = x^{i(1-j)} = x^k \quad \text{for some} \ k\text{,} \ \text{a
contradiction.}\]
So no power of \(y\) is equal to any power of \(x\).
Because \(y\) has order \(p\), it generates a subgroup of order \(p\), \(\bar{P}\) with \(P \cap \bar{P} = \bm{1}\).
The lemma tells us that both \(P\) and \(\bar{P}\) are normal, and by Lemma~\ref{lem:setprodorder}, \(|P\bar{P}| = p^2 =
|G|\) so:
\[G = P \times \bar{P} \cong C_p \times C_p\]

If \(G\) has no elements of order \(p\) or \(p^2\), then it only has elements of order 1, which is the trivial group.

Hence any group of order \(p^2\) is isomorphic to one of:
\[C_{p^2} \quad \text{or} \quad C_p \times C_p\]


% TODO: Check the above proof is actually legit

% Let \(G\) be a group of order \(p^2\) and consider \(\z{(G)} \nrmsgp G\).
% By Lagrange's Theorem, \(\z{(G)}\) has order 1, \(p\) or \(p^2\).
%
% If \(|\z{(G)}| = p^2\) then \(G\) is abelian.
%
% Assume \(|\z{(G)}| \neq p^2\) and consider an element \(x \in G\) but \(x
% \notin \z{(G)}\), and it's centraliser, \(\C_G{(x)}\).
% We know \(\C_G{(x)} \leqslant G\) and that \(x \in \C_G{(x)}\), so \(|\C_G{(x)|
% \neq 1}\), and so by Lagrange's Theorem, it must be that \(|\C_G{(x)}| = p\).
% So:
% \[|x^G| = |G:\C_G{(x)}| = \frac{p^2}{p} = p\]
%
% The Class Equation, \(|G| = |\z{(G)}| + \sum^k_{i=1}|x_i^G|\), tells us
% \(|\z{(G)}|\) must be a multiple of \(p\) because both  \(|G|\) and \(|x^G|\)
% are multiples of \(p\). Hence \(|\z{(G)}| = p\).
%
% So then \(|G:\z{(G)}| = p\), which means \(G/\z{(G)} \cong C_p\).
%
% % TODO: Finish proof
% % TODO: Tidy it up
%
% \vskip 3em
% \paragraph{Sketch}
% \begin{itemize}
%     \item Show \(G\) must be abelian. Result follows from FTFAB\@.
%     \item \(\z{(G)}\) has order 1, \(p\), or \(p^2\) by Lagrange.
%     \item If \(p^2\) then done.
%     \item Size of congruence classes is multiple of p.
%     \item Class eqn => order of centre is multiple of p. (and so is not 1)
%     \item Quotient with G is cyclic.
%     \item MT4003 showed then G must be abelian.
% \end{itemize}


% ====   Groups of order 12   ====
\section{Groups of order 12}
Let \(G\) be a group of order \(12 = 2^2 \cdot 3\), and \(n_3\) and \(n_2\) denote the number of Sylow 3-subgroups and
Sylow 2-subgroups of G respectively.
By Theorem~\ref{thm:sylow3}:
\[n_2 \equiv 1 \text{ (mod 2) and } n_2 \mid 3 \implies n_2 = 1\]
\[n_3 \equiv 1 \text{ (mod 3) and } n_3 \mid 4 \implies n_3 = 1 \ \text{or} \ 4\]

So \(G\) has a unique Sylow 2-subgroup of order 4, say \(H \nrmsgp G\), and we have already classified groups of order 4,
so \(H\) is isomorphic to either \(V_4\) (the Klein 4 group) or \(C_4\).
A Sylow 3-subgroup, \(K \leqslant G\) will have order 3, so \(K \cong C_3\).
Say \(K = \langle x \rangle\).

Lagrange's Theorem tells us \(H\) has elements of order 1, 2, and 4, and \(K\) has elements of order 1 and 3.
Hence \(H \cap K = \bm{1}\).
Lemma~\ref{lem:setprodorder} tells us:
\[|HK| = |H| \cdot |K| = 12\]
Hence \(G = HK\), \(H \nrmsgp G\), and \(H \cap K = \bm{1}\).
If we consider groups with 4 Sylow 3-subgroups then we can conclude that they are some semidirect product, \(G = H
\rtimes K\).

Since an automorphism, \(\varphi\), must map generators to generators, \(\Aut{C_4} \cong C_2\) because \(C_4\) has two
generators.
An automorphism of \(V_4\) corresponds to a permutation of the three
non-identity elements, hence \(\Aut{V_4} \cong S_3\).

\begin{enumerate}[\bfseries{Case} 1:]
    \item \(H \cong C_4\) i.e. \(G \cong C_4 \rtimes C_3\).

        Let \(H = \langle y \rangle\).

        A homomorphism \(\psi:K \to \Aut{H} \cong C_2\), preserves order and together with Lagrange's Theorem means that
        the only possibility for \(\psi\) is trivial, i.e. \(K\psi = \bm{1}\).

        Hence \(G \cong C_4 \times C_3 \cong C_{12}\).

    \item \(H \cong V_4\) i.e. \(G \cong (C_2 \times C_2) \rtimes C_3\).

        Let \(H = \langle y, z \rangle\).

        A trivial homomorphism \(K\psi = \bm{1}\) yields the direct product.
        What non-trivial homomorphisms are there?
        The automorphism group, \(\Aut{H} \cong S_3\) is of order 6, and so has a unique subgroup of order 3, by
        Theorem~\ref{thm:sylow3}.
        We know already that a homomorphism \(\psi: K \to \Aut{H}\) is determined by where it sends the generator
        \(x\), so for \(\psi\) to be non-trivial, it must send \(x\) to an element of order 3 in \(\Aut{H}\).

        There are 2 such elements, and we will think of them as the permutations of order 3 of the set \(\{1, 2, 3\}\).
        Denote them \(a = (1\ 2\ 3)\) and \(b = (1\ 3\ 2)\).
        Notice that \(b = a^{-1}\), so we have homomorphisms:
        \[\psi_1:x \mapsto a \quad \text{and} \quad \psi_2:x \mapsto a^{-1}\]
        It appears we have 2 choices, but this is not the case.
        If we define \(\theta:K \to K\) by \(x\theta = x^{-1}\) then \(\theta\psi_1 = \psi_2\).
        And notice that \(\theta\) is an automorphism of \(K\), so the semidirect products with \(\psi_1\) and \(\psi_2\)
        are isomorphic.
        Hence (up to isomorphism) there is one non-trivial homomorphism \(\psi:K \to \Aut{H}\).
        So the \(x\) acts by permuting the 3 non-identity elements of \(H\).

        We will show that in this case, \(G \cong A_4\).
        First, let's check \(A_4\) has the same subgroup structure as \(G\).
        There is a subgroup isomorphic to \(C_3\) in \(A_4\), generated by the 3-cycle \((1\ 2\ 3)\):
        \[\bar{K} = \langle (1\ 2\ 3) \rangle\]
        We can also find a subgroup isomorphic to \(V_4\):
        \[\bar{H} = \{\,1,\, (1\ 2)(3\ 4),\, (1\ 3)(2\ 4),\, (1\ 4)(2\ 3)\,\}\]
        Indeed, \(\bar{H}\) is normal in \(A_4\).
        We can see that \(\bar{H} \cap \bar{K} = \bm{1}\) because \(\bar{H}\) contains no 3-cycles, and that
        \(\bar{H}\bar{K} = A_4\).
        So we can conclude that \(A_4 = \bar{H} \rtimes \bar{K}\).

        Let's investigate haw
        If we let \(\alpha = (1\ 2)(3\ 4)\), \(\beta = (1\ 4)(2\ 3)\) and \(\gamma = (1\ 2\ 3)\), then we can write an
        element of \(A_4\) as \(\alpha^i\beta^j\gamme^k\) for some \(i\), \(j\) and \(k\).
        Define \(\phi:A_4 \to G\) by \(\phi:\alpha^i\beta^j\gamma^k \mapsto x^{i}y^{j}z^{k}\).
        Then:
        \[\beta\phi = (\gamma^{-1}\alpha\gamma)\phi = c^{-1}ac = b\]
        So conjugation acts in the same way.
        Hence we can conclude that \(G \cong A_4\).
\end{enumerate}

If we instead consider \(G\) where \(K \nrmsgp G\), i.e. \(G = K \rtimes H\), then we again have two cases:

\begin{enumerate}[\bfseries{Case} 1:]
    \item \(H \cong C_4\) i.e. \(G \cong C_3 \rtimes C_4\).

        Let \(H = \langle y \rangle\).

        We know \(\Aut{C_3} \cong C_2\) so a homomorphism \(\psi\) maps \(H\) to the trivial group or to \(\langle
        \beta:x \mapsto x^{-1} \rangle\).

        If \(H\psi = \bm{1}\) then \(G = K \times H \cong C_4 \times C_3\), which we have already seen.

        If \(H\psi = \langle \beta \rangle\) then we have:
        \[G = \langle\, x, y \mid x^3 = y^4 = 1,\ y^{-1}xy = x^{-1}\,\rangle\]
        % TODO: Classify this

    \item \(H \cong V_4\) i.e. \(G \cong C_3 \rtimes (C_2 \times C_2)\).

        Let \(H = \langle y, z \rangle\).

        If \(\psi:H \to \Aut{K}\) is trivial then we obtain the direct product again.

        The image of a non-trivial homomorphism \(\psi:H \to \Aut{K}\) is isomorphic to \(C_2\), so by
        Theorem~\ref{thm:iso1}: \(\ker{\psi} \cong C_2\).

        We can choose \(\psi\) such that \(y\psi = \beta:x \mapsto x^{-1}\) and \(z\psi = \iota:x \mapsto x\).
        Then:
        \[G = \langle\, x, y, z \mid x^3 = y^2 = z^2 = 1,\ yz = zy,\ y^{-1}xy = x^{-1},\ z^{-1}xz = x\,\rangle\]
        Let \(a = xz\).
        The order of \(a = \lcm{(\order{(x)}, \order{(z)})} = \lcm{(2, 3)} = 6\) because \(x\) and \(z\) commute.
        So:
        \[a^3 = x^3z^3 = z\]
        and
        \[ y^{-1}ay = y^{-1}xzy = y^{-1}xyz = x^{-1}z = x^2z = a^2a^3 = a^{-1}\]
        Hence:
        \[G = \langle\, y, a \mid a^6 = y^2 = 1,\ a^{-1}ay = a^{-1}\,\rangle
        \cong D_{12}\]
\end{enumerate}

So a group \(G\) of order 12 is isomorphic to one of:
\[
    C_{12}, \quad%
    C_2 \times C_6, \quad%
    A_4, \quad%
    D_{12}, \quad \text{or} \quad%
    \langle\, x, y \mid x^3 = y^4 = 1,\ y^{-1}xy = x^{-1}\,\rangle
\]


\section{Groups of Order \(p^2q\)}
Let \(p\) and \(q\) be distinct prime numbers, and \(G\) be a group of order \(p^2q\).
We shall consider the cases \(p < q\) and \(p > q\) separately.

\subsection{\(p < q\)}
Let \(p < q\), and let \(n_q\) denote the number of Sylow \(q\)-subgroups.
Then by Theorem~\ref{thm:sylow3}:
\[n_q \mid p^2\]
so \(n_q\) could be \(1\), \(p\) or \(p^2\).
Also:
\[n_q \equiv 1 \ \text{mod \(q\)}\]
If \(n_q  = p\) then \(p\) must be congruent to 1 mod \(q\), which is a contradiction since \(p < q\).
If \(n_q = p^2\) then we must have:
\[q \mid (p^2 - 1)\]
Factorising gives:
\[q \mid (p+1)(p-1)\]
So either:
\[q \mid (p+1) \quad \text{or} \quad q \mid (p-1) \quad \text{or both}\]
However, \(q\) cannot divide \(p-1\) because \(p < q\) so \(q\) must divide \(p+1\).
This is only the case when \(p = 2\) and \(q = 3\), so \(|G| = 12\) which we have already classified.

Hence if \(|G| = p^2 q \neq 12\) with \(p < q\), then \(G\) possesses a unique Sylow \(q\)-subgroup, \(Q \cong C_q\), which is
normal in \(G\).
A Sylow \(p\)-subgroup, \(P \leqslant G\), will have order \(p^2\) and by Lagrange's Theorem, intersects trivially with
\(Q\).
And by applying Lemma~\ref{lem:setprodorder}:
\[|PQ| = |P| \cdot |Q| = p^2 q\]
So we can conclude that \(G  = Q \rtimes P\).

We know, by Lemma~\ref{lem:aut}, that \(\Aut{Q} \cong \units{q}\), so we want a homomorphism, \(\psi:P \to \Aut{Q}\).
We have two possibilities for a group of order \(p^2\):

\begin{enumerate}[\bfseries{Case} 1:]
    \item \(P \cong C_{p^2}\) i.e. \(G \cong C_q \rtimes C_{p^2}\).
    \item \(P \cong C_p \times C_p\) i.e. \(G \cong C_q \rtimes (C_p \times C_p)\).
\end{enumerate}

\section{Groups of Order 30}
Let \(G\) be a group of order \(30 = 2 \cdot 3 \cdot 5\), and let \(n_3\) and \(n_5\) denote the number of Sylow
3-subgroups and Sylow 5-subgroups of \(G\) respectively.
Then by Theorem~\ref{thm:sylow3}:
\[n_3 = 1 \ \text{or} \ 10 \quad \text{and} \quad n_5 = 1 \ \text{or} \ 6\]
If \(n_3 = 10\), then there are 20 elements of order 3, and if \(n_5 = 6\) then
there are 24 elements of order 5 in \(G\).
\(G\) only has 30 elements, so then either:
\[n_3 = 1 \ \text{and} \ n_5 = 6, \quad n_3 = 10 \ \text{and} \ n_5 = 1 \quad \text{or} \quad n_3 = n_5 = 1\]
So if \(T\) is a Sylow 3-subgroup of \(G\) and \(F\) is a Sylow 5-subgroup, then at least one must be normal in \(G\).
So \(T \nrmsgp G\) or \(F \nrmsgp G\) or both.

Let \(H = TF\) and by Lagrange's Theorem, \(T \cap F = \bm{1}\), hence \(|H| = 15\) by Lemma~\ref{lem:setprodorder}.
We know from our classification of groups of order \(pq\) that \(H \cong C_{15}\).
Notice that a Sylow 2-subgroup \(K \leqslant G\) has order 2, so \(K \cong C_2\).
By the same argument as above, \(H \cap K = \bm{1}\) and \(|HK| = 30\).
Hence \(G = HK\).

Because \(|H| = 15 = \frac{30}{2}\), the index of \(H\) in \(G\) is 2, and we know a subgroup of index 2 is normal, so
\(H \nrmsgp G\).
Moreover, \(G = H \rtimes K\).

By Lemma~\ref{lem:aut}:
\[\Aut{C_{15}} = \units{15} \cong {(\Zn{3} \times \Zn{5})}^\times \cong \units{3} \times \units{5} \cong C_2 \times C_4\]

Let \(\langle x, y \rangle = C_2 \times C_4\).
A homomorphism, \(\psi:C_2 \to C_2 \times C_4\) preserves element order, and there are 3 elements of order 2 in \(C_2
\times C_4\): \((x, 1)\), \((1, y^2)\) and \((x, y^2)\).
We know \(\psi\) is determined by it's effect on a generator, so if \(\langle z \rangle = K\) then \(z\psi\) has four
possibilities:

\begin{enumerate}[\bfseries{Case} 1:]
    \item \(z\psi = (1, 1)\).

        When \(z\psi = (1, 1)\), then \(\psi\) is the trivial homomorphism, and
        so we obtain:
        \[G \cong C_2 \times C_{15} \cong C_{30}\]

    \item \(z\psi = (x, 1)\).
    \item \(z\psi = (1, y^2)\).
    \item \(z\psi = (x, y^2)\).
\end{enumerate}

% TODO: Either classify these as given, or provide the groups and prove they
% are pairwise non-isom.

\part{To Do}

\section{Groups of Order \(p^3\)}
\subsection{Groups of Order 8}
\subsection{Groups of Order 27}
\subsection{General Case?}

\section{Groups of Order 24}

\section{Groups of Order 16}

\end{document}
