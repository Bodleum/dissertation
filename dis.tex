\documentclass[a4paper, oneside, 12pt, final]{article}

\title{Classification of Finite Groups}
\author{Daniel Laing}
\date{\today}

\usepackage[left=1.5cm, right=1.5cm, top=1.5cm, bottom=2cm]{geometry}
\usepackage{amsmath, amssymb, amsthm}
\usepackage{bm} % Bold maths
\usepackage[T1]{fontenc}
\usepackage{hyperref}
\usepackage{multicol}
\usepackage{tabularx}
\usepackage[super]{nth}
\usepackage{enumerate}
\usepackage{lipsum}
\usepackage{stmaryrd}   % for \lightning

% Maths in section headings
\makeatletter
\pdfstringdefDisableCommands{\let\(\fake@math}
\newcommand\fake@math{}% just for safety
\def\fake@math#1\){\detokenize{#1}}
\makeatother

\newtheorem{theorem}{Theorem}[section]
\newtheorem{corollary}{Corollary}[theorem]
\newtheorem{lemma}[theorem]{Lemma}
\theoremstyle{definition}
\newtheorem{definition}[theorem]{Definition}

\DeclareMathOperator{\Aut}{Aut}
\DeclareMathOperator{\order}{o}
\DeclareMathOperator{\lcm}{lcm}
\DeclareMathOperator{\z}{Z}
\DeclareMathOperator{\C}{C}

\newcommand{\Z}{\mathbb{Z}}
\newcommand{\Zn}[1]{\Z/#1\Z}
\newcommand{\F}{\mathbb{F}}
\newcommand{\R}{\mathbb{R}}


% ====   Document   ====
\begin{document}
{\maketitle}
{\tableofcontents}

\part{Done}


% ====   Groups of order 6   ====
\section{Groups of Order 6}
Let \(G\) be a group of order 6, and \(n_3\) denote the number of Sylow 3-subgroups of \(G\).
Then by Theorem~\ref{thm:sylow3}:
\[n_3 \equiv 1 \ \text{(mod 3)} \quad \text{and} \quad n_3 \mid 2 \implies n_3 = 1\]
So \(G\) has one Sylow 3-subgroup, \(N\), and because 3 is prime, it is isomorphic to \(C_3\).
Let \(N = \langle x \rangle\).
Any Sylow 2-subgroup, \(H \leqslant G\), will have order 2, and so \(H \cong C_2\).
Let \(H = \langle y \rangle\).
Lagrange's Theorem tells us that \(N\) has elements of orders 1 and 3, and \(H\) has elements of
order 1 and 2 hence:
\[N \cap H = \bm{1}\]

By Lemma~\ref{lem:setprodorder}:
\[|NH| = \frac{|N| \cdot |H|}{|N \cap H|} = \frac{3 \cdot 2}{1} = 6\]
So \(G = NH\), \(N \unlhd G\) and \(N \cap H = \bm{1}\), which means \(G = N \rtimes H\), the
semidirect product of \(N\) by \(H\).

Now we need to determine \(\Aut{N}\).
An automorphism, \(\varphi\) of \(N\) preserves element order.
In particular, \(\varphi\) maps generators to generators.
Hence, \(x\varphi = x\) or \(x^2\) because they are the generators of \(N\).
So \(\Aut{N} \cong C_2\).

Now we want a homomorphism \(\psi:H \to \Aut{N}\).
If \(\psi\) is trivial, then it maps \(H\) to the trivial group, so every element of \(H\) gets sent
to the trivial automorphism.
If \(\psi\) is not trivial, then at least one element of \(H\) is not sent to the trivial
automorphism.
It cannot be 1 because then element order is not preserved, so it must be the generator, \(y\).
Hence we obtain 2 possibilities for \(G\):

\begin{enumerate}[\bfseries{Case} 1:]
    \item
        \begin{equation*}
        \begin{aligned}
            G &= \langle\, x, y \mid x^3 = y^2 = 1,\ y^{-1}xy = x \,\rangle \\
            &=\langle\, x, y \mid x^3 = y^2 = 1,\ xy = yx \,\rangle \\
            &= C_3 \times C_2 \cong C_6
        \end{aligned}
        \end{equation*}

    \item
        \begin{equation*}
        \begin{aligned}
            G &= \langle\, x, y \mid x^3 = y^2 = 1,\ y^{-1}xy = x^{-1}
            \,\rangle \\
            &\cong D_6
        \end{aligned}
        \end{equation*}
\end{enumerate}

These are clearly not isomorphic, because \(C_6\) is abelian, and \(D_6\) is not.

Hence \(G\) is isomorphic one of:
\[C_6 \quad \text{or} \quad D_6\]

% ====   Groups of order 2p   ====
\section{Generalisation to Groups of Order \(2p\)}
Now that we have seen groups of order 6, let's try and work towards a more general case: groups of
order 2 times a prime number.
So let \(G\) be a group of order \(2p\) where \(p\) is a prime number, and \(n_p\) denote the number
of Sylow p-subgroups of \(G\).
Then by Theorem~\ref{thm:sylow3}:
\[n_p \equiv 1 \ \text{(mod \(p\))} \quad \text{and} \quad n_p \mid 2 \implies n_p = 1\]
So \(G\) has one Sylow \(p\)-subgroup, say \(N\), and it is isomorphic to \(C_p\).
Let \(N = \langle x \rangle\).
A Sylow 2-subgroup, \(H \leqslant G\) will have order 2 so \(H \cong C_2\).
Let \(H = \langle y \rangle\).
Lagrange's Theorem tells us that \(N\) has elements of orders 1 and \(p\), and \(H\) has elements of
order 1 and 2 hence:
\[N \cap H = \bm{1}\]

By Lemma~\ref{lem:setprodorder}:
\[|NH| = \frac{|N| \cdot |H|}{|N \cap H|} = \frac{p \cdot 2}{1} = 2p\]
So \(G = N \rtimes H\) as before.

We know by Lemma~\ref{lem:aut} that \(\Aut{N} \cong \Zn{p}^*\), so let's look for the elements of
order 2.
An element \(x \in \Zn{p}^*\) of order 2 satisfies \(x^2 = 1\), hence \(x = 1\) or \(-1\).
But 1 has order 1, so \(x\) can only be \(-1\).
From the proof of Lemma~\ref{lem:aut}, this corresponds to the inverse map \(\beta:x \mapsto
x^{-1}\).

Now we want a homomorphism \(\psi: H \to \Aut{N}\).
By the same argument as for groups of order 6, we have two possibilities for \(G\):

\begin{enumerate}[\bfseries{Case} 1:]
    \item
        \begin{equation*}
        \begin{aligned}
            G &= \langle\, x, y \mid x^p = y^2 = 1,\ y^{-1}xy = x \,\rangle \\
            &= C_p \times C_2 \cong C_{2p}
        \end{aligned}
        \end{equation*}
    \item
        \begin{equation*}
        \begin{aligned}
            G &= \langle\, x, y \mid x^p = y^2 = 1,\ y^{-1}xy = x^{-1}
            \,\rangle \\
            &\cong D_{2p}
        \end{aligned}
        \end{equation*}
\end{enumerate}

Again, these are clearly not isomorphic, because \(C_{2p}\) is abelian, and \(D_{2p}\) is not.

Hence a group of order \(2p\) is isomorphic to one of:
\[C_{2p} \quad \text{or} \quad D_{2p}\]

% ====   Groups of order 4   ====
\section{Groups of order 4}
From the Fundamental Theorem of Finite Abelian Groups, we know that a group \(G\) of order 4 could
be isomorphic to one of \(C_4\) and \(C_2 \times C_2\).
Now to show that these are the only possibilities.
The Sylow theorems are not so helpful here, because any Sylow 2-subgroup will be of order 4, which
is just G.
Lagrange's Theorem tells us every element of \(G\) has order 1, 2 or 4.

If \(x \in G\) has order 4, then \(x\) generates G so \(G \cong C_4\).

If instead there is no element of order 4 in \(G\), then every \(x \in G\) except the identity is of
order 2.
Consider \(a, b \in G\) with \(a \neq b\), and their product, \(ab\).
It must be that \(ab\) is the third element of order 2, otherwise we reach a contradiction.
So it is easy to see that \(G \cong C_2 \times C_2\).

So any group of order 4 is isomorphic to one of:
\[C_4 \quad \text{or} \quad C_2 \times C_2\]

% ====   Groups of order p^2   ====
\section{Generalisation to Groups of Order \(p^2\)}
Let \(G\) be a group of order \(p^2\).
By Lagrange's Theorem, the elements of \(G\) have order 1, \(p\) or \(p^2\).

If \(x \in G\) has order \(p^2\), then \(x\) generates \(G\) so \(G \cong C_{p^2}\).

If \(G\) does not have an element of order \(p^2\) then all elements, except the identity, have order \(p\).
We know that \(G\) must have a subgroup of order \(p\), \(P\), and because \(p\) is prime, \(P \cong C_p\).
Pick a generator for \(P\), say \(x\) and an element \(y \in G\) such that \(y \notin P\).
Then \(y \neq x^i\) for any \(i\).

If \(y^j = x^i\) for some \(i\) and \(j\), then:
\[{(y^j)}^{1-j} = {(x^i)}^{1-j} = y^{j-j+1} = y = x^{i(1-j)} = x^k \quad \text{for some} \ k\text{,} \ \text{a
contradiction.}\]
So no power of \(y\) is equal to any power of \(x\).
Because \(y\) has order \(p\), it generates a subgroup of order \(p\), \(\bar{P}\) with \(P \cap \bar{P} = \bm{1}\).
By Lemma~\ref{lem:setprodorder}, \(|P\bar{P}| = p^2 = |G|\) so:
\[G = P \times \bar{P} \cong C_p \times C_p\]

If \(G\) has no elements of order \(p\) or \(p^2\), then it only has elements of order 1, which is the trivial group.

Hence any group of order \(p^2\) is isomorphic to one of:
\[C_{p^2} \quad \text{or} \quad C_p \times C_p\]


% TODO: Check the above proof is actually legit

% Let \(G\) be a group of order \(p^2\) and consider \(\z{(G)} \unlhd G\).
% By Lagrange's Theorem, \(\z{(G)}\) has order 1, \(p\) or \(p^2\).
%
% If \(|\z{(G)}| = p^2\) then \(G\) is abelian.
%
% Assume \(|\z{(G)}| \neq p^2\) and consider an element \(x \in G\) but \(x
% \notin \z{(G)}\), and it's centraliser, \(\C_G{(x)}\).
% We know \(\C_G{(x)} \leqslant G\) and that \(x \in \C_G{(x)}\), so \(|\C_G{(x)|
% \neq 1}\), and so by Lagrange's Theorem, it must be that \(|\C_G{(x)}| = p\).
% So:
% \[|x^G| = |G:\C_G{(x)}| = \frac{p^2}{p} = p\]
%
% The Class Equation, \(|G| = |\z{(G)}| + \sum^k_{i=1}|x_i^G|\), tells us
% \(|\z{(G)}|\) must be a multiple of \(p\) because both  \(|G|\) and \(|x^G|\)
% are multiples of \(p\). Hence \(|\z{(G)}| = p\).
%
% So then \(|G:\z{(G)}| = p\), which means \(G/\z{(G)} \cong C_p\).
%
% % TODO: Finish proof
% % TODO: Tidy it up
%
% \vskip 3em
% \paragraph{Sketch}
% \begin{itemize}
%     \item Show \(G\) must be abelian. Result follows from FTFAB\@.
%     \item \(\z{(G)}\) has order 1, \(p\), or \(p^2\) by Lagrange.
%     \item If \(p^2\) then done.
%     \item Size of congruencty classies is multiple of p.
%     \item Class eqn => order of centre is multiple of p. (and so is not 1)
%     \item Quotient with G is cyclic.
%     \item MT4003 showed then G must be abelian.
% \end{itemize}

% ====   Groups of order pq   ====
\section{Groups of Order \(pq\)}
Let \(G\) be a group of order \(pq\) where \(p\), \(q\) are prime numbers with
\(p > q\), and let \(n_p\) and \(n_q\) denote the number of Sylow p-subgroups
and Sylow q-subgroups of \(G\) respectively.
Then by Theorem~\ref{thm:sylow3}:
\[n_p \equiv 1 \ \text{(mod \(p\))} \quad \text{and} \quad n_p \mid q \implies n_p = 1\]
\[n_q \equiv 1 \ \text{(mod \(q\))} \implies n_q = 1,\ q+1,\ 2q+1,\ \dots \quad \text{and} \quad n_q \mid p\]

So \(C_p \unlhd G\) and we have 2 possibilities for \(C_q\): \(p-1\) is a
multiple of \(q\) or 1.

Let \(\langle x \rangle = C_p\) and \(\langle y \rangle = C_q\).

\begin{enumerate}[\bfseries{Case} 1:]
    \item \(q \nmid p - 1\).

        If \(p-1\) \emph{is not} a multiple of \(q\) then \(n_q = 1\) and \(C_q
        \unlhd G\), hence: \[G = C_p \times C_q \cong C_{pq}\]

    \item \( q \mid p - 1\).

        If \(p-1\) \emph{is}  a multiple of \(q\) then \(n_q = p\) and so \(C_q
        \leqslant G\).
        By Lagrange's Theorem, \(C_p \cap C_q = \bm{1}\) and by Lemma~\ref{lem:setprodorder},
        \(|C_p C_q| = pq\), hence, as well as the direct product, we have \(G = C_p \rtimes C_q\).

        By Lemma~\ref{lem:aut}, \(\Aut{C_p} \cong \Zn{p}^* \cong C_{p-1}\), so
        if \(\nu \in \Zn{p}^*\), then \(x \mapsto x^\nu\) is an automorphism.
        We know also that \(C_{p-1}\) has a unique subgroup of order \(q\),
        hence \(G\) has the presentation:

        \[G = \langle \, x, y, \mid x^p = y^q = 1,\ y^{-1}xy = x^\alpha \, \rangle\]
        where \(\alpha\) is a generator for the subgroup of order \(q\) in
        \(\Zn{p}^*\).

        Notice that picking different generators are equivalent up to
        isomorphism.

\end{enumerate}

So any group of order \(pq\) is isomorphic to either:
\begin{equation*}
\begin{aligned}
    C_{pq} \quad \text{or} \quad \langle \, x, y \mid x^p = y^q = 1,\ y^{-1}xy
    = x^\alpha \, \rangle \qquad &\text{if} \ q \mid p-1 \\
    C_{pq} \qquad &\text{if} \ q \nmid p-1
\end{aligned}
\end{equation*}



\part{In Progress}

% ====   Theorems and Lemmas   ====
\section{Theorems and Lemmas}
\subsection{Sylow Theorems}
Let \(G\) be a group of order \(p^n m\) where \(p\) is a prime and \(p\nmid m\).
\begin{theorem}[\nth{1} Sylow Theorem]\label{thm:sylow1}
    \(G\) has a Sylow p-subgroup, i.e.\ a subgroup of order \(p^n\).
\end{theorem}
\begin{theorem}[\nth{2} Sylow Theorem]\label{thm:sylow2}
    All Sylow p-subgroups of \(G\) are conjugate to each other.
\end{theorem}
\begin{corollary}
    If \(n_p = 1\) then the Sylow p-subgroup is normal in \(G\).
\end{corollary}
\begin{theorem}[\nth{3} Sylow Theorem]\label{thm:sylow3}
    Let \(n_p\) denote the number of Sylow p-subgroups of \(G\).
    Then:
    \begin{enumerate}[(i)]
        \item \(n_p \mid m\)
        \item \(n_p\equiv 1\) (mod \(p\))
    \end{enumerate}
\end{theorem}

% TODO: Write out isom theorems
\subsection{Isomorphism Theorems}
\begin{theorem}\label{thm:iso1}
\end{theorem}

\begin{theorem}\label{thm:iso2}
\end{theorem}

\begin{theorem}\label{thm:iso3}
\end{theorem}

\begin{lemma}\label{lem:setprodorder}
    For a group \(G\) with \(N \leqslant G\) and \(H \leqslant G\), then
    \[|NH| = |\{nh \mid n \in N, h \in H\}| = \frac{|N| \cdot |H|}{|N \cap H|}\]
\end{lemma}
% TODO: maybe prove?

\begin{lemma}\label{lem:aut}
    The automorphism group of \(C_n\) is isomorphic to the multiplicative group
    of integers mod \(n\).

    i.e. \(\Aut{C_n} \cong \Zn{n}^*\)
\end{lemma}

\begin{proof}
    Let \(C_n = \langle x \rangle\).
    Any automorphism, \(\varphi\) of \(C_n\) has the property:
    \[(x^i)\varphi = {(x\varphi)}^i\]
    Hence \(\varphi\) is determined by it's effect on a generator, \(x\), and preserves element
    order.
    In particular, \(\varphi\) sends generators to generators.
    So for \(\varphi\) to be an automorphism, it must send \(x\) to another generator, say \(x^k\).
    An element \(x^k\) generates \(C_n\) if \(x^k\) has order \(n\), i.e.\ when \(k\) and \(n\) are co-prime.
    Denote the automorphism sending \(x\) to \(x^k\) by \(\varphi_k\).

    Let's now investigate how these automorphisms behave.
    Consider:
    \[x\varphi_k\varphi_l = (x^k)\varphi_l = {(x^k)}^l = x^{(kl)} = x\varphi_{kl} \quad \text{(mod \(n\))}\]
    Because multiplication modulo \(n\) is commutative, \(x^{kl} = x^{lk}\), so \(\Aut{C_n}\) is abelian.

    Now consider \(\theta:\Aut{C_n} \to \Zn{n}^*\) defined by \(\varphi_k\theta = k\).
    We will show \(\theta\) is an isomorphism.
    Every \(k \in \Zn{n}^*\) is co-prime to \(n\) and so \(x^k\) is a generator of \(C_n\), hence there is some \(\varphi_k
    \in \Aut{C_n}\) such that \(\varphi_k\theta = k\).
    So \(\theta\) is surjective.
    If \(\varphi_k, \varphi_l \in \Aut{C_n}\) such that \(\varphi_k\theta = \varphi_l\theta\) then \(k = l\), so
    \(\theta\) is also injective.
    Finally, \(\theta\) is a homomorphism because
    \[(\varphi_k\varphi_l)\theta = \varphi_{kl}\theta = kl = (\varphi_k\theta)(\varphi_l\theta)\]
    So \(\theta:\Aut{C_n} \to \Zn{n}^*\) is an isomorphism.
\end{proof}



% ====   Groups of order 12   ====
\section{Groups of order 12}
Let \(G\) be a group of order \(12 = 2^2 \cdot 3\), and \(n_3\) and \(n_2\)
denote the number of Sylow 3-subgroups and Sylow 2-subgroups of G respectively.
By Theorem~\ref{thm:sylow3}:
\[n_2 \equiv 1 \text{ (mod 2) and } n_2 \mid 3 \implies n_2 = 1\]
\[n_3 \equiv 1 \text{ (mod 3) and } n_3 \mid 4 \implies n_3 = 1 \ \text{or} \ 4\]

\(G\) has a unique Sylow 2-subgroup of order \(2^2 = 4\), say \(H \unlhd G\),
and we have already classified groups of order 4, so either \(C_4\) or \(V_4
\unlhd G\).
A Sylow 3-subgroup of \(G\) will have order 3, so  \(C_3 \leqslant G\), and for
some groups, \(C_3 \unlhd G\).

Lagrange's Theorem tells us \(H\) has elements of order 1, 2, and 4, and
\(C_3\) has elements of order 1 and 3.
Hence \(H \cap C_3 = \bm{1}\).

Lemma~\ref{lem:setprodorder} tells us:
\[|H C_3| = \frac{|H| \cdot |C_3|}{|H \cap C_3|} = 12\]

Hence \(G = H C_3\), \(C_3 \leqslant G\), \(H \unlhd G\), and \(H \cap C_3 =
\bm{1} \implies G = H \rtimes C_3\).

Since an automorphism, \(\varphi\), must map generators to generators,
\(\Aut{C_4} \cong C_2\) because the generators of \(C_4\) are \(x\) and
\(x^{-1}\).
An automorphism of \(V_4\) corresponds to a permutation of the three
non-identity elements, hence \(\Aut{V_4} \cong S_3\).

\begin{enumerate}[\bfseries{Case} 1:]
    \item \(H = C_4\) i.e. \(G = C_4 \rtimes C_3\).

        A homomorphism \(\psi:C_3 \to \Aut{C_4} \cong C_2\), preserves order
        and together with Lagrange's Theorem means that the only possibility
        for \(\psi\) is trivial, i.e. \(C_3\psi = \bm{1}\).

        Hence \(G \cong C_4 \times C_3 \cong C_{12}\).

    \item \(H = V_4\) i.e. \(G = (C_2 \times C_2) \rtimes C_3\).

        A trivial homomorphism \(C_3\psi = \bm{1}\) yields the direct product
        \(G \cong C_2 \times C_2 \times C_3 \cong C_2 \times C_6\).

        \(S_3\) has one subgroup of order 3, hence there is essentially only
        one homomorphism \(\psi:C_3 \to \Aut{V_4}\).

        % TODO: Show this is A4
        Still need to show this is \(A_4\).
\end{enumerate}

If we instead consider \(G\) where \(C_3 \unlhd G\), i.e. \(G = C_3 \rtimes
H\), then we again have two cases:

\begin{enumerate}[\bfseries{Case} 1:]
    \item \(H = C_4\) i.e. \(G = C_3 \rtimes C_4\).

        Say \(C_3 = \langle x \rangle\) and \(C_4 = \langle y \rangle\).
        We know \(\Aut{C_3} \cong C_2\) so a homomorphism \(\psi\) maps \(C_4\)
        to the trivial group, \(\bm{1}\) or to \(\langle \beta:x \mapsto x^{-1}
        \rangle\).

        If \(C_4\psi = \bm{1}\) then \(G = C_3 \times C_4 \cong C_4 \times
        C_3\), which we have already seen.

        If \(C_4\psi = \langle \beta \rangle\) then we have:
        \[G = \langle\, x, y \mid x^3 = y^4 = 1,\ y^{-1}xy = x^{-1}\,\rangle\]
        % TODO: Classify this

    \item \(H = V_4\) i.e. \(G = C_3 \rtimes (C_2 \times C_2)\).

        If \(\psi:(C_2 \times C_2) \to \Aut{C_3}\) is trivial then we obtain
        \(G = C_3 \times C_2 \times C_2 \cong C_2 \times C_6\) which we have
        seen before.

        The image of a non-trivial homomorphism \(\psi:(C_2 \times C_2) \to
        \Aut{C_3}\) is \(C_2\), so by Theorem~\ref{thm:iso1}: \(\ker{\theta} =
        C_2\).

        Choose \(a, b \in C_2 \times C_2\) with \(a, b \neq 1\) such that
        \(a\theta = \beta:x \mapsto x^{-1}\) and \(b\theta = \text{id}:x
        \mapsto x\).
        Then:
        \[G = \langle\, x, a, b \mid x^3 = a^2 = b^2 = 1,\ ab = ba,\ a^{-1}xa =
        x^{-1},\ b^{-1}xb = x\,\rangle\]
        Let \(y = xb\).
        The order of \(y = \lcm{(\order{(x)}, \order{(b)})} = \lcm{(2, 3)} =
        6\) because \(x\) and \(b\) commute.
        \(y^3 = x^3b^3 = b\) so:
        \[ a^{-1}ya = a^{-1}xba = a^{-1}xab = x^{-1}b = x^2b = y^2y^3 = y^{-1}\]
        Hence:
        \[G = \langle\, a, y \mid y^6 = a^2 = 1,\ a^{-1}ya = y^{-1}\,\rangle
        \cong D_{12}\]
\end{enumerate}

So a group \(G\) of order 12 is isomorphic to one of:
\[
    C_{12}, \quad%
    C_2 \times C_6, \quad%
    A_4, \quad%
    D_{12}, \quad \text{or} \quad%
    \langle\, x, y \mid x^3 = y^4 = 1,\ y^{-1}xy = x^{-1}\,\rangle
\]

\section{Generalisation to Groups of order \(4p\)}
Suppose \(G\) is a group of order \(4p\) where \(p\) is a prime number.
Let \(n_2\) denote the number of Sylow 2-subgroups.

% TODO: Sketch

\section{Groups of Order 30}
Let \(G\) be a group of order \(30 = 2 \cdot 3 \cdot 5\), and let \(n_3\) and
\(n_5\) denote the number of Sylow 3-subgroups and Sylow 5-subgroups of \(G\)
respectively.
Then by Theorem~\ref{thm:sylow3}:
\[n_3 = 1 \ \text{or} \ 10 \quad \text{and} \quad n_5 = 1 \ \text{or} \ 6\]
If \(n_3 = 10\), then there are 20 elements of order 3, and if \(n_5 = 6\) then
there are 24 elements of order 5 in \(G\).
\(G\) only has 30 elements, so then either:
\[n_3 = 1 \ \text{and} \ n_5 = 6, \quad n_3 = 10 \ \text{and} \ n_5 = 1 \quad \text{or} \quad n_3 = n_5 = 1\]
Hence either \(C_3 \unlhd G\) or \(C_5 \unlhd G\).

Let \(H = C_3C_5\) and by Lagrange's Theorem, \(C_3 \cap C_5 = \bm{1}\), hence
\(|H| = 15\) by Lemma~\ref{lem:setprodorder}.
We know from our classification of groups of order \(pq\) that \(H \cong C_{15}\).
Notice that \(C_2\) is a Sylow 2-subgroup of \(G\), and by the same argument,
\(C_2 \cap C_{15} = \bm{1}\) and \(|C_2C_{15}| = 30\).
Hence \(G = C_2C_{15}\).

Because \(|C_{15}| = 15 = \frac{30}{2}\), the index of \(C_{15}\) in \(G\) is
2, and we know a subgroup of index 2 is normal, so \(C_{15} \unlhd G\).
Moreover, \(G = C_{15} \rtimes C_2\).

By Lemma~\ref{lem:aut}:
\[\Aut{C_{15}} = \Zn{15}^* \cong {(\Zn{3} \times \Zn{5})}^* \cong \Zn{3}^* \times \Zn{5}^* \cong C_2 \times C_4\]

A homomorphism, \(\psi:C_2 \to C_2 \times C_4\) preserves element order, and
there are 3 elements of order 2 in \(C_2 \times C_4\): \((x, 1)\), \((1, y^2)\)
and \((x, y^2)\) where \(\langle x, y \rangle = C_2 \times C_4\).
We know \(\psi\) is determined by it's effect on a generator, so if \(\langle z
\rangle = C_2\) then \(z\psi\) has four possibilities:

\begin{enumerate}[\bfseries{Case} 1:]
    \item \(z\psi = (1, 1)\).

        When \(z\psi = (1, 1)\), then \(\psi\) is the trivial homomorphism, and
        so we obtain:
        \[G \cong C_2 \times C_{15} \cong C_{30}\]

    \item \(z\psi = (x, 1)\).
    \item \(z\psi = (1, y^2)\).
    \item \(z\psi = (x, y^2)\).
\end{enumerate}

% TODO: Either classify these as given, or provide the groups and prove they
% are pairwise non-isom.

\part{To Do}

\section{Semi-Direct Product}
% TODO: Semi-direct product
% \section{Semi-Direct Product}
% We already know about the direct product:
%
% \begin{definition}[Direct Product]
%     For groups \(N\) and \(H\), the \emph{direct product}, \(G = N\times H\) is a
%     group of ordered pairs of elements \((n, h)\) where \(n \in N\) and \(h \in
%     H\) with the operation: \[(n_1, h_1)(n_2, h_2) = (n_1n_2, h_1h_2)\]
% \end{definition}
%
% \begin{theorem}
%     If \(G = N \times H\) then \(N \unlhd G\) and \(H \unlhd G\).
% \end{theorem}
%
% TODO: Prove this
% \begin{proof}
%     \lipsum[1]
% \end{proof}
%
% But what if we allow \(n_2\) to change, depending on what \(h_1\) is?
% Let's define a homomorphism \(\varphi:H \to \Aut{N}\) and

% ====   Groups of order 9   ====
\section{Groups of order 9 (Might skip)}



\section{Groups of Order 18}
\subsection{Groups of Order \(p^2q\)}

\section{Groups of Order \(p^3\)}
\subsection{Groups of Order 8}
\subsection{Groups of Order 27}
\subsection{General Case?}

\section{Groups of Order 24}

\section{Groups of Order 16}

\end{document}
