\documentclass[a4paper, oneside, 12pt, final]{article}

\title{Interim Report}
\author{Daniel Laing}
\date{\today}

\usepackage[left=1.5cm, right=1.5cm, top=1.5cm, bottom=2cm]{geometry}
\usepackage{amsmath, amssymb, amsthm}
\usepackage{bm} % Bold maths
\usepackage[T1]{fontenc}
\usepackage[utf8]{inputenc}
\usepackage{hyperref}
\usepackage{multicol}
\usepackage{tabularx}
\usepackage[super]{nth}
\usepackage{enumerate}
\usepackage{lipsum}

\newtheorem{theorem}{Theorem}[section]
\newtheorem{corollary}{Corollary}[theorem]
\newtheorem{lemma}[theorem]{Lemma}
\theoremstyle{definition}
\newtheorem{definition}[theorem]{Definition}

\DeclareMathOperator{\Aut}{Aut}

\newcommand{\Z}{\mathbb{Z}}
\newcommand{\Zn}[1]{\Z/#1\Z}

% ====   Document   ====
\begin{document}
{\maketitle}
{\tableofcontents}

% \part{Introduction and aims}
% \lipsum[2]

% \section{Semi-Direct Product}
% We already know about the direct product:
%
% \begin{definition}[Direct Product]
%     For groups \(N\) and \(H\), the \emph{direct product}, \(G = N\times H\) is a
%     group of ordered pairs of elements \((n, h)\) where \(n \in N\) and \(h \in
%     H\) with the operation: \[(n_1, h_1)(n_2, h_2) = (n_1n_2, h_1h_2)\]
% \end{definition}
%
% \begin{theorem}
%     If \(G = N \times H\) then \(N \unlhd G\) and \(H \unlhd G\).
% \end{theorem}
%
% % TODO: Prove this
% \begin{proof}
%     \lipsum[1]
% \end{proof}
%
% But what if we allow \(n_2\) to change, depending on what \(h_1\) is?
% Let's define a homomorphism \(\varphi:H \to \Aut{N}\) and ...

% ====   Theorems and Lemmas   ====
\section{Theorems and Lemmas}
\subsection{Sylow Theorems}
Let \(G\) be a group of order \(p^nm\) where \(p\) is a prime and \(p\nmid m\).
\begin{theorem}[\nth{1} Sylow Theorem]
\label{thm:sylow1}
    \(G\) has a Sylow p-subgroup, i.e. a subgroup of order \(p^n\).
\end{theorem}
\begin{theorem}[\nth{2} Sylow Theorem]
\label{thm:sylow2}
    All Sylow p-subgroups of \(G\) are conjugate to each other.
\end{theorem}
\begin{corollary}
    If \(n_p = 1\) then the Sylow p-subgroup is normal in \(G\).
\end{corollary}
\begin{theorem}[\nth{3} Sylow Theorem]
\label{thm:sylow3}
    Let \(n_p\) denote the number of Sylow p-subgroups of \(G\).
    Then:
    \begin{enumerate}[i)]
        \item \(n_p \mid m\)
        \item \(n_p\equiv 1\) (mod p)
    \end{enumerate}
\end{theorem}

\begin{lemma}
\label{lem:setprodorder}
    For a group \(G\) with \(N \leqslant G\) and \(H \leqslant G\), then
    \[|NH| = |\{nh \mid n \in N, h \in H\}| = \frac{|N| \cdot |H|}{|N \cap H|}\]
\end{lemma}

\begin{lemma}
\label{lem:aut}
    The automorphism group of \(C_n\) is isomorphic to the multiplicative group of integers mod n.

    i.e. \(\Aut{C_n} \cong \Zn{n}^*\)
\end{lemma}

\begin{proof}
    Any automorphism, \(\varphi\) of \(C_n\) has the property:
    \[(x^i)\varphi = (x\varphi)^i\]
    Hence \(\varphi\) is determined by it's effect on the generator, \(x\), and
    preserves element order.
    In particular, \(\varphi\) sends generators to generators.

    So for a generator, \(x\), \(x\varphi = x^k\) is surjective if \(x^k\)
    generates \(C_n\).
    \(x^k\) generates \(C_n\) if \(o(x^k) = n\) which is when \(\gcd{(n, k)} =
    1\).

    Denote \(\varphi_k:x \mapsto x^k\).

    Consider:
    \[x\varphi_k\varphi_l = (x^k)\varphi_l = (x^k)^l = x^{kl} = x^{lk} = (x^l)^k = x\varphi_l\varphi_k\]
    So we see that \(\Aut{C_n}\) is abelian.
    Moreover, \(x\varphi_k\varphi_l = x\varphi_{kl}\).

    Now consider \(\theta:\Aut{C_n} \to \Zn{n}^*\) defined by \(\varphi_k\theta
    \mapsto k\).
    We will show \(\theta\) is an isomorphism.

    \(\theta\) is surjective because every \(k \in \Zn{n}^*\) is coprime to
    \(n\) and so \(x^k\) is a generator of \(C_n\), hence \(\exists\,\varphi_k
    \in \Aut{C_n}\) such that \(\varphi_k\theta = k\).

    \(\theta\) is also injective because if \(\varphi_k, \varphi_l \in
    \Aut{C_n}\) such that \(\varphi_k\theta = \varphi_l\theta\) then \(k = l\).

    Finally, \(\theta\) is a homomorphism because \((\varphi_k\varphi_l\theta =
    \varphi_{kl}\theta = kl = (\varphi_k\theta)(\varphi_l\theta)\).
    So \(\theta:\Aut{C_n} \to \Zn{n}^*\) is an isomorphism.

\end{proof}

% ====   Groups of order 6   ====
\section{Groups of order 6}
The prime factorisation of \(6 = 2 \cdot 3\), so we can construct groups with
products of \(C_2\) and \(C_3\). The automorphism groups of \(C_2\),
\(\Aut{C_2} = \{id\}\), containing just the identity map. So any meaningful
products will look like \(C_3 \rtimes C_2\).

\(\Aut{C_3} = \{\text{id}, \psi\}\) where \(x\psi = x^{-1}\).
So we have two possible products: \(C_3 {\rtimes}_{\text{id}} C_2\) and \(C_3
{\rtimes}_{\varphi} C_2\) where \(\varphi\) is the homomorphism \(\varphi:C_2
\to \Aut{C_3}\) mapping \(1 \mapsto \text{id}\) and \(x \mapsto \psi\).

\subsection{\(C_3 \rtimes_\text{id} C_2\)}
By the Fundamental Theorem of Finite Abelian Groups, we know \( C_3
\rtimes_\text{id} C_2 \cong C_3 \times C_2 \cong C_6\).

\subsection{\(C_3 \rtimes_\varphi C_2\)}
So the group operation is
\[(a_1, b_1)(a_2, b_2) = (a_1\cdot a_2\varphi_{b_1}, b_1\cdot b_2)\]
Investigating elements:
\[(1,x)(1,x) = (1 \cdot 1\varphi_x, x \cdot x) = (1 \cdot 1^{-1}, x^2) = (1,
1)\]
So \((1, x)\) is of order 2.

\[(x, 1)(x, 1) = (x \cdot x\varphi_1, 1 \cdot 1) = (x \cdot x, 1) = (x^2, 1)\]
\[(x^2, 1)(x, 1) = (x^2 \cdot x\varphi_1, 1 \cdot 1) = (x^3, 1) = (1, 1)\]
So \((x, 1)\) is of order 3.

\[(x, 1)(x, x) = (x \cdot x\varphi_1, 1 \cdot x) = (x, x)\]
\[(x, x)(x, 1) = (x \cdot x\varphi_x, x \cdot 1) = (xx^{-1}, x) = (1, x)\]
Hence, \(C_3 \rtimes_\varphi C_2\) is non-abelian.

\section{Groups of Order 6 (Attempt 2)}
Let \(G\) be a group of order 6, and \(n_3\) denote the number of Sylow
3-subgroups of \(G\).
Then by Theorem \ref{thm:sylow3}:
\[n_3 \equiv 1 \text{ (mod 3) and } n_3 \mid 2 \implies n_3 = 1\]
So \(G\) has one Sylow 3-subgroup, and because 3 is prime, it is isomorphic to
\(C_3\), i.e.
\[C_3 \unlhd G\]

Any Sylow 2-subgroup of \(G\) will have order 2, and so \(C_2 \leqslant G\).

Lagrange's Theorem tells us that \(C_3\) has elements of orders 1 and 3, and
\(C_2\) has elements of order 1 and 2 hence:
\[C_3 \cap C_2 = \bm{1}\]

By Lemma \ref{lem:setprodorder}:
\[|C_3 C_2| = \frac{|C_3| \cdot |C_2|}{|C_3 \cap C_2|} = \frac{3 \cdot 2}{1} =
6\]

So \(G = C_3 C_2\), \(C_3 \unlhd G\) and \(C_3 \cap C_2 = \bm{1} \implies G =
C_3 \rtimes C_2\)

Now we need to determine \(\Aut{C_3}\).
\(C_3 = \{1, x, x^2 = x^{-1}\}\) and so \(\Aut{C_3} = \{\text{id}, \psi:x
\mapsto x^{-1}\} \cong C_2\).
So if \(C_3 = \langle x \rangle\) and \(C_2 = \langle y \rangle\), then we have
two possibilities for \(G\):
\begin{enumerate}[\bfseries{Case} 1:]
    \item
        \begin{equation*}
        \begin{aligned}
            G &= \langle\, x, y \mid x^3 = y^2 = 1,\ y^{-1}xy = x \,\rangle \\
            &=\langle\, x, y \mid x^3 = y^2 = 1,\ xy = yx \,\rangle \\
            &= C_3 \times C_2 \cong C_6
        \end{aligned}
        \end{equation*}
    \item
        \begin{equation*}
        \begin{aligned}
            G &= \langle\, x, y \mid x^3 = y^2 = 1,\ y^{-1}xy = x^{-1}
            \,\rangle \\
            &\cong D_6
        \end{aligned}
        \end{equation*}
\end{enumerate}

Hence \(G\) is isomorphic to either \(C_6\) or \(D_6\).

% ====   Groups of order 2p   ====
\section{Generalisation to Groups of Order \(2p\)}
Let \(G\) be a group of order \(2p\) where \(p\) is a prime number, and \(n_p\)
denote the number of Sylow p-subgroups of \(G\).
Then by Theorem \ref{thm:sylow3}:
\[n_p \equiv 1 \text{ (mod p) and } n_p \mid 2 \implies n_p = 1\]
So \(G\) has one Sylow p-subgroup, it is isomorphic to \(C_p = \langle x
\rangle\) hence:
\[C_p \unlhd G\]

A Sylow 2-subgroup of \(G\) will have order 2 so \(C_2 = \langle y \rangle
\leqslant G\).

Lagrange's Theorem tells us that \(C_p\) has elements of orders 1 and \(p\),
and \(C_2\) has elements of order 1 and 2 hence:
\[C_p \cap C_2 = \bm{1}\]

By Lemma \ref{lem:setprodorder}:
\[|C_p C_2| = \frac{|C_p| \cdot |C_2|}{|C_p \cap C_2|} = \frac{p \cdot 2}{1} =
2p\]

So \(G = C_p C_2\), \(C_p \unlhd G\) and \(C_p \cap C_2 = \bm{1} \implies G =
C_p \rtimes C_2\)

We want a homomorphism \(\varphi:\Aut{C_p} \to C_2\).
By Lemma \ref{lem:aut}, \(\Aut{C_p} \cong \Zn{p}^*\), so now we need to find elements of order 2 in \(Zn{p}^*\).

An element \(x \in \Zn{p}^*\) of order 2 satisfies:
\[x^2 = 1 \implies x^2 - 1 = 0 \implies (x-1)(x+1) = 0\]
Hence \(x = 1\) or \(-1\).
But 1 has order 1 so \(x\) can only be \(-1\).

So \(C_2\varphi\) could be \(\bm{1}\) or \(\langle \beta:x \mapsto x{-1}
\rangle\). This gives us two possibilities:
\[y\varphi = x \mapsto x \quad \text{or} \quad y\varphi = x \mapsto x^{-1}\]

\begin{enumerate}[\bfseries{Case} 1:]
    \item
        \begin{equation*}
        \begin{aligned}
            G &= \langle\, x, y \mid x^p = y^2 = 1,\ y^{-1}xy = x \,\rangle \\
            &=\langle\, x, y \mid x^p = y^2 = 1,\ xy = yx \,\rangle \\
            &= C_p \times C_2 \cong C_{2p}
        \end{aligned}
        \end{equation*}
    \item
        \begin{equation*}
        \begin{aligned}
            G &= \langle\, x, y \mid x^p = y^2 = 1,\ y^{-1}xy = x^{-1}
            \,\rangle \\
            &\cong D_{2p}
        \end{aligned}
        \end{equation*}
\end{enumerate}

Hence a group of order \(2p\) is isomorphic to \(C_{2p}\) or \(D_{2p}\).

% ====   Groups of order 4   ====
\section{Groups of order 4}
From the Fundamental Theorem of Finite Abelian Groups, we know that a group
\(G\) of order 4 could be isomorphic to one of \(C_4\) and \(C_2 \times C_2\).
Now to show that these are the only possibilities, i.e. a group of order 4 must
be abelian.

The Sylow theorems are not so helpful here, because \(4=2^2\) so any Sylow
2-subgroup will be of order 4, which is just G.

% ====   Groups of order 9   ====
\section{Groups of order 9 (Might skip)}

% ====   Groups of order p^2   ====
\section{Generalisation to Groups of Order \(p^2\)}

% ====   Groups of order 12   ====
\section{Groups of order 12}
Let \(G\) be a group of order \(12 = 2^2 \cdot 3\), and \(n_3\) and \(n_2\) denote the number of
Sylow 3-subgroups and Sylow 2-subgroups of G respectively.
By Theorem \ref{thm:sylow3}:
\[n_2 \equiv 1 \text{ (mod 2) and } n_2 \mid 3 \implies n_2 = 1\]
\[n_3 \equiv 1 \text{ (mod 3) and } n_3 \mid 4 \implies n_3 = 1 \ \text{or} \ 4\]

\(G\) has a unique Sylow 2-subgroup of order \(2^2 = 4\), say \(H \unlhd G\),
and we have already classified groups of order 4, so either \(C_4\) or \(V_4
\unlhd G\).
A Sylow 3-subgroup of \(G\) will have order 3, so  \(C_3 \leqslant G\).

Lagrange's Theorem tells us \(H\) has elements of order 1, 2, and 4, and \(C_3\) has elements of order 1 and 3.
Hence \(H \cap C_3 = \bm{1}\).

Lemma \ref{lem:setprodorder} tells us:
\[|H C_3| = \frac{|H| \cdot |C_3|}{|H \cap C_3|} = 12\]

Hence \(G = H C_3\), \(C_3 \leqslant G\), \(H \unlhd G\), and \(H \cap C_3 =
\bm{1} \implies G = H \rtimes C_3\).

Since an automorphism, \(\varphi\), must map generators to generators,
\(\Aut{C_4} \cong C_2\) because the generators of \(C_4\) are \(x\) and
\(x^{-1}\).
An automorphism of \(V_4\) corresponds to a permutation of the three
non-identity elements, hence \(\Aut{V_4} \cong S_3\).

\begin{enumerate}[\bfseries{Case} 1:]
    \item \(H = C_4\).

        A homomorphism \(\psi:C_3 \to \Aut{C_4} \cong C_2\), preserves order
        and together with Lagrange's Theorem means that the only possibility
        for \(\psi\) is trivial, i.e. \(C_3\psi = \bm{1}\).

        Hence \(G \cong C_4 \times C_3 \cong C_{12}\).

    \item \(H = V_4\).

        \(\Aut{V_4} \cong S_3\) has 2 elements of order 3, giving us 3
        posibilities for a homomorphism \(\psi:C_3 \to \Aut{V_4}\):
        \[C_3\psi = \bm{1}, \quad C_3\psi = (a, b, c) \quad \text{and} \quad
        C_3\psi = (a, c, b)\]

        If \(C_3\psi = \bm{1}\), then \(G \cong C_2 \times C_2 \times C_3 \cong
        C_2 \times C_6\).

        I don't how to do the other two cases. Something like:
        \[G = \langle\, a, b, x \mid x^3 = a^2 = b^2 = (ab)^2 = 1,\ x^{-1}ax =
        b,\ x^{-1}bx = c,\ x^{-1}cx = a\, \rangle\]
        \[G = \langle\, a, b, x \mid x^3 = a^2 = b^2 = (ab)^2 = 1,\ x^{-1}ax =
        c,\ x^{-1}bx = a,\ x^{-1}cx = b\, \rangle\]
\end{enumerate}




\end{document}
