\documentclass[a4paper, twoside, 12pt, titlepage, final]{report}

\title{Classification of Finite Groups}
\author{Daniel Laing}
\date{\today}

% For inputting files better
\newcommand{\inputfile}[1]{%
    \InputIfFileExists{#1}{}{\typeout{No file #1.}}%
}

\inputfile{./parts/preamble.tex}

% ====   Document   ====
\begin{document}
\inputfile{./parts/titlepage.tex}
\begingroup
\let\cleardoublepage\relax
\let\clearpage\relax
\begin{quote}
{\itshape%
I certify that this project report has been written by me, is a record of work carried
out by me, and is essentially different from work undertaken for any other purpose
or assessment.
}
\end{quote}
{\tableofcontents}
\endgroup

\begin{abstract}
This report covers the full classification, including proofs, of all groups of order less than or equal to 31.
In some cases, a classification of the general case is given; namely \(p\)-groups up to \(p^3\), and groups of
order \(pq\), \(4p\) and \(2p^2\), where \(p\) and \(q\) are distinct primes.
Where possible, the names of specific, and families of groups are given.

A table of the full classification is found in the appendix.
Note: not all presentations are included, specifically long ones, and those not especially relevant.
\end{abstract}

\chapter{Introduction}
\inputfile{./parts/intro.tex}

\chapter{Preliminaries}
Much of the content of \textit{MT4003} will be assumed knowledge, however this material can be found in group theory
textbooks, particularly Robinson's \textit{A Course in the Theory of Groups}.\footcite{robinson1982}
Let's move on to review some facts and theorems which will be valuable later on, as well as introduce some new concepts
and prove some new results!
% \inputfile{./parts/groups_basics.tex}
\inputfile{./parts/preliminaries.tex}


\chapter{Groups of Prime-Power Orders}\label{ch:prime_power_order}
In this chapter, we will focus on classifying groups which have order \(p\), \(p^2\) and \(p^3\), for a prime number
\(p\).
These three will be proven generally, applying to any prime.
However, we will leave the particular case of \(16 = 2^4\) for Chapter~\ref{ch:particular_order} because it is
significantly harder.
\inputfile{./parts/primes.tex}
% \inputfile{./parts/order4.tex}
\inputfile{./parts/orderpsq.tex}
\inputfile{./parts/orderpcb.tex}


\chapter{Groups of Composite Orders}\label{ch:composite_order}
Now we will classify groups which have order a composite number, focusing only on numbers which have two distinct prime
factors, \(p\) and \(q\).
Groups of order \(pq\) will be classified fully, but to attempt the same for \(p^2 q\) is beyond the scope of this
report.
Hence we will deal with groups which have order \(4q\) and \(2p^2\).
% \inputfile{./parts/order6.tex}
\inputfile{./parts/orderpq.tex}
\inputfile{./parts/order2p.tex}
\inputfile{./parts/orderpsqq.tex}


\chapter{Groups of Particular Orders}\label{ch:particular_order}
Finally, let's deal with the groups of particular order which have been missed by our generally classifications.
Namely these orders are \(12\), \(24\), \(30\), and \(16\).
\inputfile{./parts/order12.tex}
\inputfile{./parts/order24.tex}
\inputfile{./parts/order30.tex}
\inputfile{./parts/order16.tex}

{\printbibliography[heading=bibintoc]}

\chapter*{Appendix}
\addcontentsline{toc}{chapter}{Appendix}
\pagestyle{plain}
\inputfile{./parts/appendix.tex}

\end{document}
