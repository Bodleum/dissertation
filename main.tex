\documentclass[a4paper, oneside, 12pt, final]{article}

\title{Classification of Finite Groups}
\author{Daniel Laing}
\date{\today}

% For inputting files better
\newcommand{\inputfile}[1]{%
    \InputIfFileExists{#1}{}{\typeout{No file #1.}}%
}

\inputfile{./parts/preamble.tex}

% ====   Document   ====
\begin{document}
{\maketitle}
{\tableofcontents}

\section{Introduction}

% \inputfile{./parts/groups_basics.tex}
\inputfile{./parts/preliminaries.tex}


\part{Prime Power Orders}
First, we will prove a few useful lemmas:

\begin{lemma}\label{lem:index_p_normal}
    If \(G\) is a \(p\)-group (i.e.\ a group of prime power order), then every subgroup of index \(p\) is normal.
\end{lemma}

\begin{proof}
    Let \(H\) be a subgroup of \(G\), with index \(p\).
    We know kernels are normal subgroups, so we will show that \(H\) is the kernel of some homomorphism. % TODO: Add this as a fact
    Let \(\Omega\) be the set of all cosets of \(H\).
    So by definition, \(|\Omega| = p\).
    By Lemma~\ref{lem:actionhom}, there is a homomorphism:
    \[\rho:G \to S_p\]
    Let's investigate the kernel of \(\rho\).
    If we have \(x \in \ker{\rho}\), then:
    \[(H1)x = H1 = H\]
    which means \(x \in H\).
    So the kernel of \(\rho\) is \(H\).
    Hence, \(H \nrmsgp G\).
\end{proof}

\begin{lemma}\label{lem:z_non_trivial}
    If \(G\) is a group of prime power order, the centre of \(G\) is non-trivial.
\end{lemma}

\begin{proof}
    Let \(Z\) denote the centre of \(G\), and consider the action of \(G\) on itself by conjugation.
    The orbit of an element, \(g \in G\) is:
    \[g^G = \{\,x^{-1}gx \mid x \in G\,\}\]
    which is the conjugacy class of \(g\).
    So the size of each orbit divides some power of \(p\).
    In particular, the size of each orbit is divisible by \(p\).
    So then the sum of the sizes of all of the conjugacy classes is also divisible by \(p\).
    Looking at the class equation:
    \[|G| = |Z| + \sum_{i=1}^k |g_i^G|\]
    Then reducing mod \(p\) gives:
    \[|G| \equiv |Z| \mod{p}\]
    Because \(G\) is non-trivial, it follows that \(|Z| \neq 1\).
    
    % If \(|g^G| = 1\) for some element \(g\), then \(x^{-1}gx = g\) and so \(g\) is in the centre.
    % The centre is a subgroup of \(G\) and so by Lagrange's~Theorem, it must have order divisible by \(p\).
    % We know the centre is always non-empty: the identity element is certainly inside it.
\end{proof}

\begin{lemma}\label{lem:cyclic_over_centre}
    For a group \(G\) with centre \(\z(G)\).
    Then if \(G/\z(G)\) is cyclic, \(G\) is abelian.
\end{lemma}

\begin{proof}
    Let \(x \in G\) be the element such that \(x\z(G)\) generates \(G/\z(G)\).
    Then \(\gen{x,\,\z(G)}\) contains \(\z(G)\).
    Because \(G\) is the union of cosets of \(\z(G)\), then indeed \(\gen{x,\,\z(G)} = G\).
    The centraliser of \(x\) certainly contains \(x\), and every element of \(\z(G)\) also commutes with \(x\).
    Hence the centre of \(G\) is a subgroup of the centraliser of \(x\).
    The result follows by concluding:
    \[G = \gen{x,\,\z(G)} = \gen{\z(G)} = \z(G)\]
\end{proof}

Now onto the classification!
\inputfile{./parts/primes.tex}
% \inputfile{./parts/order4.tex}
\inputfile{./parts/orderpsq.tex}
\inputfile{./parts/orderpcb.tex}


\part{Composite Orders}
% \inputfile{./parts/order6.tex}
\inputfile{./parts/orderpq.tex}
\inputfile{./parts/order2p.tex}
\inputfile{./parts/orderpsqq.tex}


\part{Special Cases}
\inputfile{./parts/order12.tex}
\inputfile{./parts/order24.tex}
\inputfile{./parts/order30.tex}


\part{To Do}
\section{Groups of Order 16}

\end{document}
