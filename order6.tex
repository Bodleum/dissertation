% ====   Order 6   ====
\section{Groups of Order 6}
Let \(G\) be a group of order 6, and \(n_3\) denote the number of Sylow 3-subgroups of \(G\).
Then by Theorem~\ref{thm:sylow3}:
\[n_3 \equiv 1 \ \text{(mod 3)} \quad \text{and} \quad n_3 \mid 2 \implies n_3 = 1\]
So \(G\) has one Sylow 3-subgroup, \(N\), and because 3 is prime, it is isomorphic to \(C_3\).
Let \(N = \langle x \rangle\).
Any Sylow 2-subgroup, \(H \leqslant G\), will have order 2, and so \(H \cong C_2\).
Let \(H = \langle y \rangle\).
Lagrange's Theorem tells us that \(N\) has elements of orders 1 and 3, and \(H\) has elements of
order 1 and 2 hence:
\[N \cap H = \bm{1}\]

By Lemma~\ref{lem:setprodorder}:
\[|NH| = |N| \cdot |H| = 6\]
So \(G = NH\), \(N \nrmsgp G\) and \(N \cap H = \bm{1}\), which means \(G = N \rtimes H\), the
semidirect product of \(N\) by \(H\).

Now we need to determine \(\Aut{N}\).
An automorphism, \(\varphi\) of \(N\) preserves element order.
In particular, \(\varphi\) maps generators to generators.
Hence, \(x\varphi = x\) or \(x^2\) because they are the generators of \(N\).
So \(\Aut{N} \cong C_2\).

Now we want a homomorphism \(\psi:H \to \Aut{N}\).
If \(\psi\) is trivial, then it maps \(H\) to the trivial group, so every element of \(H\) gets sent
to the trivial automorphism.
If \(\psi\) is not trivial, then at least one element of \(H\) is not sent to the trivial
automorphism.
It cannot be 1 because then element order is not preserved, so it must be the generator, \(y\).
Hence we obtain 2 possibilities for \(G\):

\begin{enumerate}[\bfseries{Case} 1:]
    \item
        \begin{equation*}
        \begin{aligned}
            G &= \langle\, x, y \mid x^3 = y^2 = 1,\ y^{-1}xy = x \,\rangle \\
            &=\langle\, x, y \mid x^3 = y^2 = 1,\ xy = yx \,\rangle \\
            &= C_3 \times C_2 \cong C_6
        \end{aligned}
        \end{equation*}

    \item
        \begin{equation*}
        \begin{aligned}
            G &= \langle\, x, y \mid x^3 = y^2 = 1,\ y^{-1}xy = x^{-1}
            \,\rangle \\
            &\cong D_6
        \end{aligned}
        \end{equation*}
\end{enumerate}

These are clearly not isomorphic, because \(C_6\) is abelian, and \(D_6\) is not.

Hence \(G\) is isomorphic one of:
\[C_6 \quad \text{or} \quad D_6\]
