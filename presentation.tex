\documentclass[12pt]{beamer}
\usetheme{metropolis}
\usefonttheme[stillsansseriflarge]{serif}

\usepackage[]{kpfonts}
\usepackage[super]{nth}
\usepackage[T1]{fontenc}
\usepackage[]{bm}
\usepackage[]{mdframed}

% Enter expert LaTeX mode
\makeatletter
% Maths in section headings
\pdfstringdefDisableCommands{\let\(\fake@math}
\newcommand\fake@math{}% just for safety
\def\fake@math#1\){\detokenize{#1}}
\makeatother

\newcommand{\nrmsgp}{\unlhd}
\newcommand{\gen}[1]{\langle\,#1\,\rangle}

\title{Classification of Finite Groups}
\author{Daniel Laing}
\date{\nth{23} of March, 2023}
\institute{\textit{MT4599}}

\begin{document}
\frame{\titlepage}

\section{Groups of Order \(pq\)}

\begin{frame}
\frametitle{Setup}
Let \(G\) be a group of order \(pq\) where \(p\) and \(q\) are distinct prime numbers.

Without loss of generality, take \(p > q\).
\end{frame}

\begin{frame}
\frametitle{Sylow Subgroups}
Let \(n_p\) and \(n_q\) denote the number of Sylow p-subgroups and Sylow q-subgroups of \(G\) respectively.
\[n_p \equiv 1 \ \pmod{p} \quad \text{and} \quad n_p \mid q\]
So \(G\) has a unique Sylow \(p\)-subgroup, say \(P \nrmsgp G\), and at least one Sylow \(q\)-subgroup, \(Q \leqslant
    G\).
\[ P \cong C_{p}, \qquad Q \cong C_{q} \]
Pick generators: \(\gen{x} = P\) and \(\gen{y} = Q\).
\end{frame}

\begin{frame}
\frametitle{Sylow \(q\)-subgroup}
Because \(n_{q} \equiv 1 \pmod{q}\), we have:
\[ n_{q} = 1,\ q + 1,\ 2q + 1,\ \ldots \]
And:
\[ n_q \mid p\]
So we have two cases:
\[ q \nmid p-1 \qquad \text{or} \qquad q \mid p-1 \]
\end{frame}

\section{Case 1: \(q \nmid p-1\)}

\begin{frame}
\frametitle{\(q \nmid p-1\)}
Here, \(n_{q} = 1\) and so \(Q \nrmsgp G\).

So:
\[ G = P \times Q \cong C_{pq} \]
\end{frame}

\section{Case 2: \(q \mid p-1\)}
\begin{frame}
\frametitle{\(q \mid p-1\)}
That was the easy one!
Still have \(n_q = 1\), but now also the other possibilities!

By Lagrange's~Theorem: \(P \cap Q = \bm{1}\).

By Lemma in project: \(|PQ| = pq\).

So \(G = PQ\).
Hence:
\[ G = P \rtimes Q \]

This is unique!
\end{frame}

\begin{frame}
\frametitle{Presentation}
Describe \(P \rtimes Q\) by a \emph{presentation}.
\[ G = \gen{x,\ y \mid x^p = y^q = 1,\ y^{-1}xy = x^a} \]
where \(a\) is a generator for the subgroup of order \(q\) in \({(\mathbb{Z}/p\mathbb{Z})}^\times\).
\end{frame}

\begin{frame}
\frametitle{Classification}
For distinct primes \(p\) and \(q\), any group of order \(pq\) is isomorphic to one of:
\[C_{pq}\]
\[\gen{x, y \mid x^p = y^q = 1,\ y^{-1}xy = x^a} \tag*{additionally, if \(q \mid p-1\)}\]
\end{frame}


\section{Examples}

\end{document}
